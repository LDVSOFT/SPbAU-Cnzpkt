\setcounter{section}{88}
\section{} % 89
$F$ "--- первообразная $f$, если $F'=f$, существует не всегда. Для любой непрерывной есть (док-во: в след.
семестре). Теорема: если есть первообразная $F$, то все остальные "--- в точности $F+c$.
Неопределённый интеграл (пока хз про $\d x$, это типа <<скобки>>) "--- множество первообразных
(иногда пишем как $\int = F + c$).

\section{} % 90
0, $x^p$, $\frac 1x$, $a^x$, $\sin$, $\cos$, $\frac 1{\cos^2}$ ($\tg$), $\frac 1{\sin^2}$ ($\ctg$),
$\frac{1}{\sqrt{1-x^2}}$ ($\arcsin$), $\frac{1}{\sqrt{1+x^2}}$ ($\arccos$), $\frac{1}{1-x^2}$ ($\frac{\ln(1+x)-\ln(1-x)}{2}$),
$\frac{1}{\sqrt{x^2\pm1}}$ ($\ln |x+\sqrt{x^2\pm1}|$).
Определяем сумму множеств и домножение на не-ноль, с ним ок (если бы писали как $F+c$, ноль был бы ок,
не убил бы константу).

\section{} % 91
$\int f(\phi(t)) \phi'(t) \d t = F(\phi(t)) + c$, дифференциалы мы еще не умеем :(. Примеры:
выносим константу от переменной, $\frac{\ln^2}{x}$.

\section{} % 92
$\int fg' \d x = fg - \int f'g \d x$. Подставили.
