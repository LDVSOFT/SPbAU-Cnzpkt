\setcounter{section}{60}
\section{} % 61
Производная "--- предел (бывает бесконечной).
Функция дифференцируема в $x_0$, если $\exists A: f(x)=f(x_0)+A(x-x_0) + o(x-x_0)$.
Теорема: функция дифференцируема $\iff$ $f'$ существует и конечна.
Дифференциал функции в точке $x_0$ "--- \TODO.

\section{} % 62
Производная "--- тангенс угла наклона. Или скорость.

\section{} % 63
Левая и правая ($f'_-$ и $f'_+$)"--- берём односторонний предел. Если они равны,
то существует и обычная производная, равна им. Примеры: $|x|$ в нуле; кусок окружности
(производная "--- $\pm\infty$ или $\infty$, в зависимости от).

\section{} % 64
Доказательство: через представление $f(x)$ с $o(x-x_0)$ и последовательности.

\section{} % 65
Линейная комбинация, произведение, отношение (если знаменатель "--- не ноль в окрестности).

\section{} % 66
Док-во: домножили и разделили на $g(x)-g(x_0)$.

\section{} % 67
Д-во: $g=f^{-1}$, $g(f(x))=x$, $g'(f(x_0))f'(x_0)=1$, нужно еще показать дифференцируемость $g$ в точке.
Следствие: выразили $g'(y)$.

\section{} % 68
$c$, $x^p$, $a^x$, $\ln x$, тригонометрия.
