\section{} % 1
Множества строго не определяем, интуитивное определение: $x \in A$, $A \subset B$ (нестрогое),
равенство "--- два включения, пересечение, объединение, отрицание, симметрическая разность, $2^A$,
$\{x \in A \mid F(x) \}$. Не любая совокупность "--- множество, привет от Рассела и <<множество чисел,
которые можно записать при помощи 30 слов языка>>. Де Морган "--- $A\setminus\cup B_i=\dots$ и для пересечения.

Примеры: разные числа. Кортеж (упорядоченный), $A \times B$.

\section{} % 2
$R$ "--- отношение, $R \subset (A \times B)$, пишем $aRb$. Область определения ($\mathrm{dom}$) "---
откуда стрелочки, область значения ($\mathrm{ran}$) "--- куда, обратное "--- $R^{-1}$, композиция отношений
($a(f\circ g)b: \exists x: (a f x) \land (x g b)$). Функция "--- спец. отображение. Рефлексивность,
симметричность, транзитивность, иррефликсивность, антисимметричность (если два отношения, то равны).

\section{} % 3
$\R$ "--- поле (9 свойств), есть отношение $\le$ (рефл., антисимм., транз., все сравнимы,
$a \le b \Rightarrow a + c \le b + c$, $a \ge 0 \land b \ge 0 \Rightarrow ab \ge 0$), полнота
(можно найти число в щели между двумя множествами; это отличие от $\Q$). Принцип
Архимеда: $\forall \eps > 0 \exists n: n\eps > x$ ($x \in \R$); д-во: фиксируем $\eps$,
берём множество хороших $x$, оно лежит левее дополнения, взяли элемент из щели, он хз где. Следствия:
$\exists n: \frac 1 n < \eps$; есть (ир)рациональное между двумя любыми;

\section{} % 4
Верхняя/нижняя границы "--- подмножества $\R$, мн-во ограничено с какой-то стороны, если граница непуста
(еще бывает просто ограничено). $\sup A$ "--- наименьшая из верхних, $\inf A$ "--- симметрично.
Теорема, равносильная полноте: существуют $\sup$ и $\inf$, если есть границы. 

\section{} % 5
О вложенных отрезках: пересечение непусто, доказали из полноты. Существенна замкнутость (приблизили ноль), конечность
(не лучи, иначе приблизили $+\infty$), что в $\R$ (приблизили $\sqrt 2$). Кстати, полноту можно
вывести из вложенных отрезков и Архимеда (без док-ва, просто забавный факт).
