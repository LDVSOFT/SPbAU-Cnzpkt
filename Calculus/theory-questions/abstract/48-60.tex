\section{} % 48
Есть $f: E \to Y$, она непрер. в точке $a$, если: по Коши: $\eps-\delta$
с окрестностями; по Гейне: если $\{x_n\}\to a$, то $f(x_n) \to f(a)$
(то есть предел в каждом из смыслов совпал со значением). Непрерывна
слева (при $E \subset \R$), если сузили $E$ и непрерывно.

\pagebreak
\section{} % 49
Для $\R^d$: можно складывать, домножать на $\lambda$, брать модуль и произв.,
для $\R$: добавляем деление (если $g(a)\neq 0$).
Теорема о стаб. знака: если $f(a)\neq 0$, то в окрестности знак тот же.
Теорема о композиции: $f: E_1 \to Y$, $g: E_2 \to Z$, $f(E_1)\subset E_2$,
$f$ непр. в $a$, $g$ непр. в $f(a)$, тогда $g\circ f$ непрерывна в $a$.
Пример, непр. существенна: $f=x\sin \frac1x$, $g=(\mathtt{x!=0})$, тогда $\not\exists \lim g(f(x))$.

\section{} % 50
Пусть $f: X \to Y$, тогда непр. во всех точках $\iff$ прообраз любого открытого открыт
(д-во: ищем шарики вокруг точек).

\section{} % 51
Теорема: непрерывность в $\R^d$ равносильна покоорд. непрерывности в $a$.

\section{} % 52
Определение: отображение ограничено, если образ ограничен.
Теорема: непрерывный образ компакта "--- компакт.
Следствия: непрерывный образ компакта ограничен, т.Вейерштрасса
(любая $f$ на компакте ограничена), любая непрерывная $f: [a, b] \to \R$
ограничена, существуют и достигаются $\sup$ и $\inf$, еще т.Вейерштрасса
(если $f: [a,b]\to\R$ непрерывна, то достигает свой $\max$ и $\min$).

\section{} % 53
Если $f: K \to Y$ биективна, непрерывна ($K$ "--- компакт), то $f^{-1}$ непрерывно,
д-во: проверяем, что прообраз в $f^{-1}$ любого открытого открыт.
Следствие: если $f: [a,b]\to\R$ непрерывна и строго монотонна, то
$f^{-1}$ непрерывна на образе.

\section{} % 54
Равномерная непрерывность: по $\eps$ берём универсальное $\delta$ и для
всех близких точек получаем близкие значения.
Теорема Кантора(-Гейне): на компакте непрерывная функция равномерно непрерывна,
д-во: от противного, тогда нашли $\eps$ и последовательность пар точек,
которая портит для $\delta=\frac1n$, повыводили.
Следствие: для $[a,b]$.

\section{} % 55
Больцано-Коши: если $f:[a,b]\to\R$ непрерывна, то $\forall f(a) \le x \le f(b)$ существует
прообраз из $[a,b]$. Лемма о связности отрезка: если $[a,b]=U\cup V$, оба открыты и непересекаются,
то отрезок лежит целиком в одном. Д-во теоремы: от противного, рассмотрели
прообразы $(-\infty, x)$ и $(x, +\infty)$. Обратная теорема неверна: $f(x)=\sin1x$ (в нуле ноль).
Следствие: если $f$ непрерывна то образ отрезка "--- отрезок.
Следствие: если $f$ непрерывна на $<a,b>$, то принимает все значения в интервале от своего $\inf$ до $\sup$.

\section{} % 56
Лемма: $\sin x < x < \tg x$, д-во: картинка и площади треугольников и сектора. Следствие: $|\sin x| \le |x|$.
Следствие: $\sin$ и $\cos$ непрерывны, д-во: $\sin x - \sin y = 2\sin\frac{x-y}{2}\cos\frac{x+y}{2}$.
Следствие: $\tg$ и $\ctg$ непрерывны.
Теорема: $\lim \frac{\sin x}{x} = 1$, д-во: $\frac{\sin x}{x} \le 1 \le \frac{1}{\cos x} \cdot \frac{\sin x}{x}$,
два миллиционера.

\section{} % 57
Определили для $x^n, n\in\mathbb{N}$, больше нуля непрерывна по $x$, инфиум 0, супремум $+\infty$,
строго монотонна. $x^{\frac1n}$ "--- обратная, получили для рациональных степеней (надо проверить,
что не зависит от вида дроби $\frac ab=\frac{ka}{kb}$). Свойства ок: $x^ax^b$, $(x^a)^b$, $x^ay^a$,
$x^a<y^a$, $x^a<x^b$.
Лемма: при $a>0$ имеем $\lim a^{\frac 1 n}=1$, д-во: \TODO.
Теорема: если $x_n \to x$ и $a>0$, то $\exists \lim a^{x_n}$ и этот предел не зависит от последовательности, д-во: \TODO.
Определили степень на вещественных как предел, свойства ок и совпадает на $\Q$ со старым.

\section{} % 58
Теорема: при $a>0$ $a^x$ непрерывно. Следствие: есть обратная $\log_a x$.
Теорема: $\lim\limits_{x \to \infty} \left(1+\frac 1x\right)^x = e$ (не к $\pm\infty$, а как угодно),
д-во: взяли целую часть, ограничили миллиционерами.

\section{} % 59
В нуле: $\lim (1+x)^{\frac 1 x} = e$, $\lim \frac{\ln(1+x)}{x} = 1$, $\lim {a^x-1}{x} = \ln a$, $\lim \frac{(1+x)^p-1}{x} = p$.

\section{} % 60
Символы Ландау "--- ошки.
Есть $f, g : E \to \R$, $a$ "--- предельная точка. Если есть $\phi \mid f=\phi\cdot g$, то:
$f \sim g$ (если предел 1), $f=o(g)$ (если предел 0), $f=O(g)$ (если предел ограничен; можно перефразировать
через константу). Свойства: $\sim$ "--- эквивалентность; если $f_1\sim g_1$ и $f_2\sim g_2$, то $f_1f_2 \sim g_1g_2$,
$f\sim g \iff f=g+o(f) \iff g=f+o(g)$, $f\cdot o(g) = o(fg)$. 
