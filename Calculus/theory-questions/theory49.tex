\section{Арифметические действия с непрерывными функциями}

\begin{theorem}{Арифметические действия с непрерывными функциями}

$$f, g: E \subset X \to \R^d, a \in E f, g \text{непрерывны в точке a}$$

Тогда

\begin{enumerate}
\item f(x) + g(x) непрерывно в точке a
\item сf(x) непрерывно в точке a
\item f(x) - g(x) непрерывно в точке a
\item $\| f(x) \|$ непрерывно в точке a
\item < f(x), g(x) > непрерывно в точке a
\end{enumerate}

\end{theorem}

\begin{theorem}{}

$$f, g: E \subset X \to \R, a \in E f, g \text{непрерывны в точке a}$$

Тогда

\begin{enumerate}
\item f(x) + g(x) непрерывно в точке a
\item f(x)g(x) непрерывно в точке a
\item f(x) - g(x) непрерывно в точке a
\item |f(x)| непрерывно в точке a
\item Если g(a) $\ne 0$, то $\frac{f(x)}{g(x)}$ непрерывно в точке a
\end{enumerate}
\end{theorem}

\begin{theorem}{о стабильном знаке}

$$f: E \subset X \to \R a \in E, f \text{~--- непрерывно в точке a и} f(a) \ne 0$$

Тогда 

$$\exists B_{\delta}(a) \text{такое что } \forall x \in B_{\delta}(a) sign(f(x)) = sign(f(a))$$

\end{theorem}
 
\begin{proof}
$$\epsilon = \frac{|f(a)|}{2}$$
\end{proof}

\begin{theorem}{о непрерывности композиции}

$$f: E_1 \subset X \to Y$$

$$g:E_2 \subset Y \to Z$$

$$f(E_1) \subset E_2, a \in E_2$$

f ~--- непрерывна в точке a

g ~---  непрерывна в точке f(a)

тогда $g \circ f$ ~--- непрерывно в точке a.

\end{theorem}
\begin{proof}
Надо проверить, что 

$$\forall \epsilon > 0 \exists \delta > 0 \forall x \in B_{\delta}(a) \cap E_1: g(f(x)) \in B_{\epsilon}(g(f(a))) $$

Берем $\epsilon$

$$\exists \gamma > 0 \forall y \in B_{\gamma}(f(a)) \cap E_2: g(y) \in B_{\epsilon}(g(f(a))) \text{(по непрерывности g в точке f(a))}$$

$$\exists \delta > 0 \forall x \in B_{\delta}(a) \cap E_1: f(x) \in B_{\gamma}(f(a)) \text{(по непрерывности f в точке a)}$$

$$\Ra g(f(x)) \in B_{\epsilon}(g(f(a))$$

\end{proof}
