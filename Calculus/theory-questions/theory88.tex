\section{Неравенство о средних. Неравенства Гельдера и Минковского}

\begin{conseq}
Неравенство о средних. $x_1, x_2, \ldots, x_n \ge 0$.
$$\frac{x_1 + x_2 + \cdots + x_n}n \geqslant \sqrt[n]{x_1x_2 \cdots x_n}$$
\end{conseq}
\begin{proof}
$f(x) = -\ln x$~--- выпуклая убывающая функция. $\lambda_i = \frac1n$.
$$-\ln \frac{x_1 + x_2 + \cdots + x_n}n = f \left(\frac{x_1 + x_2 + \cdots + x_n}n\right) \leqslant \frac{f(x_1) + f(x_2) + \cdots + f(x_n)}n = -\ln \sqrt[n]{x_1x_2 \cdots x_n}$$
$$-\ln \frac{x_1 + x_2 + \cdots + x_n}n \leqslant -\ln \sqrt[n]{x_1x_2 \cdots x_n} \Ra \frac{x_1 + x_2 + \cdots + x_n}n \geqslant \sqrt[n]{x_1x_2 \cdots x_n}$$
\end{proof}
\begin{conseq}
$$x_1^{\lambda_1} + x_2^{\lambda_2} + \cdots + x_n^{\lambda_n} \leqslant \lambda_1x_1 + \lambda_2x_2 + \cdots + \lambda_n x_n$$
\end{conseq}

\begin{proof}
$f(x) = -ln(x)$ ~--- выпуклая функция.

$\lambda_1 = \ldots = \lambda_n = \frac1n$

$$f(\frac{x_1 + x_2 + \ldots + x_n}{n}) \ge \frac{f(x_1) + f(x_2) + \ldots + f(x_n)}{n}$$

$$-ln(\frac{x_1 + x_2 + \ldots + x_n}{n}) \ge \frac{-ln(x_1) + -ln(x_2) + \ldots + -ln(x_n)}{n} = \frac{-ln(x_1x_2\ldots x_n)}{n} = -ln(\sqrt[n]{x_1x_2\ldots x_n}$$


\end{proof}
\begin{conseq}
Неравенство Гельдера:
$$x_1, \ldots, x_n, y_1, \ldots, y_n \in \R \quad p,q > 1 \quad \frac1p + \frac1q = 1$$
$$\left|\sum_{i=1}^n x_iy_i\right| \leqslant \left(\sum_{i=1}^n |x_i|^p\right)^{\frac1p} \left(\sum_{i=1}^n |x_i|^q\right)^{\frac1q}$$
\end{conseq}
\begin{Rem}
Это неравенство БШ при $p=q=2$.
\end{Rem}
\begin{proof}
Если есть нули или отрицательные~--- перейдём к модулям.
$$f(x) = x^p$$
$$f\left( \lambda_1a_1 + \lambda_2a_2 + \cdots + \lambda_n a_n \right) \leqslant \lambda_1f(a_1) + \lambda_2f(a_2) + \cdots + \lambda_nf(a_n)$$
$$\left( \lambda_1a_1 + \lambda_2a_2 + \cdots + \lambda_n a_n \right)^p \leqslant \lambda_1 a_1^p + \lambda_2 a_2^p + \cdots + \lambda_n a_n^p$$
$$\lambda_1x_1 + \lambda_2x_2 + \cdots + \lambda_n x_n \leqslant \left(\lambda_1 x_1^p + \lambda_2 x_2^p + \cdots + \lambda_n x_n^p \right)^{\frac1p}$$

$$ \left\{\begin{aligned}
\lambda_i a_i^p &= x_i^p \\
\lambda_i a_i &= \frac{x_iy_i}{(\sum_{i=1}^n y_i^p) ^ {\frac1q}}
\end{aligned}\right.$$

$$\lambda_i  = (\frac{x_k}{a_k})^p$$ 
$$\frac{x_k^{p}}{a_k^{p - 1}} =\frac{x_k y_k}{(\sum y_k^q)^{\frac1q}} $$
$$a_k = (\sum y_k^q)^{\frac{1}{q(p - 1)}}\frac{1}{y_k^{\frac1{p - 1}}}x_k$$
$$\lambda = \frac{y_k^q}{\sum y_k^q}$$
$$\sum\frac{x_k y_k}{(\sum y_k ^q)^{\frac1q}} \le (\sum x_k^p)^{\frac1p} $$
\end{proof}

\begin{conseq}
Неравентсво Минковского, $p \ge 1$

$$(\sum_{k = 1}^{n}|x_k + y_k|^p)^{\frac1p} \le (\sum_{k = 1}^{n}|x_k|^p)^{\frac{1}{p}} + (\sum_{k = 1}^{n}|y_k|^p)^{\frac{1}{p}}$$
\end{conseq}

\begin{proof}
$$\frac1p + \frac1q = 1$$
$$\sum_{k = 1}^{n}|x_k + y_k| \le \sum_{k = 1}^{n}|x_k + y_k|^{p - 1}|x_k + y_k| \le $$
$$ \le \sum_{k - 1}^{n}|x_k + y_k|^{p - 1}|x_k| + \sum_{k = 1}^{n}|x_k + y_k|^{p - 1}|y_k| \le $$
$$ \le (\sum_{k - 1}^{n}|x_k + y_k|^{(p - 1)q})^{\frac1q}(\sum|x_k|^p)^{\frac{1}{p}} + (\sum_{k = 1}^{n}|x_k + y_k|^{(p - 1)q})^{\frac1q}(\sum|y_k|^p)^{\frac{1}{p}} \le $$
$$ (p - 1)q = (p - 1) \frac{p}{p - 1} = p$$
$$ = (\sum|x_k + y_k|p)^{\frac{1}{q}}((\sum|x_k|^{p})^{\frac1p} + (\sum|y_k|^{p})^{\frac{1}{p}})$$
\end{proof}
