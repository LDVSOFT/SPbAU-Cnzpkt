\section{Локальные максимумы и минимумы}

\section{Экстремумы функции}

\begin{Def}
$f\colon \left<a, b\right> \ra \R$, $x_0 \in (a, b)$. 
$x_0$~--- точка строгого локального минимума, если
$$\exists \delta>0\colon \forall x \in (x - \delta, x + \delta) \ \{x_0\} f(x) > f(x_0)$$
$x_0$~--- точка нестрогого локального минимума, если
$$\exists \delta>0\colon \forall x \in (x - \delta, x + \delta) f(x) \geqslant f(x_0)$$

$x_0$~--- точка строгого локального максимума, если
$$\exists \delta>0\colon \forall x \in (x - \delta, x + \delta) \ \{x_0\} f(x) < f(x_0)$$
$x_0$~--- точка нестрогого локального максимума, если
$$\exists \delta>0\colon \forall x \in (x - \delta, x + \delta) f(x) \leqslant f(x_0)$$

Точка локального максимума или минимума также называется точкой локального экстремума.
\end{Def}
\begin{theorem}{Необходимое условие экстремума}
$f\colon \left<a, b\right> \ra \R$, $x_0 \in (a, b)$, $f$ дифференцируема в $x_0$.
$$x_0\text{~--- экстремум} \Ra f'(x_0) = 0$$
\end{theorem}
\begin{proof}
Сузим до окрестности, там по теореме Ферма всё работает.
\end{proof}
\begin{Rem}
Обратное неверно, смотри $f(x) = x^3$.
\end{Rem}