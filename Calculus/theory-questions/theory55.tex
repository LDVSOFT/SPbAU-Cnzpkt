\section{Теорема Больцано-Коши}

\begin{lemma}{о связности отрезка}
 Пусть $[a, b] \subset U \cup V, U, V$ ~--- открытые и  $U \cap V = \emptyset$ тогда либо $[a, b] \subset U$, либо $[a, b] \subset V$ 
\end{lemma}

\begin{proof}
 Рассмотрим точку b. Пусть b $\in V$

 $S = [a, b] \cap U$, пусть $S \ne \emptyset$

 $$b_1 = sup S$$

 Поскольку b $\in V$ ~--- открытое $\Ra (b - \epsilon, b + \epsilon) \subset V$ для некоторого $\epsilon \Ra (b - \epsilon, b + \epsilon) \cap S = \emptyset$

 $\Ra b_1 \le b - \epsilon \Ra b_1 < b$

 Пусть $b_1 \in V \Ra (b_1 - \epsilon_1, b_1 + \epsilon) \subset V$

 $(b_1 -\epsilon_1, b_1 + \epsilon) \cap S = \emptyset \Ra sup S \le b_1 - \epsilon$. Противоречие.
 
 Тогда $b_1 \in U \Ra (b_1 - \epsilon_1, b_1 + \epsilon_1) \subset U$

 $\delta = min \{\epsilon_1, b - b_1\} > 0$

 $[b_1, b_1 + \epsilon_1) \subset S \Ra sup S \ge b_1 + \delta $ ~--- противоречие.
 
\end{proof}


\begin{theorem}{Больцано-Коши}
 $f:[a, b] \to \R$ f ~--- непрерывно на [a, b]

 $\forall C$ между f(a) и f(b) $\exists c \in (a, b) f(c) = C$

\end{theorem}

\begin{proof}
  От противного. Пусть $f(x) \ne C \forall x \in [a, b],$ тогда $[a, b] \subset f^{-1}((-\infty, c))\cup(f^{-1}(C, +\infty))$ ~--- открытые и не пересекаются, a и b принадлежат разным множествам. Противоречие. 
\end{proof}

\begin{conseq}{}
   \begin{enumerate}
       \item $f:[a,b] \to \R$ и непрерывно на [a,b], тогда $f([a, b])$ ~--- отрезок.
       \begin{proof}
          $\exists u, v \in [a, b], f(u) \le f(x) \le f(v) \forall x \in [a, b] \Ra f([a, b]) \subset [f(u), f(v)]$

          По теореме Б-К $\forall C \in (f(u), f(v)) \exists c \in (u, v)$, т.ч. $f(c) = C$ т.е. $f([a, b]) = [f(u), f(v)]$
       \end{proof}
       \item $f:<a, b> \to \R$  и непрерывно на $<a, b>$, тогда f принимает все значения из (inf f(x), sup f(x))
       \begin{proof}
          Пусть $C \in (inf, sup) \Ra \exists u:f(u) < C, \exists v:f(v) > C \Ra C$ лежит между $f(u)$ и $f(v)$, но f непрерывно на [u, v] $\Ra$ принимает все промежуточные значения.
       \end{proof}
   \end{enumerate}

\end{conseq}


                                        