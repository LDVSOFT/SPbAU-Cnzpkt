\section{Теорема о вложенных отрезках}

\begin{theorem}{Теорема о вложенных отрезках}
Вместе с теоремой Архимеда выводят полноту.
$\left\{\left[a_n, b_n\right]\right\}_{i=1}^n: \forall i \in \N\: \left(a_i \leqslant a_{i + 1} \land b_i \geqslant b_{i + 1}\right) \land \forall i, j \in \N\: a_i < b_j$. 
Тогда 
$$\bigcap_{i=1}^\infty [a_i, b_i] \neq \emptyset$$
\end{theorem}
\begin{proof}
$A = \{a_i\}, B = \{b_i\}$. 
Тогда по аксиоме полноты 
$$\exists c \in \R\colon \forall i \in \N\: c \in \left[a_i, b_i\right] \Ra c \in \bigcap_{i=1}^\infty [a_i, b_i] \neq \emptyset$$
\end{proof}

\begin{Rem} 
Существенна замкнутость отрезков.
$$\bigcap_{n=1}^\infty \left(0, \frac1n\right] = \emptyset$$
\end{Rem}
\begin{Rem} 
Не лучи.
$$\bigcap_{n=1}^\infty \left[n, +\infty\right) = \emptyset $$
\end{Rem}
\begin{Rem} 
$\R$. Рассмотрим приблежения $\sqrt{2}$.
\end{Rem}
