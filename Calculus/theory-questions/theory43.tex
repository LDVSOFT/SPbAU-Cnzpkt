\section{Свойства функций, имеющих предел}
\begin{Rem}
Если в определении по Гейне все пределы существуют, то они будут равны. То есть достаточно доказать,
что для любой сходящейся последовательности $\{x_n\}$ предел $f(x_n)$ существует, из этого
будет следовать по Гейне.
\end{Rem}
\begin{proof}
Возьмём две сходящиеся последовательности $x_n$ и $y_n$, после применения функций стремящиеся к каким-то разным значениям $b$ и $c$. Но тогда
у последовательности 
$$x_1, y_1, x_2, y_2, x_3, y_3$$
сходящейся к той же точке, будет предел. Но тогда у подпоследовательностей одинаковые пределы.
\end{proof}

\begin{assertion}{Единственность предела}
$f\colon E \subset X \ra Y$, $a$~--- предельная точка. Тогда предел $\lim\limits_{x\ra a} f(x)$ единственнен.
\end{assertion}
\begin{proof}
Пусть есть два различных предела. Тогда из определения по Коши с какого-то расстояния весь хвост должен быть ближе к одному пределу, чем к другому.
\end{proof}

\begin{theorem}{Ограниченность}
$f\colon E \subset X \ra Y$, $\lim\limits_{x\ra a} = b$. Тогда 
$$\exists r>0\colon f \mid_{E \cap B_r(x)}\text{ограничена}$$
\end{theorem}

\begin{theorem}{Уход от нуля}
$f\colon E \ra \R^d$, $\lim\limits_{x\ra a} = b \ne \vec 0$. Тогда
$$\exists r>0\colon \forall x \in \dot B_r(a) \cap E\; f(x) \ne 0$$
\end{theorem}
\begin{proof}
$\epsilon \lrh \rho(x, \vec 0)$
\end{proof}