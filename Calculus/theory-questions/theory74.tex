\section{Производные высших порядков}

\begin{Def}
Производной $n \geqslant 2$ порядка функции $f$ называется производная производной $n - 1$ порядка.
$$f^{(n)} = \left(f^{(n-1)}\right)'$$
\end{Def}

\begin{Def}
$C(E)$, $C[a, b]$, $C(a, b)$~--- множество непрерывных на $E$, $[a, b]$, $(a, b)$ функций. Соотвественно, $C^n(E)$~--- множество $n$ раз дифференцируемых функций, таких что n-ая производня непрерывна.
$$C^\infty(E) = \bigcap_{i=1}^\infty C^i(E)$$
\end{Def}

\begin{assertion}
$$C^n(E) \supset C^{n+1}(E)$$
\end{assertion}
\begin{Rem}
При том, что множества вложены друг в друга, они не равны.
$$f(x) = x^{n + \frac13}$$
Тогда 
$$f^{(n)} (x) = \prod_{i=1}^n \left(i+\frac13\right) x^{\frac13}$$
Т.о. $f \in C^n(\R)$, но $f^{(n)} = C \sqrt[3]{x}$ не дифференцируема в $0$, поэтому $f \notin C^{(n+1)}(\R)$
\end{Rem}