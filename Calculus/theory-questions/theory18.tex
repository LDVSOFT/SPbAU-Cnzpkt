\section{Конечное векторное пространство}

\begin{Def}
Вектор~--- кортеж $x = (x_1, x_2, \ldots, x_d) \in \R^d$. Операция сложения 
$$+\colon \R^d \times \R^d \ra \R^d;x+y = (x_1+y_1, x_2+y_2, \ldots, x_d + y_d)$$ 
и умножения 
$$\times\colon \R \times \R^d \ra \R^d; \lambda x = (\lambda x_1, \lambda x_2, \ldots, \lambda x_n)$$
\end{Def}
\begin{enumerate}
\item Сложение
\begin{enumerate}
\item Коммутативно
\item Ассоциативно
\item Существует ноль $\vec 0 = \underbrace{(0, 0, \ldots, 0)}_d$
\item Существует обратный элемент
\end{enumerate}
\item $\alpha (x + y) = \alpha x + \alpha y$
\item $(\alpha + \beta) x = \alpha x + \beta x$
\item $(\alpha\beta)x = \alpha(\beta x)$
\item $1x = x$
\end{enumerate}
\begin{Def}
Общее определение векторного пространства~--- 

$$"+": X + X \to X$$

$$"\times": \R \times X \to X$$

Обладает свойствами 1-4 и $1X = X$ 
\end{Def}

\begin{Def}
Скалярное произведение векторов (евклидово):
$$\langle x, y\rangle = \sum_{i=1}^d x_iy_i$$
\end{Def}
Свойства:~%
\begin{enumerate}
\item $\langle x, x\rangle \geqslant 0; \langle x, x\rangle = 0 \Lra x = \vec 0$
\item $\langle \lambda x, y\rangle = \lambda \langle x, y\rangle$
\item $\langle x, y\rangle = \langle y, x\rangle$
\item $\langle x + y, z\rangle = \langle x, z\rangle + \langle y, z\rangle$
\end{enumerate}

\begin{Def}
Общее определение скалярного произведения: $X$~--- веторное пространство. Задана операция $\langle x,y\rangle\colon X \times X \ra \R$ обладающая указынными свойствами.
\end{Def}
Например, если приписать в определение положительную константу~--- ничего не поменяется.

\begin{Def}
(Евклидова) норма:
$$\|x\| = \sqrt{\langle x, x\rangle}$$
\end{Def}
\begin{enumerate}
\item $\|x\| \geqslant 0; \|x\| = 0 \Lra x = \vec 0$
\item $\|\lambda x\| = |\lambda| \|x\|$
\item $|\langle x,y\rangle| \leqslant \|x\|\|y\|$ (нер-во Коши--Вуняковкского)
\item $\|x + y\| \leqslant \|x\| + \|y\|$ (нер-во треугольника)
\item $\|x - z\| \leqslant \|x - y\| + \|y - z\|$ (нер-во Минковского)
\item $\|x - y\| \geqslant \left|\|x\| - \|y\|\right|$
\begin{proof}
$\|x - y\| = \|y - x\|$. Таким образом достаточно показать, что 
$$\|x - y\| \geqslant \|x\| - \|y\| \La \|x - y\| + \|y\| \geqslant \|x\|$$
А это неравнство треугольника.
\end{proof}
\item $\rho(x, y) = \|x - y\|$~--- метрика. Это ровно евклидово пространтво на $\R^d$.
\end{enumerate}

\begin{Def}
Общее определение нормы: $\|x\|\colon X \Ra \R$, обладает свойствами 1, 2 и 4.
\end{Def}
Свойство 3 касается скаляроного произведения, которого может и не быть.

Примеры:~%
\begin{enumerate}
\item $\|x\|_1 = \sum\limits_{k=1}^d |x_k|$
\item $\|x\|_\infty = \max\limits_{k=1..d} |x_k|$
\begin{proof}
$$\|x + y\| = \max_{k=1..d} |x_k + y_k| \leqslant \max_{k=1..d} (|x_k| + |y_k|) = |x_{k_0}| + |y_{k_0}| \leqslant \|x\| + \|y\|$$
\end{proof}
\item $$\|x\|_d = \sqrt[p]{\sum\limits_{k=1}^d |x_k|^p}$$
\end{enumerate}