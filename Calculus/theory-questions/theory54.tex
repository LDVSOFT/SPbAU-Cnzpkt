\section{Равномерная непрерывность на функции. Теорема Кантора}

\begin{Def}
$f:X \to Y$ равномерно непрерывна, если

$\forall \epsilon > 0 \exists \delta > 0 \forall x, y \in X \rho(x, y) < \delta: \rho(f(x), f(y)) < \epsilon$ 

\end{Def}

\begin{theorem}{Кантора}

$f: K \to Y$ K ~-- компакт, f непрерывен на K $\Ra f$ равномерно непрерывно.

\end{theorem}

\begin{proof}

От противного.

Пусть для некоторого $\epsilon > 0$ нет $\delta > 0$, т.е не подходит $\delta = \frac1n$

$$\exists x_n, \tilde x_n \in K \text{т.ч} \rho(x_n, \tilde x_n) < \frac{1}{n} \rho(f(x_n), f(\tilde x_n)) \ge \epsilon$$

$x_n,\tilde x_n$ последовательность точек из K извлеем из $x_n$ сходящуюся подпоследовательность $x_{n_{k}}$, $x_{n_k} \to a \in K$

$\tilde x_{n_k} \to a$, т.к. $\rho(x_{n_k}, \tilde x_{n_k}) < \frac{1}{n_k} \to 0$

f непрерывно в точке a.

$\exists \delta > 0 \forall x \in B_{\delta}(a): f(x) \in B_{
frac{\epsilon}{2}}(f(a))$

Начиная с какого-то N $x_{n_k}, \tilde x_{n_k} \in B_{\delta}(a)$

$$\Ra f(x_{n_k}), f(\tilde x_{n_k}) \in B_{\epsilon}(f(a))$$

$$\Ra \rho(f(x_{n_k}), f(\tilde x_{n_k})) < \epsilon \text{противоречие}$$

\end{proof}

\begin{conseq}{}
 Непрерывное на [a, b] функция равномерно непрерывна.
\end{conseq}
