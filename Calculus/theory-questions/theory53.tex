\section{Теоремы о непрерывности обратного отображения и о непрерывности монотонной функции}

\begin{theorem}{}
$f: K \to Y$ непрерывно на K биекция между K и Y, тогда $f^{-1}: Y \to K$ непрерывно.
\end{theorem}

\begin{proof}

Надо проверить, что для $f^{-1}$ прообраз открытого множество ~--- открытое. Т.е. надо проверить для f, что образ открытого ~--- открыто.

Берем $G \subset K$ ~--- открытое.

$\Ra K \setminus G$ ~--- замкнутое подмножество  K.
 
$\Ra K\setminus G$ ~--- компакт.

$\Ra f(K \setminus G)$ ~--- компакт $\Ra$ замкнутое

$\Ra f(G)$ ~--- открыто. 
\end{proof}


\begin{conseq}{}
 \begin{enumerate}
 \item $f:[a, b] \to \R$ строго монотонно и f непреывна на [a, b] $\Ra f^{-1}$ непрерывно на множестве задания.
  \begin{proof}
   [a, b] = K ~--- компакт.

   строго монотонная $\Ra$ инъекция.

   f ~--- биекция между [a, b] и f([a,b])
  \end{proof}
 \item $f:([a,b], (a, b], [a, b), (a,b)) \to \R$ строго монотона и непрерывна на нем $\Ra f^{-1}$ непрерывна на множестве задания.
   \begin{proof}
   $$ y = f(<a,b>) $$

   $$f^{-1}:y \to \R$$

   Надо доказать непрерывность $\forall c \in y$

   Берем $c \in y \Ra c = f(x_0)$ для некоторого $x_0 \in <a, b>$

   Возьмем $x_0 \in [\alpha, \beta] \subset <a, b>$

   $g = f|_{[\alpha, \beta]}: [\alpha, \beta] \to \R $ применяем следствие 1.

   $\forall \epsilon > 0 \exists \delta > 0 \forall y \in B_{\delta}(c) \cap f([\alpha, \beta]): g^{-1}(y) \in B_{\epsilon}(g^{-1}(c))$

   $f: X \to Y$ непрерывно на X.
    
   $$\forall a \in X \forall \epsilon > 0 \exists \delta > 0 \forall x \in B_{\delta}(a): f(x) \in B_{\epsilon}(f(a)) $$

   \end{proof}
 \end{enumerate}
\end{conseq}

