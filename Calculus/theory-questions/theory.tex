%% texify: XeLaTeX + MakeIndex + BibTeX, modificated
%% texify
%% --pdf
%% --engine=xelatex
%% --tex-option=$synctexoption // You may delete this, affords to skip from preview to code in one click, that i seldom do
%% --tex-option=-8bit // Else minted fails on tabs
%% --tex-option=--shell-escape // For minted to live
%% $fullname
%% I prefer to build with TeXworks for better view of errors and warnings. It's hard to read all log file. 

\documentclass[12pt,a4paper]{article}
\usepackage{polyglossia} %% Better than babel on XeLaTeX
\usepackage{amsmath, amssymb} %% Cool math!
\usepackage{color} %% Coloring almost anything
\usepackage[russian]{hyperref} %% Clickable links is pdf
\usepackage{indentfirst}
\usepackage[left=1cm,right=1cm,top=2cm,bottom=2cm]{geometry}
%% WARNING: latest minted is used. Download from github!
%% Works fine, though
%\usepackage{minted} %% Highlighting code. Installation is hard: requires python2 and script Pygments. Look at documentation for help!
\usepackage[math-style=ISO,vargreek-shape=unicode]{unicode-math} %% MAGIC! INCLUDE AS LAST!

\setdefaultlanguage[spelling=modern]{russian} %% Languages for polyglossia
\setotherlanguage{english}

\defaultfontfeatures{Ligatures={TeX}} %% Fonts and ligatures.
\setmainfont{CMU Serif} %% There are original Knuth's fonts in Unicode, called Computer Modern Unicode. Download anywhere, just install them
\setsansfont{CMU Sans Serif}
\setmonofont{CMU Typewriter Text}  
\setmathfont{Latin Modern Math} %% Download too. You may change it :)
\AtBeginDocument{\def\setminus{\mathbin{\backslash}}}
%\setmathfont{XITS}

%% Magic as black as my working table
%\DeclareSymbolFont{cyrletters}{\encodingdefault}{\familydefault}{m}{it}
%\newcommand{\makecyrmathletter}[1]{%
%  \begingroup\lccode`a=#1\lowercase{\endgroup
%  \Umathcode`a}="0 \csname symcyrletters\endcsname\space #1
%}
%\count255="409
%\loop\ifnum\count255<"44F
%  \advance\count255 by 1
%  \makecyrmathletter{\count255}
%\repeat
%% Simpy adds cyrillic to maths!

\frenchspacing %% One space before sentence, not two!

%% Shortcuts:
\def\la{\leftarrow}
\def\ra{\rightarrow}
\def\lra{\leftrightarrow}
\def\La{\Leftarrow}
\def\Ra{\Rightarrow}
\def\Lra{\Leftrightarrow}
\def\lrh{\leftrightharpoons}
\def\xra{\xrightarrow}
\def\btu{\bigtriangleup}

\def\N{\mathbb{N}}
\def\Z{\mathbb{Z}}
\def\Q{\mathbb{Q}}
\def\R{\mathbb{R}}
\def\C{\mathbb{C}}

\def\LraDef{\stackrel{\mathrm{Def}}{\Lra}}
\def\eqDef{\stackrel{\mathrm{Def}}{=}}
\def\d{\mathup{d}}

% ======================================

%% Change Chapter and Section numeration style
%%\renewcommand{\thechapter}{\Roman{chapter}}
%%\renewcommand{\thesection}{\thechapter.\arabic{section}}

%% Indent for first par in chapter
\makeatletter
%%\renewcommand{\chapter}{\clearpage %% no double page, only
\thispagestyle{empty}%% not plain, empty. wanna number of page!
\global\@topnum=0
\@afterindenttrue %% Set to true!
%%\secdef\@chapter\@schapter}
\makeatother

%% Environment for theorem body
\newcounter{theorem}[section]
\renewcommand{\thetheorem}{\thesection.\arabic{theorem}}
\newcommand*{\theoremheader}[1]{\par\refstepcounter{theorem}%
\textbf{Теорема \thetheorem. #1.}}
\newenvironment*{theorem}[1]{
\theoremheader{#1}%
}{%
}

%% Environment for consequence body
\newcounter{conseq}[theorem]
\renewcommand{\theconseq}{\thetheorem.\arabic{conseq}}
\newcommand*{\conseqheader}{\par\refstepcounter{conseq}%
\textit{Следствие \theconseq.} }
\newenvironment*{conseq}{
\conseqheader%
}{%
}

\newcounter{lemma}[section]
\renewcommand{\thelemma}{\thesection.\arabic{lemma}}
\newcommand*{\lemmaheader}{\par\refstepcounter{lemma}%
\textit{Лемма \thelemma. }}
\newenvironment*{lemma}{
	\lemmaheader%
}{%
}

\newenvironment{assertion}{%
\par\textbf{Утверждение. }%
}{%
%
}

%% Environment for proof body. I like this style, but you are free to change it.
\newenvironment{proof}{%
\par$\blacktriangleright$%
}{%
\hfill$\blacktriangleleft$%
}

%% Environment for definitions. Pretty raw one.
\newenvironment{Def}{%
\par$\mathfrak{Def\colon}$%
}{%
}

%% Environment for remarks.
\newenvironment{Rem}{%
\par\textit{REM: }%
}{%
}

\setcounter{MaxMatrixCols}{40}

% ==================================

%% In-line code highlighting. Using: \py|a = input()|
%\newmintinline[cinl]{c}{} %\c is defined :(
%\newmintinline[cpp]{cpp}{}
%\newmintinline[python]{python}{}
%\newmintinline[bash]{bash}{}
%\newmintinline[make]{make}{}

%% Escaped code highlighting. Using: \begin{cppcode} ... \end{cppcode}
%\setminted{obeytabs,tabsize=4,linenos,texcomments}
%\newminted{c}{}
%\newminted{cpp}{}
%\newminted{python}{}
%\newminted{bash}{}
%\newminted{make}{}

% ==================================
\DeclareMathOperator{\Int}{int}
\DeclareMathOperator{\cl}{cl}
\DeclareMathOperator{\diam}{diam}
\newcommand{\emod}[1]{\mathop{\equiv}\limits_{#1}}


\begin{document}
\begin{center}
  {\Large \bf Лекции по математическому анализу} \\ 
  \vspace{0.5em}
  {\Large \bf Лектор: Храбров Александр Игоревич} \\
  \vspace{0.5em}
  {\Large Автор конспекта: Лапшин Дмитрий} \\
\end{center}

\vspace{-1em}
\noindent \underline{\hbox to 1\textwidth{{ } \hfil{ } \hfil{ } }}

\vspace{1em}
\tableofcontents
\pagebreak

\section{Множества}

Не любая совокупность элементов --- множество. Про каждый объект можно сказать, принадлежит ли он множеству ($x \in A$) или нет ($x \notin A$).

\begin{Def}
Множество $A$ - подмножество $B$, если все элементы $A$ содержатся и в $B$. 
$$ A \subset B \LraDef \forall x \in A\; x \in B $$
\end{Def}
\begin{Def}
Множества называются равными, если они содержатся друг в друге.
$$ A = B \LraDef A \subset B \land B \subset A $$
\end{Def}
\begin{Def}
Пустое множество --- это множество без элементов.
$$ \forall x\: x \notin \emptyset $$
\end{Def}
\begin{Def}
$2^A$ --- множество всех подмножеств $A$.
$$ 2^A \eqDef \left\{B \mid B \subset A \right\} $$
\end{Def}

\begin{itemize}
\item $\N$ --- множество натуральных чисел. 
\item $\Z$ --- множество целых чисел.
\item $\Q$ --- множество рациональных чисел.
\item $\R$ --- множества вещественных чисел.
\item $\C$ --- множества комплексных чисел.
\end{itemize}

Задание множеств:
\begin{itemize} 
\item $\left\{a,b,c\right\}$
\item $\left\{a_1, a_2, \ldots, a_n\right\}$
\item $\left\{a_1, a_2, \ldots\right\}$
\item $\left\{x \in A \mid \Phi(x)\right\}, \Phi(x) - \text{условие}$.
\end{itemize} 
Например, $\left\{p \in \N \mid p \text{ имеет ровно 2 натуральных делителя}\right\}$.

Бывают некорректно заданные <<множества>>. Например, множество художественных произведений на русском языке --- плохо заданное множество. Рассмотрим 
$\Phi(n)$ --- истина, если n нельзя записать в не более чем тридцать слов русского языка. Тогда
$\left\{n \in \N \mid \Phi(n)\right\}$~--- не множество. Если бы это было множеством, то в нём есть наименьший элемент, 
который описывается как <<Наименьший элемент множества...>>

\begin{Def}
Пересечение двух множеств~--- множество, состоящие из всех элементов, находящихся одновременно в обоих множествах.
$$ A \cap B \eqDef \left\{x \in A \mid x \in B \right\} $$
\end{Def}
\begin{Def}
Объединение двух множеств~--- множество, состоящее из элементов обоих множеств.
$$ A \cup B \eqDef \left\{x \mid x \in A \lor x \in B \right\} $$
\end{Def}
\begin{Def}
Разность множеств~--- это множество тех элементов, которые лежат в первом, но не во втором.
$$ A \setminus B \eqDef \left\{ x \in A \mid x \notin B \right\}$$
\end{Def}
\begin{Def}
Симметрическя разность~--- объединение разностей.
$$ A \btu B \eqDef \left(A \setminus B\right) \cup \left(B \setminus A\right) $$
\end{Def}

Объединение и пересечение множно записать для многих множеств.
$$ \bigcup_{i \in I} A_i = \left\{x \mid \exists i \in I\colon x \in A_i\right\}; 
\bigcap_{i \in I} A_i = \left\{x \mid \forall i \in I\: x \in A_i \right\} $$

Свойства операций со множествами:
\begin{enumerate}
\item Ассоциативность
$$ A \cap B = B \cap A; A \cup B = B \cup A $$
\item Коммутативность
$$ \left(A \cap B \right) \cap C = A \cap \left(B \cap C \right); \left(A \cup B \right) \cup C = A \cup \left(B \cup C \right) $$
\item Рефлексивность
$$ A \cap A = A; A \cup A = A $$
\item Дистрибутивность
$$ A \cap \left(B \cup C \right) = \left(A \cap B\right) \cup \left(A \cap C \right) $$
$$ A \cup \left(B \cap C \right) = \left(A \cup B\right) \cap \left(A \cup C \right) $$
\item Нейтральный элемент
$$ A \cap \emptyset = \emptyset$$
$$ A \cup \emptyset = A$$
\end{enumerate}

\begin{theorem}{Правила де Моргана}
$ A, B_\alpha, \alpha \in I $.
Тогда 
$$ A \setminus \bigcup_{\alpha \in I} B_\alpha = \bigcap_{\alpha \in I} \left(A \setminus B_\alpha\right) ; 
A \setminus \bigcap_{\alpha \in I} B_\alpha = \bigcup_{\alpha \in I} \left(A \setminus B_\alpha\right) $$
\end{theorem} 
\begin{proof}
$$
x \in A \setminus \bigcup_{\alpha \in I} B_{\alpha} \Lra \left\{\begin{aligned}x &\in A \\ x &\notin \bigcup_{\alpha \in I} B_{\alpha}\end{aligned}\right. \Lra 
\left\{\begin{aligned} x &\in A \\ \forall \alpha \in I\: x &\notin B_\alpha \end{aligned}\right. \Lra
\forall \alpha \in I\: \left\{\begin{aligned} x &\in A \\ x &\notin B_\alpha \end{aligned}\right.  
\Lra x \in \bigcap_{\alpha \in I} \left(A \setminus B_\alpha\right) 
$$
$$
x \in A \setminus \bigcap_{\alpha \in I} B_{\alpha} \Lra \left\{\begin{aligned}x &\in A \\ x &\notin \bigcap_{\alpha \in I} B_{\alpha}\end{aligned}\right. \Lra 
\left\{\begin{aligned} x &\in A \\ \lnot \forall \alpha \in I\: x &\in B_\alpha \end{aligned}\right. \Lra
\exists \alpha \in I\colon \left\{\begin{aligned} x &\in A \\ x &\notin B_\alpha \end{aligned}\right.  
\Lra x \in \bigcup_{\alpha \in I} \left(A \setminus B_\alpha\right) 
$$
\end{proof}

\begin{theorem}{Обобщение дистрибутивности}
$ A, B_\alpha, \alpha \in I $.
Тогда 
$$ A \cap \bigcup_{\alpha \in I} B_\alpha = \bigcup_{\alpha \in I} (A \cap B_\alpha) $$
$$ A \cup \bigcap_{\alpha \in I} B_\alpha = \bigcap_{\alpha \in I} (A \cup B_\alpha) $$
\end{theorem}
\begin{proof}
$$
x \in A \cap \bigcup_{\alpha \in I} B_{\alpha} \Lra \left\{\begin{aligned}x &\in A \\ x &\in \bigcup_{\alpha \in I} B_{\alpha}\end{aligned}\right. \Lra 
\left\{\begin{aligned} x &\in A \\ \exists \alpha \in I\colon x &\in B_\alpha \end{aligned}\right. \Lra
\exists \alpha \in I\colon \left\{\begin{aligned} x &\in A \\ x &\in B_\alpha \end{aligned}\right.  
\Lra x \in \bigcup_{\alpha \in I} \left(A \cap B_\alpha\right) 
$$
$$
x \in A \cup \bigcap_{\alpha \in I} B_{\alpha} \Lra \left[\begin{aligned}x &\in A \\ x &\in \bigcap_{\alpha \in I} B_{\alpha}\end{aligned}\right. \Lra 
\left[\begin{aligned} x &\in A \\ \forall \alpha \in I\: x &\in B_\alpha \end{aligned}\right. \Lra
\forall \alpha \in I\: \left[\begin{aligned} x &\in A \\ x &\in B_\alpha \end{aligned}\right.  
\Lra x \in \bigcap_{\alpha \in I} \left(A \cup B_\alpha\right) 
$$
\end{proof}

\begin{Def}
Упорядоченная пара $\langle a, b \rangle$ или $(a, b)$ --- объект
$$ (a_1; b_1) = (a_2; b_2) \LraDef a_1 = a_2 \land b_1 = b_2 $$
\end{Def}
\begin{Def}
Упорядоченная $n$-ка, или кортеж --- объект
$$ (a_1, a_2, \ldots, a_n) = (b_1, b_2, \ldots, b_n) \LraDef \forall i=1..n\: a_i = b_i $$
\end{Def}

\begin{Def}
Декартого произведение множеств --- множество кортежей, состоящих из элементов соответствующих множеств.
$$ \left(a_1, a_2, \ldots, a_n\right) \in A_1 \times A_2 \times \ldots \times A_n \LraDef \forall i=1..n\: a_i \in A_i $$
\end{Def}
\section{Обратимые отображения и их свойства}

$f: A \to B$

\begin{Def}
f "--- обратное справа, если $\exists g: B \to A$

$f \circ g = id_B$

f "--- обратим слева, если $\exists g: B \to A$

$g \circ f = id_A$

f обратимо, если $\exists g: B \to A$

$$g \circ f = id_A, f \circ g = id_B$$

g "--- отображение, обратное к f.(обозначение $f^{-1}$)
\end{Def}

\begin{theorem}{}

\begin{enumerate}
\item f обратимо $\Lra$ f обратимо слава и справа.
\item f обратимо, то обратное отображение единственно.
\end{enumerate}

\end{theorem}

\begin{proof}
\begin{enumerate}
\item f обратимо $\Ra$ f обратимо слева и справа.

Если у f есть и левый и правый обратный, то они совпадают. 

g "--- правый обратный к f, h "--- левый.

$(h \circ f) \circ g = id_A \circ g = g$

$h \circ (f \circ g) = h \circ id_B = h$

$\Ra g = h$

\item Пусть f обратимое и g и h "--- два обратных. В частности g "--- обратное справа, h "--- обратное слева.
\end{enumerate}
\end{proof}

\begin{theorem}{}
$f:A \to B, g:B \to C$

$g \circ f: A \to C$

\begin{enumerate}
\item Если f, g обратимы справа, то и $g \circ f$ обратима справа.
\item Если f, g обратимы слева, то и $g \circ f$ обратима слева.
\item Если f, g обратимы, то $g \circ f$ обратима $(g \circ f)^{-1} = f^{-1} \circ g^{-1}$
\end{enumerate}
\end{theorem}

\begin{proof}
\begin{enumerate}
\item
$$u: B \to A, f \circ u = id_B$$
$$v: C \to B g \circ v = id_C$$
$$(g \circ f) \circ (u \circ v) = g \circ (f \circ (u \circ v)) = $$
$$= g \circ ((f \circ u) \circ v) = g \circ (id_B \circ v) = g \circ v = id_C$$

$u \circ v$ "--- правый обратный к $g \circ f$

\item аналогично

\item 
$$(g \circ f)(f^{-1} \circ g^{-1}) = g \circ ((f \circ f^{-1}) \circ g^{-1}) = g \circ (id_B \circ g^{-1}) = g \circ g^{-1} = id_C$$

$$(f^{-1} \circ g^{-1})\circ(g \circ f) = f^{-1}(g^{-1} \circ g) \circ f = f^{-1} \circ id_B \circ f = f^{-1} \circ f = id_A$$

\end{enumerate}
\end{proof}

\begin{conseq}{}
Композиция сюръективных "--- сюръективна.

Композиция инъективных "--- инъективна.

Композиция биективных "--- биекция.

\end{conseq}

\begin{theorem}{}
$f: A \to B$ f "--- обратима, тогда $f^{-1}$ обратима и $(f^{-1})^{-1} = f$
\end{theorem}

\begin{proof}
$f \circ f^{-1} = id_{B}$

$f^{-1} \circ f = id_A \Ra f$ "--- обратное к $f^{-1}$

В силу единственности обратного $(f^{-1})^{-1} = f$
\end{proof}
\section{Тождественное отображение}

\begin{Def}
$A, id_{A}: A \to A$

$\forall a \in A id_A(a) = a$

$id_A$ ~--- тождественное  отображение множетсва A.

$\Gamma_{id_A}$ = диагональ $A \times A \{(a, a)| a \in A\}$ 
\end{Def}

\begin{theorem}{}
$f: A \to B$

$f \circ id_A = f =  id_B \circ f$
\end{theorem}

\begin{proof}

Области определения и назначения совпадают. 

$\forall y \in B, id_{B}(y) = y$

$a \in A$

$(f \circ id_A)(a) = f(id_A(a)) = f(a)$

$a \in A$

$(id_B \circ f)(a) = id_B(f(a)) = f(a)$

\end{proof}
\section{Равносильность инъективности и обратимости слева}

\begin{theorem}{}
Пусть $f:A \to B$ и $A \ne \varnothing$. Тогда $f$ обратима слева $\iff$ $f$ инъективна.
\end{theorem}

\begin{proof}
\begin{enumerate}
\item $\Ra$

$\exists g \colon g \circ f = id_A \Ra f$ инъективно.

\item $\La$

Пусть $C = f(A)$. Построим $h_1: C \to A$ такое, что
\[(c, a) \in \Gamma_{h_1} \Lra (a, c) \in \Gamma_{f}\]. Проверим, что это график:

\begin{enumerate}
\item Определённость для $c \in C$:
\begin{gather*}
\forall c \in C, \exists a \in A \colon (a, c) \in \Gamma_{f}; \\
\forall c \in C, \exists a \in A \colon (c, a) \in \Gamma_{h_1}; \\
\end{gather*}
\item Однозначность. Знаем, что $f$ инъективно. 
\begin{gather*}
\forall a_1, a_2 \in A, \exists b \in B \colon (a_1, b) \in \Gamma_{f} \wedge (a_2, b)\in \Gamma_{f} \Ra a_1 = a_2; \\
\forall a_1, a_2 \in A, \exists b \in C \colon (a_1, b) \in \Gamma_{f} \wedge (a_2, b)\in \Gamma_{f} \Ra a_1 = a_2; \\
\forall a_1, a_2 \in A, \exists b \in C \colon (b, a_1) \in \Gamma_{h_1} \wedge (b, a_2)\in \Gamma_{h_1} \Ra a_1 = a_2;
\end{gather*}
\end{enumerate}

$\Ra \Gamma_{h_1}$ "--- график.

Теперь построим $h: B \to A$. Для этого выберем произвольный $a \in A$ и положим:

$h(b) = \begin{cases} h_1(b), & \text{если~} b \in C\\ a, &\text{если~} b \notin C\end{cases}$

Проверим, что $h \circ f = id_A$. Рассмотрим $x \in A$:
\[(h \circ f)(x) = h(f(x)) = h_1(f(x)) = x\]
\end{enumerate}
\end{proof}

\section{Теорема о вложенных отрезках}

\begin{theorem}{Теорема о вложенных отрезках}
Вместе с теоремой Архимеда выводят полноту.
$\left\{\left[a_n, b_n\right]\right\}_{i=1}^n: \forall i \in \N \left(a_i <= a_{i + 1} \land b_i >= b_{i + 1}\right) \land \forall i, j \in \N a_i < b_j$. 
Тогда 
$$\bigcap_{i=1}^\infty [a_i, b_i] \neq \emptyset$$
\end{theorem}
\begin{proof}
$A = \{a_i\}, B = \{b_i\}$. 
Тогда по аксиоме полноты 
$$\exists c \in \R\colon \forall i \in \N\: c \in \left[a_i, b_i\right] \Ra c \in \bigcap_{i=1}^\infty [a_i, b_i] \neq \emptyset$$
\end{proof}

\begin{Rem} 
Существенна замкнутость отрезков.
$$\bigcap_{n=1}^\infty \left(0, \frac1n\right] = \emptyset$$
\end{Rem}
\begin{Rem} 
Не лучи.
$$\bigcap_{n=1}^\infty \left[n, +\infty\right) = \emptyset $$
\end{Rem}
\begin{Rem} 
$\R$. Рассмотрим приблежения $\sqrt{2}$.
\end{Rem}

\section{Инъективное отображение конечного множества на себя является биективным}

\begin{theorem}{}
A "--- конечное множество. 

$f: A \rat A $, тогда f "--- биекция.
\end{theorem}

\begin{proof}
f "--- сюръекция? 

$a_0 = a$

$a_{i + 1} = f(a_i)$

$\exists m \ne n a_m = a_n m > n$

\begin{lemma}{}
    $a_{m - n} = a$    
\end{lemma}
    \begin{proof}
        Индукция по n.
        {\bf База:} n = 0, $a_m = a_0 = a$
        {\bf Переход} $n \ge 1$

        $$f(a_{m - 1}) = a_m = a_n = f(a_{n - 1})$$
        Так как инъекция $a_{m - 1} \le a_{n - 1}$
        $$a_{m - n} = a_{(m - 1) - (n - 1)} = a$$
        $$a_{m - n} = a$$
        $$m - n \ge 1$$
        $$a = a_{m - n} = f(a_{m - n - 1})$$

        a есть образ $a_{m - n - 1} \Ra f$ "--- сюръекция.
    \end{proof}
\end{proof}
\section{Сюръективное отображение конечного множества на себя является биективным}

\begin{theorem}{}
    A "--- конечное множество. 
    $f: A \thra A$, тогда f "--- биекция.
\end{theorem}

\begin{proof}
    \begin{enumerate}
        \item $\forall a \exists n_a \{f \circ f \circ \ldots \circ f\}(a) = a$
        \item $\exists n \forall a (f \circ \ldots \circ f)(a)  = a$
        \item f "--- инъекция.
    \end{enumerate}
    \begin{enumerate}
    \item
    $$a_0 = a$$
    $$a_i f^{-1}(\{a_i\}) \ne 0$$
    $$\exists a_{i + 1} \in f^{-1}(\{a_i\})$$
    $$\exists m > n a_m = a_n$$
    \begin{lemma}{}
        $a_{m - n} = a$   
    \end{lemma}
    \begin{proof}
        Индукция по n.
        {\bf База:} $n = 0,  a_m = a_0 = a$
        {\bf Переход:} $$a_m = a_n$$
        $$f(a_m) = f(a_n)$$
        $$a_{m - 1} = f(a_m) = f(a_n) = a_{n - 1}$$
        По индукционному предположению 
        $$a_{m - n} = a_{(m - 1) - (n - 1)} = a$$
    \end{proof}
    $$a_{m - n} \in  f^{-1}(f^{-1}\ldots(\{a\}))$$
    $$f(f(\ldots f(a_{m - n}))) = a$$
    $$f(f(\ldots f(a))) = a$$
    $$(f \circ f \circ \ldots)(a) = a$$
    $$\forall a \in A \exists n_a \ge 1 \ub{(f \circ \ldots \circ f)}_{n_a}(a) = a$$
    \item 
    $$k \in N \ub{(f \circ \ldots \circ f)}_{n_a k}(a) = a$$
    (индукция по k)
    
    $$N = \prod_{a \in A} n_a  \ub{(f \circ \ldots \circ f)}_{N}(a) = a$$
    \item
    $$a, b \in A$$
    $$f(a) = f(b)$$
    $$a = (\ub{f \circ \ldots \circ f}_{N - 1} \circ f)(a) = (\ub{f \circ \ldots \circ f}_{N - 1} \circ f)(b) = b$$
    \end{enumerate}
\end{proof}
\section{Бинарные отношения}
\begin{Def}
На А задано бинарное отношение R, если задано $R \subset A$

$(a, b) \in R$

a и b находятся в отношение с R

$a R b$

R = 0 пустое

$R = A^2$ полное. 
\end{Def}

\begin{Def}
$A, R \subset A^2$
\begin{enumerate}
\item R рефлексивно,если $\forall a \in A, aRa (a, a)\in R$
\item R антирефлексивно, если $\forall a \in A \neg (aRa)$
\item R симметрично, если $\forall a, b \in A aRb \Ra bRa$
\item R асимметрично, если $\forall a, b \in A aRb \Ra \neg(bRa)$
\item R антисимметрично, если $\forall a, b \in A (aRb \wedge bRa) \Ra a = b$
\item R транзитивно, если  $\forall a, b, c \in A (aRb \wedge bRc) \Ra aRc$
\end{enumerate}
\end{Def}

\begin{Def}
R называется отношением несторого частичного порядка, если оно рефлексивно, транзетивно и антисимметрино. 
\end{Def}
\begin{Def}
R называется отношением сторого частичного порядка, если оно антирефлексивно, транзетивно и асимметрино. 
\end{Def}

Если на А задано отношение частичного порядко, то А ~--- частично упорядоченное множество.
\section{Отношение эквивалентности}

\begin{Def}
    $R$ "--- отношение эквивалентности, если оно рефлексивное, симметрично и транзитивно. Стандартное
    обозначение: $a \sim b$.
\end{Def}
\begin{Def}
    Для $a \in A$ определён
    класс эквивалентности: $[a] = \{b \in A \mid a \sim b\}$
\end{Def}

\begin{theorem}{}
Пусть $a, b \in A$. Тогда либо $[a]\cap [b] = \varnothing$, либо $[a] = [b]$ 
\end{theorem}

\begin{proof}
\begin{enumerate}
\item $[a] \cap [b] = \varnothing$ "--- всё доказано. Заметим, что тогда $a \nsim b$, в противном случае $a, b \in [a], [b]$.
\item $\exists c \in [a] \cap [b]$. Тогда $c \sim a$ и $c \sim b$ $\Ra a \sim b$.

Покажем, что $[a] \subset [b]$ и $[b] \subset [a]$:

\begin{enumerate}
\item $x \in [a] \Ra x \sim a \Ra x \sim b \Ra x \in [b]$
\item Аналогично
\end{enumerate}
\end{enumerate}

\end{proof}

Множество классов эквивалентности называется фактормножеством по отношению $R$.

\section{Замкнутые множества}

\begin{Def}
Замкнутые множество --- множество, дополнение которого открыто.
\end{Def}

\begin{theorem}{О свойствах закмнутых множеств}
Пусть $(X, \rho)$ --- метрическое пространство.
\begin{enumerate}
\item $\varnothing$ и $X$ --- закмнуты.
\item Перечечение замкнутых --- замкнуто.
\item Объеднинение конечного числа замкнутых замкнуто.
\item Замкнутый шар замкнут.
\end{enumerate}
\end{theorem}
\begin{proof}
\begin{enumerate}
\item Очевидно
\item По формулам де Моргана
$$X \setminus \bigcap_{\alpha \in I} F_\alpha = \bigcup_{\alpha \in I} \left(X \setminus F_\alpha \right)$$
\item По формуле де Моргана
$$$$
\item Докажем, что $X \setminus \bar B_r(a)$ открыт. Рассмотрим $x \in X \setminus \bar B_r(a)$. Тогда по определению $$\rho(a, x) > r$$
Покажем, что $$B_{\rho(a, x) - r}(x) \cap \bar B_r(a) = \varnothing$$
Пусть $\exists y \in B_{\rho(a, x) - r}(x) \cap \bar B_r(a)$. Тогда
$$y \in \bar B_r(a) \Ra \rho(a, y) \leqslant r$$
$$y \in B_{\rho(a, x) - r}(x) \Ra \rho(x, y) < \rho(a, x) - r$$
$$\rho(a, x) \leqslant \rho(a, y) + \rho (x, y) < r + (\rho(a, x) - r) = \rho(a, x) \text{ --- противоречие}$$
\end{enumerate}
\end{proof}
\begin{Rem}
$$\bigcup_{n=1}^\infty \left[\frac1n;1\right] = \left(0; 1\right]$$
\end{Rem}

\begin{Def}
$A \subset X$, $(X, \rho)$. Тогда замыкание множества $A$ --- перечесение всех замкнутых множеств, содержащих A.
$$\cl A = \bigcap_{\substack{F \text{ замкнуто}\\F \supset A}}F$$
\end{Def}

\begin{theorem}{О связи замыкания и внутренности}
$$X \setminus \cl A = \Int (X \setminus A)$$
$$X \setminus \Int A = \cl (X \setminus A)$$
\end{theorem}
\begin{proof}
$$X \setminus \cl A = X \setminus \bigcap_{\substack{F \text{ замкнуто}\\F \supset A}} F = \bigcup_{\substack{F \text{ замкнуто}\\F \supset A}} (X \setminus F)$$
$$X \setminus F \text{ открыто}$$
$$X \setminus F \subset X \setminus A$$
То
$$\bigcup_{\substack{F \text{ замкнуто}\\F \supset A}} (X \setminus F) = \bigcup_{\substack{G \text{ открыто}\\G \subset X \setminus A}} G = \Int (X \setminus A)$$
Аналогично
\end{proof}
\begin{conseq}
$$ \Int A = \cl (X \setminus A)$$
$$ \cl A = \Int (X \setminus A)$$
\end{conseq}

Свойства замыкания:
\begin{enumerate}
\item $A \subset \cl A$
\item $\cl A$ замкнуто.
\item $A \text{ замкнуто} \Lra A = \cl A$
\item $A \subset B \Ra \cl A \subset \cl B$
\item $\cl (A \cup B) = \cl A \cup \cl B$
\item $\cl \cl A = \cl A$
\end{enumerate}
\section{Подгруппы. Критерий того, что непустое подмножество группы является подгруппой. Пересечение подгрупп}
\begin{Def}
	Пусть $f: A \to B$ и $C \subset A$. Введём $g$ "--- сужение $f$ на $C$:
	\begin{gather*}
	g: C \ra B \\
	\forall c \in C\colon g(c) = f(c)
	\end{gather*}
\end{Def}

\begin{Def}
	$H \subset G$ "--- подгруппа в $G$, если она является группой относительно сужения операции в $G$ на $H$.
\end{Def}


\begin{Def}
	Множество $A$ замкнуто относительно операции $\cdot$, если $\forall a, b \in A \colon a \cdot b \in A$

	Множество $A$ замкнуто относительно операции взятия обратного, если $\forall a \in A \colon a^{-1} \in A$ \\
\end{Def}

\begin{theorem}{Достаточные условия для подгруппы}
Для того, чтобы доказать, что $H$ "--- подгруппа $G$ ($H \neq \varnothing$, $H \subset G$), достаточно проверить только замкнутость относительно операциий $\cdot$ и взятия обратного элемента. \\
\end{theorem}
\begin{proof}
Ассоциативность к нам переходит из исходной группы $G$.

Если существует обратный элемент, то $a a^{-1} = e \in H$

Так как есть замкнутость, то операция $\cdot$ действует из $H \times H$ в $H$.
\end{proof}

\begin{conseq}
Пусть $\varnothing \ne H \subset G$ и $\forall a, b \in H: ab^{-1} \in H \Ra H$ "--- подгруппа.  
\end{conseq}
\begin{proof}
Нейтральный элемент есть: $a \in H \Ra aa^{-1} \in H \Ra e \in H$.

Замкнутость относительно взятия обратного: $\forall a \in H \colon ea^{-1} \in H \Ra a^{-1} \in H$.

Замкнутость относительно операции $\cdot$: $\forall a, b \in H \Ra a, b^{-1} \in H \Ra a\left(b^{-1}\right)^{-1} \in H \Ra ab \in H$.
\end{proof}

\begin{theorem}{}
	$H_\alpha$ "--- подгруппы в $G$. Тогда $\bigcap H_\alpha$ "--- подгруппа в $G$.
\end{theorem}
\begin{proof}
    \begin{gather*}
	H = \bigcap H_\alpha; \\
	e \in H_{\alpha} \Ra e \in H \Ra H \ne \varnothing; \\
	a, b \in H \Ra \forall \alpha\colon a, b \in H_\alpha \Ra ab \in H_\alpha \Ra ab \in H; \\
	a \in H \Ra \forall \alpha \colon a \in H_\alpha \Ra a^{-1} \in H_\alpha \Ra a^{-1} \in H;
	\end{gather*}
\end{proof}

\section{Предельные точки}

\begin{Def}
Проколотая окрестность точки:
$$\dot B_r(x) = B_r(x) \setminus \{x\}$$
\end{Def}
\begin{Def}
Точка $x \in X$ предельная у множества $A$, если
$$\forall r > 0\: \dot B_r(x) \cap A \ne \varnothing$$
\end{Def}
\begin{Def}
$A'$~--- множество предельных точек.
\end{Def}

Свойства:
\begin{enumerate}
\item $\cl A = A \cup A'$
\item $A \subset B \Ra A' \subset B'$
\item $(A \cup B)' = A' \cup B'$
\begin{proof}
$\supset$:
$$A \cup B \supset A \Ra (A \cup B)' \supset A'$$
$$A \cup B \supset B \Ra (A \cup B)' \supset B'$$
Тогда $$(A \cup B)' \supset A' \cup B'$$
$\subset$: Пусть $x \in (A \cup B)' \land x \notin B'$.
$$x \in (A \cup B)' \Ra \forall r > 0\: \dot B_r(x) \cap (A \cup B) \ne \varnothing$$
$$x \notin B' \Ra \exists r_0 > 0\colon \dot B_{r_0}(x) \cap B = \varnothing \Ra \forall r \leqslant r_0\: \dot B_r(x) = \varnothing$$
Тогда $$\forall r > 0\: \dot B_r(x) \cap A \ne \varnothing \Ra x \in A'$$
\end{proof} 
\end{enumerate}

\begin{theorem}{Об окрестности предельной точки}
$$x \in A' \Lra \forall r > 0 \left|B_r(x) \cap A\right| = \infty$$
\end{theorem}
\begin{proof}
$$x \in A' \Ra \dot B_r(x) \cap A \ne \varnothing \Ra \exists y_1 \in A\colon y_1 \ne x \land y \in B_r(x)$$
Тогда
$$\dot B_{\rho(x,y_1)} \cap A \ne \varnothing \Ra \exists y_2 \in A\colon y_2 \ne x \land y_2 \ne y_1 \land y \in B_{\rho(x,y_1)}$$
Тогда рассмотрим
$$\{y_i\}_{i=1}^\infty\colon y_i \ne y_j \land y_i \ne x \land y_i \in A$$
\end{proof}
\begin{conseq}
$|A| < \infty \Ra A' = \varnothing$
\end{conseq}
\section{Гомоморфизмы групп. Свойства гомоморфизмов}
\begin{Def}
	$H$, $G$ "--- группы, $f: G \to H$
	\begin{enumerate}
		\item $f$ "--- гомоморфизм, если $\forall a, b \in G \colon f(ab) = f(a)f(b)$
		\item $f$ "--- изоморфизм, если $f$ "--- и гомоморфизм, и биекция
	\end{enumerate}
\end{Def}

\begin{Def}
	$H$, $G$ "--- группы. Если между $H$, $G$ есть изоморфизм, то группы называются изоморфными: $H \cong G$
\end{Def}						

\begin{theorem}{Свойства гомоморфизма}
	\begin{enumerate}
		\item $f(e_G) = e_H$
		\item $f(x^{-1}) = (f(x))^{-1}$
		\item $f(x) \subset H$
		\item $G \xrightarrow{f} H \xrightarrow{g} K$, $f$, $g$ "--- гомоморфизмы, тогда $g \circ f: G \to K$ "--- тоже гомоморфизм
		\item $f: G \to H$ "--- изоморфизм, тогда $f^{-1}: H \to G$ "--- изоморфизм
	\end{enumerate}
\end{theorem}
\begin{proof}
	\begin{enumerate}
		\item $f(e_G) = f(e_G e_G) = f(e_G)f(e_G) = e_H e_H = e_H$
		\item $f(e_G) = f(x x^{-1}) = f(x)f(x^{-1}) = e_H \Ra f(x^{-1}) = (f(x))^{-1}$
		\item
		    \begin{gather*}
			e_H \in f(G); \\
			c, d \in f(G), \exists a, b \in G\colon c = f(a), d = f(b); \\
			cd = f(a)f(b) = f(ab), cd \in f(G); \\
			\end{gather*}
			По п. 2 $f(a)^{-1} = f(a^{-1}) \in G \Ra G$ "--- подгруппа(замкнутость относительно операции и взятия обратного).
		\item Пусть $a, b \in G$, тогда:
			\[(g \circ f)(ab) = g(f(ab)) = g(f(a)f(b)) = g(f(a))g(f(b)) = (g \circ f)(a)(g \circ f)(b)\]
		\item
			\begin{gather*}
			c, d \in H; \\
			\exists a, b \in G\colon c = f(a); d = f(d); \\
			a = f^{-1}(c); b = f^{-1}(d); \\
			f^{-1}(cd) = ab = f^{-1}(c)f^{-1}(d);
			\end{gather*}
			Тогда $f^{-1}$ "--- изоморфизм
	\end{enumerate}   
\end{proof}		

\begin{conseq}
    \begin{enumerate}
    \item  $G \cong G$ (взяли $id_G$).
    \item  $G \cong H \Ra H \cong G$
    \item  $G \cong H$, $H \cong K$ $\Ra$ $G \cong K$
    \end{enumerate}

\end{conseq}

\begin{exmp}
пусть $G=H=\left< \mathbb{R}, +\right>$, тогда $x \ra e^x$ "--- изоморфизм.
\end{exmp}

\section{Предел последовательности}

\begin{Def}
Пусть есть пространство $(X, \rho)$ и последовательность $(x_i)$. Тогда
$$x^* = \lim_{n\ra\infty} x_n \LraDef x^* \in X \land \forall \epsilon > 0\: \exists N\colon \forall n \geqslant N\: \rho(x^*;x_i) < \epsilon$$
\end{Def}
Примеры:
\begin{itemize}
\item $\lim_{n\ra\infty} x = x$
\item $\R\colon \lim_{n\ra\infty} \frac1n = 0$
\end{itemize}
\begin{Rem}
Определение зависит от метрического пространства, в котором мы находимся. Последнего предела на $(0; +\infty)$ нет. А на метрике
$$\rho(x; y) = \begin{cases}0 & x = y \\ 1 & x \ne y \end{cases}$$ предел есть только у стационарных последовательностей.
\end{Rem}

\begin{theorem}{Свойства предела}
\begin{enumerate}
\item $x^* = \lim_{n\ra\infty} x_n \Lra$ каждая окрестность $x^*$ содержит всю последовательность с некотрого элемента
\item $x^* = \lim_{n\ra\infty} x_n \land x^{**} = \lim_{n\ra\infty} x_n \Ra x^* = x^{**}$
\item $\exists x^* = \lim_{n\ra\infty} x_n \Ra (x_n) \text{ ограниченна}$
\item $x \in A' \Ra \exists (x_n) \subset A\colon \lim_{n\ra\infty} x_n = x$
\end{enumerate}
\end{theorem}
\begin{proof}
\begin{enumerate}
\item $\Ra$: Пусть $x^* \in U$ --- открытое множество. Тогда
$$\exists r > 0\colon B_r(x^*) \subset U$$
$$\forall \epsilon > 0\: \exists N\colon \forall n \geqslant N\: \rho(x^*;x_n) < \epsilon \Ra \exists N\colon \forall n \geqslant N\: x_n \in U$$
$\La$: $U \lrh B_\epsilon(x^*)$.
$$\forall \epsilon > 0\: \exists N\colon \forall n \geqslant N\: x_n \in U \Ra x_* = \lim_{n\ra\infty} x_n$$
\item Пусть $\epsilon \lrh \frac{\rho(x^*;x^{**})}2 > 0$
$$x^* = \lim_{n\ra\infty} x_n \Ra \exists N_1\colon \forall n \geqslant N_1\: \rho(x^*;x_n) < \epsilon$$
$$x^{**} = \lim_{n\ra\infty} x_n \Ra \exists N_2\colon \forall n \geqslant N_2\: \rho(x^{**};x_n) < \epsilon$$
Тогда
$$\forall n \geqslant \max\{N_1; N_2\} \left\{\begin{aligned}\rho(x^*;x_n) < \epsilon \\ \rho(x^{**};x_n) < \epsilon\end{aligned}\right. \Ra$$
$$\Ra 2\epsilon = \rho(x^*;x^{**}) \leqslant \rho(x^*;x_n) + \rho(x^{**}; x_n) < 2\epsilon$$
\item $x^* = \lim_{n\ra\infty} x_n \Ra \exists N\colon \forall n \geqslant N\: \rho(x^{*}; x_n) < 1$. Рассмотрим 
$$R = 1 + \max_{n < N}\{\rho(x^*;x_n)\}$$
Тогда $$\forall n\: x_n \in B_R(x^*)$$
\item $x \in A'$. Рассмотрим 
$$x_1 \in \dot B_1(x) \cap A \ne \varnothing$$
$$x_2 \in \dot B_{\min\left\{\frac12;\rho(x;x_1)\right\}}(x) \cap A \ne \varnothing$$
$$x_3 \in \dot B_{\min\left\{\frac13;\rho(x;x_2)\right\}}(x) \cap A \ne \varnothing$$
$$\vdots$$
$$x_n \in \dot B_{\min\left\{\frac1n;\rho(x;x_n)\right\}}(x) \cap A \ne \varnothing$$
Тогда $$\forall n \geqslant N\: \rho(x; x_n) < \frac1N \Ra x = \lim_{n\ra\infty} x_n$$
\end{enumerate}
\end{proof}
\begin{Rem}
В пункте 4 можно выбрать различные $x_n$.
\end{Rem}
\begin{Rem}
Если $x_n$ --- различные и $x^*$ --- их предел, то $x^* \in \{x_n\}'$
\end{Rem}
\begin{Rem}
$$x = \lim_{n\ra\infty} x_n \land x_n \in A \Ra x \in \cl A$$
\end{Rem}

Далее будем работать с $(\R; |x - y|)$.
\section{Предельный переход в неравенстве}
\begin{theorem}{Предельный переход в неравенстве}
Пусть $x_n, y_n \in \R; x = \lim x_n; y = \lim y_n; x_n \leqslant y_n$ (или $x_n < y_n$). Тогда $x \leqslant y$.
\end{theorem}
\begin{proof}
Пусть $y < x$; $\epsilon \lrh \frac{x - y}2$. Тогда 
$$\exists N_1: \forall n \geqslant N_1\: |x - x_n| < \epsilon$$
$$\exists N_2: \forall n \geqslant N_2\: |y - y_n| < \epsilon$$
Тогда
$$\forall n \geqslant \max\{N_1, N_2\}\: x_n > x - \epsilon = y + \epsilon > y_n$$
\end{proof}
\begin{Rem}
Понятно, что можно потребовать отношение между последовательностями только с некоторого номера.
\end{Rem}                                                 
\begin{Rem}
Строгие неравенства не сохраняются.
\end{Rem}
\begin{conseq}
$x_n \leqslant b \Ra x \leqslant b$
\end{conseq}
\begin{conseq}
$x_n \geqslant a \Ra x \geqslant a$
\end{conseq}
\begin{conseq}
$x_n \in [a;b] \Ra x \in [a; b]$
\end{conseq}
\section{Теорема Лагранжа и следствия из нее}
\begin{lemma}
	$H \subset G, f: H \ra aH, h \ra ah$. Тогда $f$ - биекция. В частности, из этого будет следовать, что $|H| = |aH|$, то есть мощности всех левых классов смежности равны друг другу \\	
\end{lemma} 
\begin{proof}
	Заметим, что это отображение - сюрьекция по определению $aH$ (есть прообраз). Докажем, что это инъекция.  \\
	Пусть $ah_1 = ah_2$, домножим слева на $a^{-1}$, получим $h_1 = h_2$, значит инъекция и сюрьекция, значит биекция. \\
\end{proof}


\begin{Def}
	Число левых классов смежности по $H$ называется индексом $H$ в $G$. Обозначение: $[G:H]$
\end{Def}

\begin{theorem}{Теорема Лагранжа}
	Пусть $G$ - конечная группа, $G = \cup aH_{\alpha}, H_{\alpha_1} \cap H_{\alpha_2} = \emptyset$ \\
	Тогда $|G| = [G : H] * |H|$                                                                                                    \\
\end{theorem}
\begin{proof}
	Все эти классы имеют одинаковую мощность, равную $|H|$ (лемма). Тогда $|G| = [G : H] * |H|$, так как эти классы не пересекаются
\end{proof}
\begin{conseq}
	Количество правых и левых классов смежности одинаково(достаточно провести аналогичные действия для правых классов смежности)
\end{conseq}
\begin{conseq}
	Порядок любой подгруппы делит порядок конечной группы.
\end{conseq}
\begin{conseq}
	Порядок любого элемента делит порядок конечной группы (рассмотрим циклическую подгруппу, порожденную этим элементом)
\end{conseq}
\begin{conseq}
	Группа порядка $p$, $p$ - простое число, циклична(порядок любого элемента равен либо 1($e$), либо $p$(все остальные)), тогда все элементы кроме $e$ порождают группу порядка $p$ 
\end{conseq}
\section{Теорема о разложении перестановки в произведение непересекающихся циклов}

\begin{theorem}{Всяка перестановка может быть представлена в виде произведения непересекающихся циклов}
\begin{Def}
	$\sigma \in S_n, j \in \{1, \dotsc, n\}. j$ -- неподвижная точка относительно $\sigma$, есди $\sigma(j) = j$
\end{Def}
\end{theorem}
\begin{proof}
	Индукция по $m$ -- числу подвижных точек $\sigma$ (число подвижных точек = $n$ - число неподвижных точек).\\
База: $m = 0 \Leftrightarrow n $ -- число неподвижных точек, то есть $\forall j \in \{1, \dotsc, n\} \sigma(j) = j, $ то есть $\sigma = id$\\
Переход: $m > 0 \Rightarrow \exists i: \sigma(i) \ne i$ -- с него и начнём.\\
$i_1 = i$\\
$i_2 = \sigma(i_1)$\\
$i_3 = \sigma(i_2) = \sigma^2(i_1)$\\
$\dotsb$\\
И так до тех пор, пока не встретим повторение(а его мы обязательно встретим, потому что чисел у нас всего $n$ -- конечное число)

$i_1 \mapsto i_2 \mapsto \dotsc \mapsto i_k \mapsto i_{k+1}$ -- уже встречался. 

Заметим, что $i_{k+1} \ne i_j \forall j \in \{2, \dotsc, n\}$ в силу инъективности \sigma (иначе у какого-то элемента было бы два различных прообраза -- $i_{j-1}$ и $i_{k+1}$) $\Rightarrow i_{k+1} = i_1$.  

Итак, мы получили цикл $\tau = (i_1, i_2, \dotsc, i_k)$. 

Рассмотрим $\sigma\tau^{-1}$. 

Неподвижные точки $\sigma\tau^{-1}$ - это неподвижные точки $\sigma $ плюс $i_1, i_2, \dotsc, i_k \Rightarrow \sigma\tau^{-1}$ и $\tau$ - незацепляющиеся $\Rightarrow($по индукционному предположению$) \sigma\tau^{-1} = \sqcap_{j = 1}^r \tau_j$. 

Домножим обе части равества на $\tau$ и получим, что $\sigma = (\sqcap_{j = 1}^r \tau_j)\tau$ -- произведение незацепляющихся циклов.\\
\end{proof}
\begin{theorem}{Следствие}
$S_n$ порождается всеми циклами (так как для каждой $\sigma$ можно выбрать свой набор).
\end{theorem}

Небольшой забавный бонус:\\
$\sigma = \sqcap_{j = 1}^r \tau_j$, где $\tau_j$ -- попарно непересекающиеся циклы. Тогда $\sigma^m = (\sqcap_{j = 1}^r \tau_j)^m$, так как непересекающиеся циклы коммутируют.\\
\begin{Def}
Прядок цикла -- его длина.
\end{Def}
\begin{Def}
Прядок $\sigma$ -- наименьшее общее кратноедлин циклов при разложении $\sigma$ на непересекающиеся циклы.
\end{Def}
\begin{Def}
Цикловым типом $\sigma \in S_n$ называется набор длин её непересекающихся циклов, упорядоченных по неубыванию, + набор единиц для неподвижных элементов.
\end{Def}
\section{Симетрическая группа. Порождение симметрической группы транспозициями}

\begin{Def}
$S_n$ "--- биекции на $\{1, \dotsc, n\}$.
$S_n$ "--- симметрическая группа (группа перестановок) степени $n$.
Произведение двух перестановок "--- это их композиция: $\sigma \cdot \tau = \sigma \circ \tau$,
то есть первой выполняется биекция $\tau$.
\end{Def}

\begin{exmp}

\begin{tabular}{ c c }
  \begin{tabular}{|c|c|c|c|c|c|c|}
  \hline
  $x$       & 1 & 2 & 3 & 4 & 5 & 6 \\ \hline
  $\tau(x)$ & 5 & 1 & 6 & 4 & 2 & 3 \\ \hline
  \end{tabular}
  &
  \begin{tabular}{|c|c|c|c|c|c|c|}
  \hline
  $x$       & 1 & 2 & 3 & 4 & 5 & 6 \\ \hline
  $\sigma(x)$ & 2 & 4 & 5 & 1 & 3 & 6 \\ \hline
  \end{tabular}
  \\
  \rule{0pt}{4ex}
  \begin{tabular}{|c|c|c|c|c|c|c|}
  \hline
  $x$                    & 1 & 2 & 3 & 4 & 5 & 6 \\ \hline
  $(\tau\cdot\sigma)(x)$ & 1 & 4 & 2 & 5 & 6 & 3 \\ \hline
  \end{tabular}
  &
  \begin{tabular}{|c|c|c|c|c|c|c|}
  \hline
  $x$                    & 1 & 2 & 3 & 4 & 5 & 6 \\ \hline
  $(\sigma\cdot\tau)(x)$ & 3 & 2 & 6 & 1 & 4 & 5 \\ \hline
  \end{tabular}
\end{tabular}
\end{exmp}

\begin{Def}
	Цикл $(i_1, i_2, \dotsc, i_k)$ "--- это перестановка $\sigma$ такая, что:
	\begin{gather*}
	\sigma(i_1) = i_2; \\ 
	\sigma(i_2) = i_3; \\
	\dotsc \\
	\sigma(i_k) = i_1; \\
	\sigma(j) = j, j \notin \{i_1, \dotsc, i_k\}
	\end{gather*}
	То есть цикл переводит $i_1\to i_2 \to \dotsc \to i_k \to i_1$ и оставляет остальные элементы на месте.
	$k$ называется длиной цикла.
\end{Def}

\begin{Rem}
    \[(i_1, i_2, \dotsc, i_k) = (i_2, i_3, \dotsc, i_k) = \dotsc = (i_s, i_{s+1}, \dotsc, i_k, i_1, \dotsc, i_{s - 1})\]
\end{Rem}
\begin{Rem}
    Порядок цикла длины $k$ равен $k$. В частности, $(i_1, i_2, \dotsc, i_k)^k = id$.
\end{Rem}

\begin{Def}
	Транспозиция "--- цикл длины 2.
	$(i, j)$ "--- $i$ и $j$ меняются местами, остальные остаются на месте.
\end{Def}

\begin{Def}
	Пусть $(i_1, i_2, \dotsc, i_k)$, $(j_1, j_2, \dotsc, j_s)$ "--- циклы.
	Эти циклы называются незацепляющимися (непересекающимися), если $\{i_1, i_2, \dotsc, i_k\} \bigcap \{j_1, j_2, \dotsc, j_s\} = \varnothing$.
\end{Def}

\begin{Rem}
    Если $\sigma, \tau$ "--- непересекающиеся циклы, то $\sigma\tau = \tau\sigma$, легко перемножать.
\end{Rem}

В билете 17 показывается, что любая перестановка представляется как произведение циклов. Да, тут
как бы идёт 17 билет.

\begin{theorem}{$S_n$ порождается транспозициями}
\end{theorem}
\begin{proof}
Покажем, что любой цикл $(i_1, i_2, \dotsc, i_k)$ есть произведение транспозиций $(i_1 i_2)(i_2 i_3)\dotsc(i_{k - 1}i_k)$.
Заметим, что если применить это произведение к перестановке (вот просто взять и последовательно применить справа налео все транспозиции аккуратно), то мы и получим тот же самый результат, как и от применения исходного цикла.

Осталось лишь заметить, что каждая из остальных $n - k$ точек любо неподвижная, либо также лежит на каком-то цикле "--- а бить циклы на произведение транспозиций мы только что научились.
\end{proof}

\section{Четность перестановки. Теорема об изменении четности перестановки при умножении на транспозицию. Следствия из неё}

\begin{Def}
Пусть $\sigma \in S_n$ и $i, j = 1..n$. Тогда $i$ и $j$ образуют инверсию относительно $\sigma$, если $i < j$, а $\sigma(i) > \sigma(j)$.
\end{Def}
\begin{Def}
$Inv(\sigma)$ "--- множество всех инверсий относительно $\sigma$.
\end{Def}
\begin{Def}
$\sigma$ "--- четная, если $|Inv(\sigma)|$ четно.
\end{Def}
\begin{Def}
$\sigma$ "--- нечетная, если $|Inv(\sigma)|$ нечетно.
\end{Def}

\begin{theorem}{}
$\sigma \in S_n, \tau = (i~j)$ "--- транспозиция. Тогда $\sigma$ и $\sigma\tau$ имеют различную четность.
\end{theorem}
\begin{proof}
Не умаляя общности, будем считать, что $i < j$. Тогда
\begin{gather*}
\sigma=\left(
\begin{matrix}
1 & \cdots & i & \cdots & j & \cdots & n \\
\sigma_1 & \cdots & \sigma_i & \cdots & \sigma_j & \cdots & \sigma_n \\
\end{matrix}
\right) \\
\sigma\tau=\left(
\begin{matrix}
1 & \cdots & i & \cdots & j & \cdots & n \\
\sigma_1 & \cdots & \sigma_j & \cdots & \sigma_i & \cdots & \sigma_n \\
\end{matrix}
\right)
\end{gather*}

$$
	\begin{tabular}{|c|c|c|c|}
	\hline
	\multicolumn{2}{|c|}{$\sigma$} & \multicolumn{2}{c|}{$\sigma\tau$}\\
	\hline
	$(k~l), \{k, l\} \bigcap \{i, j\} = \varnothing$ & есть инверсия & $(k~l), \{k, l\} \bigcap \{i, j\} = \varnothing$ & есть инверсия\\
	\hline
	$(k~l), \{k, l\} \bigcap \{i, j\} = \varnothing$ & есть инверсия & $(k~l), \{k, l\} \bigcap \{i, j\} = \varnothing$ & есть инверсия\\
	\hline
	$(k~i), k < i$ & есть & $(k~j), k < i$ & есть\\
	\hline
	$(k~i), k < i$ & нет & $(k~j), k < i$ & нет\\
	\hline
	$(k~i), k > j$ & есть & $(k~j), k > j$ & есть\\

	\hline
	$(k~j), k < i$ & есть & $(k~i), k < i$ & есть\\
	\hline
	$(k~j), k < i$ & нет & $(k~i), k < i$ & нет\\
	\hline
	$(k~j), k > j$ & есть & $(k~i), k > j$ & есть\\
	\hline
	$(k~j), k > j$ & нет & $(k~i), k > j$ & нет\\
	
	\hline
	\multicolumn{4}{|c|}{$\sigma(\sigma(i)\sigma(k)\sigma(j))$}\\
	\hline
	\multicolumn{4}{|c|}{$\sigma\tau(\sigma(j)\sigma(k)\sigma(i))$}\\
	\hline
	$(k~i), i < k < j$ & есть & $(k~j), i < k < j$ & нет\\
	\hline
	$(k~i), i < k < j$ & нет & $(k~j), i < k < j$ & есть\\
	\hline
	$(k~j), i < k < j$ & есть & $(k~i), i < k < j$ & нет\\
	\hline
	$(k~j), i < k < j$ & нет & $(k~i), i < k < j$ & есть\\

	\hline
	(i~j) & есть & (i~j) & нет\\
	\hline
	(i~j) & нет & (i~j) & есть\\
	\hline

	\end{tabular}
$$	
Как глубокоуважаемый читатель уже догадался, самое интересное здесь "--- это последние 6 строк.
\begin{gather*}
r = |\{k \mid i < k < j \land (i~k)\text{ образует инверсию}\}| \\
s = |\{k \mid i < k < j \land (k~j)\text{ образует инверсию}\}| \\
|Inv(\sigma\tau)| = |Inv(\sigma)| - r + (j - i - 1 - r) - s + (j - i - 1 - s) \pm 1
\end{gather*}
Таким обазом, чётность изменилась.
\end{proof}

Следствия:
\begin{enumerate}
\item Пусть $\sigma = \prod_{j = 1}^r \tau_j$, где $\tau_j$ "--- транспозиция.
Тогда $\sigma$ четна (нечетна) $\iff$ число сомножителей четно (нечетно).
  \begin{proof}
  \begin{itemize}
  \item $\Leftarrow$: $id$ четна. При каждом добавлении $\tau_j$ четность меняется.
  \item $\Rightarrow$: $\sigma$ четна, $r$ "--- число транспозиций, и если у него другая четность, то приходим к противоречию.
  \end{itemize}
  \end{proof}
\item $\sigma \in S_n$, $\tau$ "--- транспозиция. $\sigma$ и $\tau\sigma$ имеют различную четность. (из первого следствия)
\item При перемножении двух перестановок их четность меняется так же, как при суммировании их четностей как чисел.
\item Множество четных перестановок "--- подгруппа $S_n$. Называется <<знакопеременная группа>>, обозначается $A_n$.
  \begin{proof}
  \begin{enumerate}
  \item Непусто ($id$ четна)
  \item Замкнуто (по третьему следствию)
  \item Обратный элемент к $\sigma = \prod_{j = 1}^{2r} \tau_j$ "--- это $\sigma^{-1} = \prod_{j = 1}^{2r} \tau_{2r + 1 - j}$ 
  \end{enumerate}
  \end{proof}
\end{enumerate}

\section{Кольца, тела, поля}
$A \neq \oslash$\\
$+: A \times A \ra A$\\
$\cdot: A \times A \ra A$\\

\begin{enumerate}
\item ассоциативнсть сложения:
	 $$ \forall a, b, c \in A (a + b) + c = a + (b + c) $$
\item существование нейтрального элемента по сложению:
	$$ \exists 0 \in A \forall a \in A a + 0 = 0 + a = a $$
\item существование обратного элемента по сложению:
	$$ \forall a \in A \exists -a \in A a + (-a) = (-a) + a = 0 $$
\item коммутативность сложения:
	$$ \forall a, b \in A a \cdot b = b \cdot a $$
\item ассоциативность умножения:
	$$ \forall a, b, c \in A a \cdot b = b \cdot a $$
\item коммутативность умножения:
	$$ \forall a, b \in A a \cdot b = b \cdot a $$
\item существование нейтрального элемента по умножению:
	$$ \exists 1 \in A \forall a \in A a \cdot 1 = 1 \cdot a = a $$
\item существование обратного элеменрта по умножению:
 	$$ \forall a \in A \setminus \lbrace 0 \rbrace \exists a^{-1} \in A a \cdot a^{-1} = a^{-1} \cdot a = 1 $$
\item дистрибутивность: \\
	a) $ \forall a, b, c \in A (a + b) \cdot c = a \cdot c + b \cdot c $\\
	b) $ \forall a, b, c \in A c \cdot (a + b) = c \cdot a + c \cdot b $\\
\end{enumerate}

\begin{Def}
	Кольцо - непустое множество $A$ с операциями $"+"$, $"\cdot"$, удовлетворяющее свойствам 1 - 5, 9 (a, b)
\end{Def}
\begin{Def}
	Кольцо, в котором выполнена аксиома 6 - коммутативное кольцо
\end{Def}
\begin{Def}
	Кольцо, в котором выполнена аксиома 7 - кольцо с единицей
\end{Def}
\begin{Def}
	Тело - кольцо с $1$, в котором $1 \neq 0$ и выполнена аксиома 8
\end{Def}
\begin{Def}
	Поле - коммутативное кольцо с $1$, в котором $1 \neq 0$ и выполнена аксиома 8 (т.е. все 9 аксиом)
\end{Def}

\begin{Rem}
	иногда кольца, для которых выполнены аксиомы 1-4, 9 называют ассоциативными кольцами
\end{Rem}

\begin{Rem}
$(A, +, \cdot)$ - кольцо, $(A, +)$ - абелева группа\\
\end{Rem}

\textbf{ Примеры: }
\begin{itemize}
\item $\Z$ - коммутативное кольцо с $1$, но $2$ не имеет обратного в $\Z$ $\Ra$ не поле
\item $N$ - не кольцо
\item $2\Z$ - кольцо без $1$
\item $\Q, \R$ - поля
\end{itemize}

\textbf{Простейшие свойства колец}

\begin{enumerate}
\item $0$ - единственный
\item $-a$ - единственный
\item $1$ - единственная (если есть) 
	\begin{proof}
		$$ 1 = 1 \cdot 1' = 1 $$
	\end{proof}
\item если у $a$ есть обратный по умножению, то он единственен 
	\begin{proof}
		$$ a' = a'aa''=a'' $$
	\end{proof}
\item если в кольце с 1 у элемента $a$ есть 2 левых обратных, то левых обратных к $a$ бесконечно много (упражнение)
	
\item $0 \cdot a = a \cdot 0 = 0 $
	\begin{proof}
		$$ a \cdot 0 + a \cdot 0 = a(0 + 0) = a\cdot0 |  + (a\cdot0)' $$
		$$ a \cdot 0 + a \cdot 0 + (a \cdot 0)' = a \cdot 0 + (a \cdot 0)' = a \cdot 0 = 0 $$
	\textit{ второе равенство аналогично }
	\end{proof}
\item $ a(-b) = (-a)b = -(ab)$
	\begin{proof}
		$$ a + (-a) = 0 $$
		$$ ab + (-a)b = (a + (-a))b = ab = 0 $$
		$$ (-a)b = -ab $$
	\textit { второе равенство аналогично }
	\end{proof}
\item $ 0 = 1, возможно лишь если |A| = 1, A = \lbrace 0 \rbrace $
	\begin{proof}
		$$ a \in A a = 1 \cdot a = 0 \cdot a = 0 $$
	\end{proof}
\end{enumerate}

\begin{Def}
$A$ - кольцо (тело, поле)\\
	$A \supseteq B \neq \oslash$ - подкольцо (подтело, подполе), если являктся кольцом (телом, полем), относительно сужения операций на $B$\\
\end{Def}


\begin{Rem}
\begin{itemize}
	\item $ B \neq \oslash ,	B \supset A$ - подкольцо в $A$ если оно замкнуто относительно умножения, сложения, взятия обратного по сложению
	\item $B$ - подтело, если подкольцо и замкнуто по взятию обратного ненулевого элемента по умножению и содержит элементы отличные от нуля:\\
	$ \forall a, b \in B a + b \in B$\\
	$ \forall a \in B a -a \in B$\\
	$ \forall a,b \in B ab \in B$\\
	$ \forall a \in B \setminus a^{-1} \in B$\\
\end{itemize}
\end{Rem}

\begin{Def}
 $A, B$ - кольца\\
	$f: A \ra B$\\
	$f$ - гомоморфизм, если:\\
		$$ \forall a_1, a_2 \in A f(a_1 + a_2) = f(a_1) + f(a_2) $$
		$$ \forall a_1, a_2 \in A f(a_1a_2) = f(a_1)f(a_2) $$
\end{Def}

\begin{Def}		
	$f$ - изоморфизм, если $f$ - гомоморфизм и биекция\\
\end{Def}

$A, B$ изоморфны, если существует изоморфизм между $A$ и $B$
	 $$ A \cong B $$
\begin{Rem}
	$f$ - гомоморфизм и $f(0_A)$обратим по сложению, тогда $f(0_A) = 0_B$
\end{Rem}
\begin{proof}
	$f(0_A) = f(0_A + 0_A) = f(0_A) + f(0_A)$, говорим, что у $f(1_A)$ есть обратный по сложению,
	прибавляем его и получаем: $f(0_A) = 0_B$
\end{proof}

\begin{Rem}
	Если $f$ - гомоморфизм и $f(1_A)$ обратим в B то $f(1_A)=1_B$
\end{Rem}
\begin{proof}
	$f(1_A) = f(1_A \cdot 1_A) = f(1_A)f(1_A)$, говорим, что у $f(1_A)$ есть обратный по умножению,
	умножаем на него и получаем: $f(1_A) = 1_B$
\end{proof}
	
\textbf{Делимость в кольцах}

$A$ - кольцо, $a, b, c \in A, c = ab$\\
$a$ - левый делитель $c$\\
$b$ - правый делитель $c$\\
$ 0 = a, 0 = 0 \cdot b$ \\
    
\begin{Def}
	$a, b$ - нетривиальные делители нуля, eсли $0 = ab, a \neq 0, b \neq 0$\\
\end{Def}
	
\begin{Def}	
	Область целостности - коммутативное кольцо с $1$, без нетривиальных делителей нуля\\
\end{Def}

\begin{Rem}
	Поле - область целостности ($\forall a, b (ab = 0 \Ra a = 0 \vee b = 0)$)\\
\end{Rem}
	
\begin{theorem}{}
	$A$ - область целостности $a \in A \setminus \lbrace 0 \rbrace$\\
	$ab = ac \Ra b = c$
\end{theorem}
	
\begin{proof}
	$$ \underbrace{a}_{neq 0}(b - c) = 0 \Ra b - c = 0 \Ra b = c$$
\end{proof}	

\section{Бесконечно малые и большие}

\begin{Def}
$$\lim x_n = +\infty \LraDef \forall E\; \exists N\colon \forall n > N\; x_n > E$$
$$\lim x_n = -\infty \LraDef \forall E\; \exists N\colon \forall n > N\; x_n < E$$
$$\lim x_n = \infty \LraDef \forall E\; \exists N\colon \forall n > N\; \left|x_n\right| > E$$
\end{Def}
\begin{Rem}
$$\left[\begin{array}{ll}\lim x_n = +\infty\\\lim x_n = -\infty\end{array}\right.\Ra \lim x_n = \infty$$
Также заметим, что обратное неверно ($x_n = (-1)^n n$).
\end{Rem}

\begin{Rem}
$\lim x_n = \infty \Ra x_n\text{ неограниченна}$
\end{Rem}
\begin{Rem}
Единтсвенность предела справедлива и расширенная на $\pm \infty$.
\end{Rem}
\begin{Rem}
Теорема о двух миллиционерах справедлива и для бесконечно больших.
\end{Rem}

\begin{Rem}
${\bar\R} = \R \cup \{+\infty, -\infty\}$
\begin{enumerate}
\item $\pm c+\pm\infty = \pm\infty$
\item $\pm c-\pm\infty = \mp\infty$
\item $c>0\colon \pm \infty \times c = \pm \infty$
\item $c<0\colon \pm \infty \times c = \mp \infty$
\item $c>0\colon \frac{\pm \infty}{c} = \pm \infty$
\item $c<0\colon \frac{\pm \infty}{c} = \mp \infty$
\item $\frac{c}{\pm \infty} = 0$
\item $(+\infty) + (+\infty) = +\infty$
\item $(+\infty) - (-\infty) = +\infty$
\item $(-\infty) + (-\infty) = -\infty$
\item $(-\infty) - (+\infty) = -\infty$
\item $\pm \infty \times (+ \infty) = \pm \infty$
\item $\pm \infty \times (- \infty) = \mp \infty$
\end{enumerate}
\end{Rem}

\begin{Def}
Последовательность называют бесконечно большой, если её предел бесконечнен.
\end{Def}
\begin{Def}
Последовательность называют бесконечно малой, если её предел равен нулю.
\end{Def}
\section{Кольца многочленов}
	\begin{Def}
		$A$ - коммутативное кольцо с $1$\\
		$A[x] = \lbrace \underbrace{a_1, a_2, .. }_{счётная последовательность} | a_i \in A,$ почти все нули $\rbrace$\\
		$A[x]$ - кольцо многочленов от одной переменной над кольцом $A$\\
	\end{Def}
		 
	\begin{Rem}
		<<почти все>> ~--- все кроме конечного числа\\
	\end{Rem}
	 
	\begin{Def}	
		$ "+": (a_1, a_2, ...) + (b_1, b_2, ...) = (a_1 + b_1, a_2 + b_2, ...) $\\
	\end{Def}
	
	\begin{Rem}
		$ \exists n, m: a_i = 0, b_i = 0 \forall i > max(n, m) \Ra a_i + b_j = 0 $\\
	\end{Rem}
	 
	\begin{Def}	
		$ "\cdot": (a_1, a_2, ...) \cdot (b_1, b_2, ...) = (c_1, c_2, ...) $\\
		где $ c_n = \sum_{i = 0}^n a_ib_{n - i} = \sum_{i + j = n} a_ib_j $\\
	\end{Def}
		 
	\begin{Rem}
		$ \exists n, m: a_i = 0, b_j = 0 \forall i > n, j > m \Ra \forall k > n + m \Ra c_k = 0 $\\
	\end{Rem}
	
	\begin{proof}
		$$ c_k = \sum_{i = 0}^k a_ib_{k - i} = \underbrace{\sum_{i=0}^n a_ib_{k - i}}_{ 0 \leq i \leq n \Ra
		 k - i \geq k - n \geq n + m - n = m \Ra b_{k - i} = 0 \Ra \sum = 0} +
		 \underbrace{\sum_{i = n + 1}^k a_ib_i}_{i > n \Ra a_i = 0 \Ra \sum = 0} = 0 $$
	\end{proof}	
	
	\begin{theorem}{}
		$(A[x], +, \cdot)$ - коммутативное кольцо с $1$
	\end{theorem}

	\begin{proof}
		\begin{enumerate}
			\item аксиомы 1 - 4 покомпонентно выполнены в $A$
			\item $\exists 0 = (0, 0, 0, ...)$
			\item $\exists 1 = (1, 0, 0, ...)$ 
				$ 1 \cdot \alpha = \alpha \cdot 1 = \alpha $
			\begin{proof}
				по определению операции умножения:\\
				$ (a_0, \underbrace{a_0 + a_1 \cdot 1}_{a_1}, \underbrace{...}_{a_2}) = \alpha $
			\end{proof}
			\item коммутативность:\\
			$\beta = (b_0, b_1, ...), \alpha = (a_0, a_1, ...) \Ra \alpha\beta = \beta\alpha$
			\begin{proof}
				$$ \alpha\beta = (c_0, c_1, ...) \Ra c_k = \sum_{i = 0}^k a_ib_{k - i} $$
				$$ \beta\alpha = (d_0, d_1, ...) \Ra d_k = \sum_{i = 0}^k b_ia_{k - i} = \sum_{j = 0}^k b_{k - j}a_j
				|j = k - i, i = k - j| =$$
				$$ \underbrace{\sum_{i = 0}^k b_{k - i}a_i  = \sum_{i = 0}^k a_ib_{k - i}}_{т.к. A - коммутативно} = c_k $$
			\end{proof}
			\item дистрибутивность (упражнение)
			\item ассоциативность:\\
			$\alpha = (a_0, a_1, ...), \beta = (b_0, b_1, ...), \gamma = (c_0, c_1, ...)$\\
			$\alpha\beta = d, (\alpha\beta)\gamma = e$\\
			$\beta\gamma = f, \alpha(\beta\gamma) = g$\\
			$e_k = g_k \forall k$
			\begin{proof}
				$$ e_k = \sum_{i = 0}^k f_ic_{k-i} = \sum_{i = 0}^k(\sum_{j = 0}^i a_jb_{i-j})c_{k-i} = $$
				\textit{меняем порядок суммирования}
				$$ = \sum_{j = 0}^k(\sum_{i = j}^k a_jb_{i - j}c_{k - i}) = \sum_{j = 0}^k a_j(\sum_{i = j}^k b_{i - j}c_{k - i}) = $$
				\textit{делаем замену $l = i - j$}
				$$ = \sum_{j = 0}^k a_j(\sum_{l = 0}^{k - j} b_lc_{k - l - j}) = \sum_{j = 0}^k a_jf_{k-j} = g_k$$
			\end{proof}
		\end{enumerate}
	\end{proof}

\section{Степень многочлена}	
\begin{Rem}
	Альтернативное обозначение многочленов (тут $a \in A$, а $x$ "--- переменная):
	\begin{align*}
	a &= (a, 0, 0, \dots); \\
	x &= (0, 1, 0, \dots); \\
	x^i &= (0, \dots, \underbrace{1}_{i-\text{я позиция}}, \dots); \\
	(a_0, a_1, a_2, \dots) &= (a_0, 0, 0, \dots) + (0, a_1, 0, 0, \dots) + \dots =  \\
    &= (a_0, 0, 0, \dots) \cdot (1, 0, 0, \dots) + (0, a_1, 0, 0, \dots) \cdot (0, 1, 0, 0, \dots) + \dots = \\
    &= a_0 + a_1x + a_2x^2 + \dots + a_nx^n
	\end{align*}
	Последняя строка "--- общий способ записать многочлен.
\end{Rem}
	
\begin{Def}
	Альтернативное определение кольца многочленов:
	\[A[x] = \lbrace a_0 + a_1x + \dots + a_nx^n \mid n \in \N \cup \lbrace 0 \rbrace \wedge a_i \in A \rbrace\]
\end{Def}
	
\begin{Def}
	Пусть $f = a_0 + a_1x + \dots + a_nx^n$, $a_n \neq 0$ и $f \neq 0$.
	Тогда $n$ "--- степень многочлена $f$, обозначается $n = \deg f$.
	Если $f = 0$, то положим $\deg f = -\infty$.
\end{Def}
	 
\begin{theorem}{}
	\begin{enumerate}
	\item $deg(f + g) \leq \max(\deg f, \deg g)$
	\item $deg(fg)  \leq \deg f + \deg g$
	\begin{Rem}
		Если $A$ - область целостности, то $\deg (fg) = \deg f + \deg g$
	\end{Rem}
	\end{enumerate}		
\end{theorem}			 

\begin{proof}
	\begin{enumerate}
	\item Следует из доказательства замкнутости относительно сложения (просто сложили и посмотрели на определение степени):
	\begin{gather*}
	f = a_0 + \dots + a_nx^n \wedge a_n \neq 0;\\
	g = b_0 + \dots + b_mx^m \wedge a_m \neq 0;
	\end{gather*}
	\item Раскрыли скобки по дистрибутивности кольца многочленов, обнаружили, что там не может возникнуть
	      $x^k$, где $k > \deg f + \deg g$.
	\item Для области целостности. Рассмотрим $a_n\neq 0$ и $b_m \neq 0$ и пусть $fg$ имеет вид
	\[ c_0x^0 + c_1x^1 + \dots + c_{n+m}x^{n+m} \]
	Так как $c_{n+m}=a_nb_m$, то $c_{n+m} \neq 0$ (иначе есть нетривиальные делители нуля), то
	есть $\deg (fg) = \deg f + \deg g$.
	Если же один из многочленов ноль, то его степень равна $-\infty$, сумма степеней тоже равна
	$-\infty$, а результат ноль $\Ra$ его степень равна $-\infty$.
	\end{enumerate}
\end{proof}	 

\begin{conseq}
	Если $A$ "--- область целостности, то и $A[x]$ "--- область целостности
\end{conseq}
\begin{proof}
	Пусть $f, g \neq 0$, тогда:
	\[
	\deg f, \deg g \geq 0
	\Ra \deg (fg) \geq 0
	\Ra  fg \neq 0	
	\]
\end{proof}
	 
\begin{exmp}
	Возьмём $A = \Z / 4 \Z$ (кольцо остатков по модулю 4, не область целостности)
	положим $f = 2x$, $g = 2x^2$, $fg = 4x^2 = 0$, то есть $A[x]$ "--- не область целостности.
\end{exmp} 
	 
\begin{conseq}
	$A$ "--- область целостности $\Ra (A[x])^* = A^*$ 
\end{conseq}	 
\begin{proof}
\begin{itemize}
\item $\subset$: пусть $fg=1$ (при этом $f,g \neq 0$, иначе точно необратимо), тогда:
	\begin{equation*}
	\left.
		\begin{aligned}
		\deg f + \deg g &= 0 \\
		\deg f &\ge 0 \\
		\deg g &\ge 0
		\end{aligned}
	\right\} \Ra \deg f = \deg g = 0
	\end{equation*}
\item $\supset$: если элемент обратим в кольце $A$, то он обратим и в кольце многочленов
\end{itemize}
\end{proof}

\section{Теорема о деление с остатком}
\begin{theorem}{}
	$A$ "--- коммутативное кольцо с $1$б
	$f, g \in A[x]$ и
	\[f = a_0 + a_1x + a_2x^2 + \dots + a_nx^n\]
	
	где $n = \deg f$ и $a_n \in A^*$. Тогда $\exists q, r \in A[x] \colon  g = qf + r, \deg r < \deg f$
	(разделили $g$ на $f$ с остатком).
\end{theorem}

\begin{Rem}
	Если $A$ "--- область целостности, то такое представление единственно
\end{Rem}

\begin{proof}
\begin{itemize}
\item \textbf{Существование}:
	Индукция по $m = \deg g$.
	\begin{itemize}
	\item
	\textbf{База:} $m < n$. Тогда положим $q=0, r=g$.
	\item
	\textbf{Переход:} доказали для всех многочленов $\deg g < m$, докажем для $m$
	\begin{gather*}
	g = b_mx^m + ... + b_0; \\
	g_1 = g - b_m a_n^{-1}x^{m-n}f;
	\end{gather*}
	коэффицент при $x^m$ в $g_1$: $b_m - b_m a_n^{-1} a_n = 0 \Ra \deg g_1 < m$\\
	по предположению индукции $g_1 = fq_1 + r_1, \deg r_1	< \deg f $, тогда разделим $g_1$ на $f$ с остатком и положим:
	\begin{gather*}
		r = r_1; \\
		q = q_1 + b_m a_n^{-1} x^{m-n}; \\
		g = fq	+ r;
	\end{gather*}
	\end{itemize}

\item	
	\textbf{Единственность:}
	Знаем, что $A$ "--- область целостности. Пусть представление не единственно.

	\begin{gather*}
	g = fq + r = f \tilde{q} + \tilde{r}; \\
	\deg r, \deg \tilde{r} < f; \\
	f (q - \tilde{q}) = \tilde{r} - r;
	\end{gather*}

	Если $q - \tilde{q} \neq 0$, то степень левого многочлена $\geq \deg f$ и степень правого $< \deg f$, а
	это невозможно. $\Ra$
	$q -\tilde{q} = 0, r - \tilde{r} = 0 \Ra q = \tilde{q}, r = \tilde{r}$
\end{itemize}
\end{proof}

\begin{Rem}
	Условие обратимости старших коэффицентов существенно, пример:
	$A = \Z$, $f = 2x, g = x^2 + 1$. Разложения в $\Z[x]$ не существует, а вот в $\R[x]$ существует
	и единственно: $g=0.5x \cdot f + 1$.
\end{Rem}

\section{Теорема о пересечение семейства компактов}
\begin{theorem}{Пересечение компактных}
Дан набор компактных множеств, любое конечное пересечение которых не пусто. Тогда их пересечение не пусто.
\end{theorem}
\begin{proof}
$K_0$~--- любое их них. Пусть пересечение всех пусто. 
$$\bigcap_{\alpha\in I} K_\alpha = \emptyset$$
Тогда 
$$\bigcup_{\alpha\in I} \left(X \setminus K_\alpha\right) \supset K_0$$
Но тогда можно выбрать конечное покрытие. Тогда 
$$\bigcup_{i=1}^k \left(X \setminus K_{x_i}\right) \supset K_0$$
Но тогда 
$$\bigcap_{i=0}^k K_{x_i} = \emptyset \quad\text{противоречие}$$
\end{proof}

\begin{conseq}
Пусть есть цепочка вложенных непустых компактных. Тогда их пересечение не пусто.
\end{conseq}
\begin{Def}
Параллелепипедом на $\R^d$ и $a, b \in \R^d$ назовём
$$[a, b] = \left\{x \in \R^d \mid \forall i=1..d\: a_i \leqslant x_i \leqslant b_i\right\} \text{ (закрытый)}$$
$$(a, b) = \left\{x \in \R^d \mid \forall i=1..d\: a_i \leqslant x_i \leqslant b_i\right\} \text{ (открытый)}$$
\end{Def}

\begin{theorem}{О вложенных параллелепипедах}
$P_1 \supset P_2 \supset P_3 \supset \ldots$ имеют непустое пересечение.
\end{theorem}
\begin{proof}
Применим теорему о вложенных отрезках по каждой координате.
\end{proof}

\section{Производная многочлена}

$A$ - коммутативное кольцо с 1\\
$f \in A[x]$\\
$f = a_nx^n + ... + a_0$\\
$K \in \N, K \cdot a = \underbrace{a + ... + a}_{K раз} = \underbrace{1 + .. + 1}_{K раз}a $ \\
$0 \cdot a = 0$

\begin{Def}
	$f' = n \cdot a_n x^{n-1} + ... + 2a_2x + a_1 = \sum_k = 0^n ka_nx^{k-1}$\\
	($k = 0$ - фиктивное слагаемое, $0 \cdot a_0 x^{-1} = 0$)\\
\end{Def}

\begin{theorem}{}
\textbf{cвойства производной}
	\begin{enumerate}
		\item $(f + g)' = f' + g'$\\
		$f_1 + ... + f_k = f_1' + ... + f_k'$
		\item $c \in A, (c\cdot f)' = c\cdot f'$
		\item $(fg)' = f'g + fg'$
		\item $(f_1 \cdot f_2 \cdot ... \cdot f_k)' = f_1' \cdot f_2 \cdot ... f_k + f_1\cdot f_2'\cdot ... \cdot f_k + ... + f_1 \cdot .. \cdot f_k'$
		\item $A$ - поле, $f \in A[x]$
		\begin{itemize}
			\item $\mathrm{char} A = 0$
				$f' = 0 \Lra f = \mathrm{const}$
			\item $\mathrm{char} A = p > 0$
				$f' = 0 \Lra f \in A[x^p]$
		\end{itemize}
	\end{enumerate}
\end{theorem} 

\begin{proof}
	\begin{enumerate}
		\item упражнение
		\item упражнение
		\item доказываем по частям:
		\begin{itemize}
			\item $f = x^n, g = x^m$\\
				$$(fg)' = (x^{n + m})' = (n + m)x^{n + m -1} = nx^{n-1}x^m + x^n mx^{m-1} = f'g + fg'$$
			\item $f = x^n, g = \sum_{k = 0}^m c_k x^k$\\
				$$(fg)' = (\sum_{k=0}^m c_k x^n x^k)' = \sum_{k=0}^m c_k(x^n x^k)' =$$
				$$ = \sum_{k=0}^m c_k(f'x^k + fkx^{k-1}) = f' \sum_{k=0}^m c_k x^k + f \sum_{k=0}^m kx^{k-1}) = f'g + fg'$$
			\item $f = \sum_{k = 0}^n a_k x^k, g$ - произвольный многочлен\\
				$$ (fg)' = \sum_{k = 0}^n a_k(x^k g)' = \sum_{k = 0}^n a_k(kx^{k-1}g + x^kg') = $$
				$$ = g\sum_{k = 0}^n ka_kx^{k-1} + g'\sum_{k=0}^n a_kx^k = f'g + fg' $$			
		\end{itemize}
		\item упражнение
		\item следствие п.4
		\item 
		\begin{itemize}
			\item $\mathrm{char} A = 0$\\
				$$ f = c_0 + c_1x + ... + c_nx^n$$
				$$ 0 = f' = c_1 + 2c_2x + ... + nc_nx^{n-1} $$
				$$	\forall k, k c_k = 0 \underbrace{(1 + 1 + ... + 1)}_{\neq 0, k раз}c_k = 0 \Ra f = c_0 = \mathrm{const} $$
				\textit{обратное очевидно}
			\item $\mathrm{char} A = p > 0, f' = \sum_{k = 1}^nkc_kx^{k-1}$
				$\forall k \geq 1 kc_k = 0$
		 		Пусть $p \nmid k, k = pq + r, 1 < r <p$
		 		$$ kc_k = \underbrace{(1+1+...+1)}_{k}c_k = (\underbrace{\underbrace{1+...+1}_p+ .. + \underbrace{1+...+1}_p}_q + \underbrace{1+...+1}_r)c_k = \underbrace{(1+..+1)}_{\neq 0, r раз,  т.к. 0 < r < \mathrm{char} A}c_k $$
		 		$$ \Ra c_k = 0$$
		 		$"\La"$ $f = c_0 + c_px^p + c_{2p}x^2p + .. \in A[x^p]$\\
		 		Если $f = \sum_{j = 0}^rc_{jp}x^{jp}$, то $ f' = \sum_{j=0}^r j\underbrace{p}_{= 0, \mathrm{char} A = p}c_{jp}x^{jp-1} = 0 $
		 		
		\end{itemize}	
	\end{enumerate}
\end{proof}

\section{Кратные корни}
A "--- поле. 
$f \in A[x],f \ne 0$
c "--- корень f в A $\Lra (x - c)|f$ в A[x](теорема Безу)

\begin{Def}
Если для некоторого $k \ge 2$, $(x - c)^k|f$, но $(x - c)^{k + 1} \nmid f$, то говорим, что c "--- корень f кратности k.
\end{Def}

c "--- корень f кратности k, если  $f(x) = (x - c)^{k}g(x), (x - c) \nmid g(x) \Lra f(x) = (x - c)^{k}g(x), g(c) \ne 0$

\begin{theorem}{}
A "--- поле, $char A = 0, f \in A[x], f \ne 0$

c "--- корень f кратности $k \ge 1 \Lra$
\begin{enumerate}
\item c "--- корень f.
\item с "--- корень f' кратности k - 1.
\end{enumerate}

\end{theorem}

\begin{proof}
$$\Ra$$
$$f = (x - c)^{k}g(x), g(c) \ne 0 \Ra c \text{ "--- корень}$$
$$f' = k(x - c)^{k - 1}g(x) + (x - c)^{k}g' = (x - c)^{k - 1}(kg + (x - c)g')$$
$$\Ra (x - c)^{k - 1}|f'$$
c "--- не корень kg + (x - c)g'

$$kg(c) + (x - c)g'(c) = kg(c) \ne 0$$

$$\La$$
c "--- корень f $\Ra$ корень f кратности l, по доказаному с "--- корень f' кратности l - 1.
$$l - 1 = k - 1$$
$$l = k$$ 
\end{proof}

\begin{Rem}
Предположение $char A = 0$ существенно. 
$$\F_2, f = x^7 + x^2$$
0 "--- корень кратности 2.
$$f' = x^6$$
0 "--- кратности 6.
\end{Rem}
              
\begin{conseq}
    A "--- поле характеристики 0. $0 \ne f \in A[x]$, c "--- корень f кратности $\ge k \Lra$ выполняется равенство 
    $$0 = f(c) = f'(c) = \ldots = f^{(k - 1)}(c)$$
    $$f^{(k)} = (f^{(k - 1)})'$$
\end{conseq}

$$(fg)^(n) = \sum_{r = 0}^{n}C_n^r f^{(r)}g^{(n - r)}$$


\section{Секвенциальная компактность}

\begin{theorem}{Компактность в $\R^d$}
Следующее в $\R^d$ равносильно:
\begin{enumerate}
\item Компактно
\item Замкнуто и ограниченно
\item Для любой последовательности в множестве можно выбрать подпоследовательность, сходящуюсю к некоторой точке множества (\textit{секвенциально компактно})
\end{enumerate}
\end{theorem}
\begin{proof}
$2 \Ra 1$: $К$ ограниченно, значит можно его ограничить кубом, значит оно подмножество компактного и замкнуто, значит компактно.

$1 \Ra 3$:
Возьмём последовательность $\{x_n\}\lrh E$ элементов множества $F$. Если множество элементов $E$ конечно, то какой-то элемент повторился бесконечно. Возьмём новую стационарную последовательность ровно из этого элемента, имеющую предел. Если же оно бесконечно, докажем, что у него есть предельная точка.

Пусть ни одна точка не предельна. Значит 
$$\forall x \in X\: \exists r_x > 0\colon \dot B_{r_x}(x) \cap F = \emptyset$$
Но тогда возьмём покрытие
$$\bigcup_{x\in X} B_{r_x} (x)$$
В нём есть конечное подпокрытие. Возьмём его
$$\bigcup_{i=1}^k \dot B_{r_{y_i}} \supset K \supset E$$
Но также
$$\bigcup \dot B_{r_{y_i}} \cap E = \varnothing$$
Значит 
$$E \subset \bigcup_{i=1}^k \{y_i\}$$
Получили, что $E$ конечное. 

Таким образом предельная точка существует, а значит можно выбрать подпоследовательность можно.

$3 \Ra 2$:
Пусть $K$ не замкнуто. Возьмём предельную точку, которой нет в $K$. Значит есть последовательность, сходящаяся к ней. Из неё нельзя выбрать подпоследовательность, сходящуюся к элементу $K$.

Пусть $K$ не ограничено. Значит есть точка, не лежащая в данном шарике.
$$K \nsubset B_1(a) \Ra \exists x_1\colon \rho(x_1, a) > 1$$
$$K \nsubset B_{\rho(a, x_1) + 1}(a) \Ra \exists x_2\colon \rho(x_2, a) > \rho(x_1, a) + 1$$
$$ \vdots $$
Рассмотрим сходящуюся подпоследовательность. Она ограничена шариком радиуса $R$. Но
$$\rho(a, x_n) > \rho(a, x_{n-1}) + 1 > \cdots > n$$
$$R > \rho\left(b, x_{n_k}\right) > \rho\left(a, x_{n_k}\right) + \rho(a, b) > n_k + \rho(a, b) \ra \infty$$
Значит $K$ ограниченно.  
\end{proof}

\begin{Rem}
$1\Ra 3; 3\Ra 2; 1 \Ra 2$ справедливы для всех пространств. $2 \Ra 1$ ломается, например, на $\R$ с дискретной метрикой.
\end{Rem}

\section{Алгебраические замкнутые поля}
\begin{Def}
Поле А "--- алгебраически замкнуто, если любой $f \in A[x] \backslash A$ имеет в А хотя бы 1 корень.
\end{Def}

\begin{theorem}{}
Следующие условия равносильны. 
\begin{enumerate}
\item A "--- алгебраически замкнуто. 
\item $\forall f \in A[x]$ c $\deg f \ge 1$ делится на линейный многочлен. 
\item $\forall f \in A[x]$ c $\deg f \ge 1$ имеет $deg f$ корней (с учетом кратности).
\item $\forall f \in A[x]$ c $deg f \ge 1$ полностью раскладывается на линейные множества в колце многочленов.
\end{enumerate}
\end{theorem}

\begin{proof}
$1 \Lra 2$(следствие теоремы Безу)

$3 \Ra 1$ очевидно.

$1 \Ra 3$ Индукция и $deg f$

\begin{enumerate}
\item {\bf База:} $deg f = 1$ 

$ax = b$

$x = \frac{b}{a}$ "--- корень.
\item {\bf Переход:} $f deg f \ge 2$

$\exists c \in A$ корень f кратности $k \ge 1, f = (x - c)^{k}g$

По индукционному предположению число корней g = deg g.

Все корни f отличные от с это в точности корни g, причем той же кратности. 

Число корней $f = k + $ число корней g = k + deg g = deg f.

$4 \Ra 2$ очевидно.

$2 \Ra 4$ индукция по deg f.


\end{enumerate}

\end{proof}

\noindent \underline{\hbox to 1\textwidth{{ } \hfil{ } \hfil{ } }}
\end{document}
