\section{Предельный переход в неравенстве}
\begin{theorem}{Предельный переход в неравенстве}
Пусть $x_n, y_n \in \R; x = \lim x_n; y = \lim y_n; x_n \leqslant y_n$ (или $x_n < y_n$). Тогда $x \leqslant y$.
\end{theorem}
\begin{proof}
Пусть $y < x$; $\epsilon \lrh \frac{x - y}2$. Тогда 
$$\exists N_1: \forall n \geqslant N_1\: |x - x_n| < \epsilon$$
$$\exists N_2: \forall n \geqslant N_2\: |y - y_n| < \epsilon$$
Тогда
$$\forall n \geqslant \max\{N_1, N_2\}\: x_n > x - \epsilon = y + \epsilon > y_n$$
\end{proof}
\begin{Rem}
Понятно, что можно потребовать отношение между последовательностями только с некоторого номера.
\end{Rem}                                                 
\begin{Rem}
Строгие неравенства не сохраняются.
\end{Rem}
\begin{conseq}
$x_n \leqslant b \Ra x \leqslant b$
\end{conseq}
\begin{conseq}
$x_n \geqslant a \Ra x \geqslant a$
\end{conseq}
\begin{conseq}
$x_n \in [a;b] \Ra x \in [a; b]$
\end{conseq}