\section{Непрерывность отображений из метрического пространства в $\R^m$}

\begin{theorem}{ааа}

 $f: E \subset X \to \R^{d} a \in E$

 Тогда f  непрерывна в точке a $\Lra$ все координаты функции f непрерывны в точке a.
 
\end{theorem}

\begin{proof}
 \begin{enumerate}
 \item $\Ra$

  $$\forall \epsilon > 0 \exists \delta > 0 \forall x \in B_{\delta}(a) \cap E: f(x) \in B_{\epsilon}(f(a)) $$

  то есть 

  $$\rho(f(x), f(a)) < \epsilon$$

  $$|f_i(x) - f_i(a)| \leqslant \sqrt{((f_1(x) - f_1(a))^2 + (f_2(x) - f_2(a))^2 + \ldots)}$$

  $$\Ra |f_i(x) - f_i(a)| < \epsilon \Ra f_i \in B_{\epsilon}(f_i(a)) $$

  $$f_i ~---$$ непрерывна в точке a.
 \item $\La$
  
  Возьмем $\delta = \min\{\delta_1, \ldots, \delta_d\} > 0$

  Тогда $\forall x \in B_{\delta}(a) \forall i = 1 \ldots d |f_i(x) - f_i(a)| < \epsilon$

  $\Ra (f_1(x) - f_1(a))^2 + \ldots < d \epsilon$

 $\rho(f(x), f(a)) < \sqrt{(d)} \epsilon$ 
 
 \end{enumerate}
\end{proof}