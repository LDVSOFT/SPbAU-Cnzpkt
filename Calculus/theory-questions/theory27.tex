\section{Теорема Гейне-Бореля}

\begin{theorem}{Теорема Гейне-Бореля}
Замкнутый куб компактен
\end{theorem}
\begin{proof}
$$I = \left\{x \in \R^d \mid \forall i=1..d\: 0 \leqslant x_i \leqslant a\right\}$$
Рассмотрим произвольное покрытие. Пусть из него нельзя выбрать конечное подпокрытие. Тогда разобъём куб по кажому измерению пополам. Хотя бы один из результирующих не покрываем. 
Повторим процесс до бесконечности. У них есть точка в пересечении. Но она тогда есть покрывающее её множество. Оно открыто, а значит оно покроет ещё и некоторый хвост подкубов.
Ну а тогда возьмём его и все вышестоящие покрытия. Результат конечен и покрыл куб.
\end{proof}
