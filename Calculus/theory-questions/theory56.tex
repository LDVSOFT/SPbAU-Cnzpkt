\section{Непрерывность тригонометрических функций}
\begin{theorem}{}
$$\forall x \in \left(0; \frac{\pi}2\right)\: \sin x < x < \tg x$$
\end{theorem}
\begin{proof}
\begin{wrapfigure}{r}{6.0cm}
\def\svgwidth{6.0cm}
\input{theory56-sin.pdf_tex}
\end{wrapfigure}
$$S_{AOB} < S_{\text{сектор}} < S_{AOC}$$
\begin{align*}
S_{AOB} &= \frac12 \sin x \\
S_{\text{сектор}} &= \frac{x}2 \\
S_{AOC} &= \frac12 \tg x
\end{align*}
$$\sin x < x < \tg x$$
\end{proof}
                        
\begin{conseq}
$\sin$ и $\cos$ непрерывны. 
\end{conseq}
\begin{proof}
$$\left|\sin x - \sin y\right| = 2 \left|\sin \frac{x-y}2\right| \left|\cos \frac{x+y}2\right| \leqslant \left|x - y\right|$$
\end{proof}

\begin{conseq}
$\tg$ и $\ctg$ непрерывны. 
\end{conseq}

\begin{conseq}
$$\sin \uparrow   \left[-\frac\pi2,\frac\pi2\right]$$
$$\cos \downarrow \left[0,\pi\right]$$
$$\tg  \uparrow   \left(-\frac\pi2,\frac\pi2\right)$$
\end{conseq}

\begin{Def}
$$ \arcsin = \left(\sin \mid_{\left[-\frac\pi2,\frac\pi2\right]}\right)^{-1} $$
$$ \arccos = \left(\cos \mid_{\left[0,\pi\right]}\right)^{-1} $$
$$ \arctg  = \left(\tg  \mid_{\left(-\frac\pi2,\frac\pi2\right)}\right)^{-1} $$
\end{Def}

\begin{theorem}{}
$$\lim_{x \ra 0} \frac{\sin x}x = 1$$
\end{theorem}
\begin{proof}
$0 < x < \frac\pi2$:
$$\sin x < x < \tg x \Ra \frac{\sin x}x < 1 < \frac1{\cos x} \frac{\sin x}x \Ra \cos x < \frac{\sin x}x < 1 \xra{x\ra0} 1 \leqslant \lim_{x\ra0} \frac{\sin x}x \leqslant 1$$
\end{proof}