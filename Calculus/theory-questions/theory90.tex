\section{Таблица интегралов}

Таблица интегралов:
\begin{align*}
\int 0\d x &= c \\
\int x^p\d x &= \frac{x^{p+1}}{p + 1} + c \\
\int \frac{\d x}{x} &= \ln |x| + c \\
\int a^x \d x &= \frac{a^x}{\ln a} + c \\
\int \sin x \d x &= -\cos x + c \\
\int \cos x \d x &= \sin x + c \\
\int \frac{\d x}{\cos^2 x} &= \tg x + c \\
\int \frac{\d x}{\sin^2 x} &= -\ctg x + c \\
\int \frac{\d x}{\sqrt{1 - x^2}} &= \arccos x + c \\
\int \frac{\d x}{1 + x^2} &= \arctg x + c \\
\int \frac{\d x}{1 - x^2} &= \frac12 \ln \left|\frac{1+x}{1-x}\right| + c\\
\int \frac{\d x}{\sqrt{x^2 \pm 1}} &= \ln \left|x + \sqrt{x^2 \pm 1}\right| + c
\end{align*}

\begin{Def}
Пусть $A, B$~--- множества. Тогда
$$A + B = \left\{a + b \mid a \in A \land b \in B\right\}$$
$$A - B = \left\{a - b \mid a \in A \land b \in B\right\}$$
$$\alpha A = \left\{\alpha a \mid a \in A\right\}$$
\end{Def}

\begin{theorem}{Об арифметических операциях с интегралами}
$$\int (f \pm g) \d x = \int f \d x \pm \int g \d x$$
$\alpha \ne 0$
$$\int \alpha f \d x = \alpha \int f \d x$$
\end{theorem}
\begin{Rem}
Именно из-за того, что константы в записи нет, мы исключаем ноль.
\end{Rem}
\begin{proof}
$F, G$~--- первообразные соотвественно $f, g$.
$$\int f \d x = \left\{F + c_1\right\}$$
$$\int g \d x = \left\{G + c_2\right\}$$
$$\int f \d x \pm \int g \d x = \left\{F + c_1\right\} \pm \left\{G + c_2\right\} = \left\{F+G+c_3\right\} = $$
$(F+G)' = f + g$
$$ = \int (f+g)\d x$$
$$\alpha \int f \d x = \alpha \left\{F + c_1\right\} = \left\{\alpha F + c_2\right\} = $$
$(\alpha F)' = \alpha f$
$$ = \int \alpha f\d x$$
\end{proof}

