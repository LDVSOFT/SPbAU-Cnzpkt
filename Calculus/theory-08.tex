\begin{Rem}
Если в определении по Гейне все пределы существуют, то они будут равны.
\end{Rem}
\begin{proof}
Возьмём две сходящиеся последовательности $x_n$ и $y_n$, после применения функций стремящиеся к каким-то разным значениям $b$ и $c$. Но тогда
у последовательности 
$$x_1, y_1, x_2, y_2, x_3, y_3, \ldots$$
сходящейся к той же точке, будет предел. Но тогда у подпоследовательностей одинаковые пределы.
\end{proof}

\begin{assertion}{Единственность предела}
$f\colon E \subset X \ra Y$, $a$~--- предельная точка. Тогда предел $\lim\limits_{x\ra a} f(x)$ единственнен.
\end{assertion}
\begin{proof}
Пусть есть два различных предела. Тогда из определения по Коши с какого-то расстояния весь хвост должен быть ближе к одному пределу, чем к другому.
\end{proof}

\begin{theorem}{Ограниченность}
$f\colon E \subset X \ra Y$, $\lim\limits_{x\ra a} = b$. Тогда 
$$\exists r>0\colon f \mid_{E \cap B_r(x)}\text{ограничена}$$
\end{theorem}

\begin{theorem}{Уход от нуля}
$f\colon E \ra \R^d$, $\lim\limits_{x\ra a} = b \ne \vec 0$. Тогда
$$\exists r>0\colon \forall x \in \dot B_r(a) \cap E\: f(x) \ne 0$$
\end{theorem}
\begin{proof}
$\epsilon \lrh \rho(x, \vec 0)$
\end{proof}

\begin{theorem}{Арифметические свойства предела функции}
$f, g\colon E \subset \ra \R^d$, $\lambda\colon E \ra \R$, $a$ предельная точка $E$.
\begin{enumerate}
\item $\lim\limits_{x\ra a} (f(x) + g(x)) = f_0 + g_0$
\item $\lim\limits_{x\ra a} (\lambda(x)  g(x)) = \lambda_0 g_0$
\item $\lim\limits_{x\ra a} (f(x) - g(x)) = f_0 - g_0$
\item $\lim\limits_{x\ra a} \left\|f(x)\right\| = \left\|f_0\right\|$
\item $\lim\limits_{x\ra a} \left<f(x), g(x)\right> = \left<f_0, g_0\right>$
\end{enumerate}
\end{theorem}
\begin{proof}
Возьмём любые сходящиеся к $a$ последовательности. Для них будет справедлива теорема об арифметических действиях с пределами последовательности.
\end{proof}

\begin{theorem}{Арифметические свойства предела функции}
$f, g\colon E \subset \ra \R$, $a$ предельная точка $E$.
\begin{enumerate}
\item $\lim\limits_{x\ra a} (f(x) \pm g(x)) = f_0 \pm g_0$
\item $\lim\limits_{x\ra a} (f(x) g(x)) = f_0 g_0$
\item $\lim\limits_{x\ra a} \left|f(x)\right| = \left|f_0\right|$
\item $\lim\limits_{x\ra a} \cfrac{f(x)}{g(x)}=\cfrac{f_0}{g_0}$
\end{enumerate}
\end{theorem}
\begin{proof}
Аналогично.
\end{proof}

\begin{Rem}
Арифметические свойства расширяются на бесконечности.
\end{Rem}

\begin{theorem}{Предельный переход в неравенстве}
$f, g\colon E \ra Y$, $a$ предельная точка $E$, $\forall x \in E\setminus \{a\} f(x) \leqslant g(x)$. Тогда 
$$f_0 \leqslant g_0$$
\end{theorem}

\begin{theorem}{О двух миллиционерах}
$f, g, h\colon E \ra Y$, $a$ предельная точка $E$, $f(x) \leqslant g(x) \leqslant h(x)$, $\lim\limits_{x\ra a} f(x) = \lim\limits_{x\ra a} h(x) = b$. Тогда
$$\lim_{x\ra a} g(x) = b$$
\end{theorem}

\begin{Def}
Пределы слева и справа. $f\colon E \cap \R \ra Y$.
$$\lim\limits_{x\ra a-} = \lim\limits_{x\ra a-0} \eqDef \lim\limits_{x\ra a} f\mid_{E\cap(-\inf,a)}$$
$$\lim\limits_{x\ra a+} = \lim\limits_{x\ra a+0} \eqDef \lim\limits_{x\ra a} f\mid_{E\cap(a,+\inf)}$$
\end{Def}

\begin{theorem}{Существование предела возрастающей и ограниченой функции}

\end{theorem}

\begin{theorem}{Критерий Коши}
Функция с полной областью значений имеет предел в точке тогда и только тогда, когда для любого разброса существует выколотый шарик вокруг предельной точки, все расстояния в котором малы.
\end{theorem}
