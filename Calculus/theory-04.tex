\section{Конечное векторное пространство}

\begin{Def}
Вектор~--- кортеж $x = (x_1, x_2, \ldots, x_d) \in \R^d$. Операция сложения 
$$+\colon \R^d \times \R^d \ra \R^d;x+y = (x_1+y_1, x_2+y_2, \ldots, x_d + y_d)$$ 
и умножения 
$$\times\colon \R \times \R^d \ra \R^d; \lambda x = (\lambda x_1, \lambda x_2, \ldots, \lambda x_n)$$
\end{Def}
\begin{enumerate}
\item Сложение
\begin{enumerate}
\item Коммутативно
\item Ассоциативно
\item Существует ноль $\vec 0 = \underbrace{(0, 0, \ldots, 0)}_d$
\item Существует обратный элемент
\end{enumerate}
\item $\alpha (x + y) = \alpha x + \alpha y$
\item $(\alpha + \beta) x = \alpha x + \beta x$
\item $(\alpha\beta)x = \alpha(\beta x)$
\item $1x = x$
\end{enumerate}
\begin{Def}
Общее определение векторного пространства~--- всё то же самое, но без конкретики.
\end{Def}

\begin{Def}
Скалярное произведение векторов (евклидово):
$$\langle x, y\rangle = \sum_{i=1}^d x_iy_i$$
\end{Def}
Свойства:~%
\begin{enumerate}
\item $\langle x, x\rangle \geqslant 0; \langle x, x\rangle = 0 \Lra x = \vec 0$
\item $\langle \lambda x, y\rangle = \lambda \langle x, y\rangle$
\item $\langle x, y\rangle = \langle y, x\rangle$
\item $\langle x + y, z\rangle = \langle x, z\rangle + \langle y, z\rangle$
\end{enumerate}

\begin{Def}
Общее определение скалярного произведения: $X$~--- веторное пространство. Задана операция $\langle x,y\rangle\colon X \times X \ra \R$ обладающая указынными свойствами.
\end{Def}
Например, если приписать в определение положительную константу~--- ничего не поменяется.

\begin{Def}
(Евклидова) норма:
$$\|x\| = \sqrt{\langle x, x\rangle}$$
\end{Def}
\begin{enumerate}
\item $\|x\| \geqslant 0; \|x\| = 0 \Lra x = \vec 0$
\item $\|\lambda x\| = |\lambda| \|x\|$
\item $|\langle x,y\rangle| \leqslant \|x\|\|y\|$ (нер-во Коши--Вуняковкского)
\item $\|x + y\| \leqslant \|x\| + \|y\|$ (нер-во треугольника)
\item $\|x - z\| \leqslant \|x - y\| + \|y - z\|$ (нер-во Минковского)
\item $\|x - y\| \geqslant \left|\|x\| - \|y\|\right|$
\begin{proof}
$\|x - y\| = \|y - x\|$. Таким образом достаточно показать, что 
$$\|x - y\| \geqslant \|x\| - \|y\| \La \|x - y\| + \|y\| \geqslant \|x\|$$
А это неравнство треугольника.
\end{proof}
\item $\rho(x, y) = \|x - y\|$~--- метрика. Это ровно евклидово пространтво на $\R^d$.
\end{enumerate}

\begin{Def}
Общее определение нормы: $\|x\|\colon X \Ra \R$, обладает свойствами 1, 2 и 4.
\end{Def}
Свойство 3 касается скаляроного произведения, которого может и не быть.

Примеры:~%
\begin{enumerate}
\item $\|x\|_1 = \sum\limits_{k=1}^d |x_k|$
\item $\|x\|_\infty = \max\limits_{k=1..d} |x_k|$
\begin{proof}
$$\|x + y\| = \max_{k=1..d} |x_k + y_k| \leqslant \max_{k=1..d} (|x_k| + |y_k|) = |x_{k_0}| + |y_{k_0}| \leqslant \|x\| + \|y\|$$
\end{proof}
\item $$\|x\|_d = \sqrt[p]{\sum\limits_{k=1}^d |x_k|^p}$$
\end{enumerate}

\section{Арифметические свойства предела}
Пусть есть $(\R^d, \rho)$ со стандартной метрикой и нормой.

\textbf{Утверждение.} $x_n \in \R^d$. $$\lim_{n\ra\infty} x_n = \vec 0 \Lra \lim_{n\ra\infty} \|x_n\| = 0$$
\begin{proof}
$$\lim x_n = 0 \Lra \forall \epsilon > 0\;\exists N\colon\forall n>N\; \|x_n\| < \epsilon \Lra \lim \|x_n\| = 0$$
\end{proof}
\begin{Rem}
$A \subset \R^d\text{ ограниченно} \Lra \exists M\colon \forall x \in A\: \|x\| \leqslant M$
\end{Rem}

\begin{theorem}{Арифметические свойства предела}
$x_n, y_n \in \R^d$, $\lambda \in \R$, $\lim x_n = x_0$, $\lim y_n = y_0$, $\lim \lambda = \lambda_0$.
\begin{enumerate}
\item $\lim (x_n + y_n) = x_0 + y_0$
\item $\lim (\lambda x_n) = \lambda_0x_0$
\item $\lim (x_n - y_n) = x_0 - y_0$
\item $\lim \langle x_n, y_n\rangle = \langle x_0, y_0\rangle$
\item $\lim \|x_n\| = \|x_0\|$
\end{enumerate}
\end{theorem}
\begin{proof}
$$\forall \epsilon > 0\; \exists N_1\colon \forall n > N_1\; \|x_n - x_0\| < \epsilon$$
$$\forall \epsilon > 0\; \exists N_2\colon \forall n > N_2\; \|y_n - y_0\| < \epsilon$$
$$\forall \epsilon > 0\; \exists N_3\colon \forall n > N_3\; |\lambda - \lambda_0| < \epsilon$$
\begin{enumerate}
\item $$\forall \epsilon > 0\;\begin{cases}\|x_n-x_0\| < \epsilon \\ \|y_n-y_0\| < \epsilon\end{cases} \Ra 
\|x_n + y_n - x_0 - y_0\| \leqslant \|x_n - x_0\| + \|y_n - y_0\| < \epsilon + \epsilon = 2\epsilon$$
\item $$\|\lambda_nx_n-\lambda_0x_0\| = \|\lambda_nx_n - \lambda_nx_0 + \lambda_nx_0 - \lambda_0x_0\| \leqslant
\|\lambda_nx_n - \lambda_nx_0\|+\|\lambda_nx_0-\lambda_0x_0\| = $$
$$ = |\lambda_n| \|x_n-x_0\| + |\lambda_n - \lambda_0| \|x_0\| \leqslant
M \|x_n-x_0\| + |\lambda_n - \lambda_0| \|x_0\|$$
Но тогда
$$\forall n > \max{N_1, N_3}\; \begin{cases}\|x_n-x_0\| < \frac{\epsilon}M \\ |\lambda_n - \lambda_0| < \frac{\epsilon}{\|x_0\|}\end{cases} \Ra
\|\lambda_nx_n-\lambda_0x_0\| < \epsilon$$
\item Следствие 1 и 2
\item $x_n = \left(x_n^{(1)}, x_n^{(2)}, \ldots, x_n^{(d)}\right); y_n = \left(y_n^{(1)}, y_n^{(2)}, \ldots, y_n^{(d)}\right)$
Это докажем позже
\item $$0 \leqslant \left|\|x_n\|-\|x_0\|\right| \leqslant \|x_n-x_0\| \longrightarrow 0 \Ra \|x_n\| - \|x_0\| \longrightarrow 0 \Ra \|x_n\| \longrightarrow \|x_0\|$$
\end{enumerate}
\end{proof}

\begin{theorem}{Свойства предела на вещественных}
$x_n, y_n \in \R; \lim x_n = x_0; \lim y_n = y_0$
\begin{enumerate}
\item $\lim (x_n + y_n) = x_0 + y_0$
\item $\lim x_ny_n = x_0y_0$
\item $\lim (x_n - y_n) = x_0 - y_0$
\item $\lim |x_n| = |x_0|$
\item Если $y_n, y_0 \ne 0$, то $\lim \frac{x_n}{y_n} = \frac{x_0}{y_0}$
\end{enumerate}
\end{theorem}
\begin{proof}
Докажем, что $\lim \frac1{y_n} = \frac1{y_0}$.
$$\left|\frac{1}{y_n} - \frac{1}{y_0}\right| = \frac{|y_n - y_0|}{|y_n||y_0|} \lrh A$$
$$\exists N_1\colon \forall n > N_1\; |y_n-y_0| < \frac{|y0|}2 \Ra |y_n| \geqslant |y_0| - |y_0 - y_n| > |y_0| - y_0 / 2 = |y_0|/2$$
Тогда
$$A < \frac{|y_n - y_0|}{\frac{|y_0|}2 |y_0|} < \frac{\frac{\epsilon|y_0|^2}2}{\frac{|y_0|}2 |y_0|}$$
\end{proof}
