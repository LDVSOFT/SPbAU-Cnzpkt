\documentclass[12pt,a4paper]{book}
\usepackage{polyglossia}
\usepackage{fontspec}
\usepackage{metalogo}
\usepackage{amsmath, amssymb}
\usepackage{xcolor}
\usepackage[russian]{hyperref}
\usepackage{indentfirst}
\usepackage{datetime}
\usepackage{extarrows}
\usepackage{fancyvrb}
\usepackage[left=1cm,right=1cm,top=2cm,bottom=2cm]{geometry}
%% WARNING: latest minted is used. Download directly from github!
%% Works fine, though
%% Hard installation!
%% \usepackage{minted}
\usepackage[math-style=ISO,vargreek-shape=unicode]{unicode-math} %% MAGIC! INCLUDE AS LAST!

\setdefaultlanguage[spelling=modern,babelshorthands=true]{russian} %% Languages for polyglossia
\setotherlanguage{english}

% \RequirePackage{cm-unicode,lm-math}

\frenchspacing %% One space before sentence, not two!

\defaultfontfeatures{Ligatures={TeX}} %% Fonts and ligatures.
\setmainfont{CMU Serif} %% There are original Knuth's fonts in Unicode, called Computer Modern Unicode. Download anywhere, just install them.
\setsansfont{CMU Sans Serif}
%\setmonofont{CMU Typewriter Text}  
\setmonofont{Courier New}
\newfontfamily\cyrillicfonttt{Courier New}
\setmathfont{Latin Modern Math} %% Download too. Very close to original Computer Modern. You may change it :)
\newcommand{\Setminus}{\mathbin{\backslash}} %% Latin Modern Math contains an Elder bug~--- no setminus(

%% Magic as black as my working table
%% Add cyrilic in math mode
\DeclareSymbolFont{cyrletters}{\encodingdefault}{\familydefault}{m}{it}
\newcommand{\makecyrmathletter}[1]{%
  \begingroup\lccode`a=#1\lowercase{\endgroup
  \Umathcode`a}="0 \csname symcyrletters\endcsname\space #1
}
\count255="409
\loop\ifnum\count255<"44F
  \advance\count255 by 1
  \makecyrmathletter{\count255}
\repeat

\def\la{\leftarrow}
\def\ra{\rightarrow}
\def\lra{\leftrightarrow}
\def\La{\Leftarrow}
\def\Ra{\Rightarrow}
\def\Lra{\Leftrightarrow}
\def\lrh{\leftrightharpoons}
\def\btu{\bigtriangleup}

\def\N{\mathbb{N}}
\def\Z{\mathbb{Z}}
\def\Q{\mathbb{Q}}
\def\R{\mathbb{R}}
\def\C{\mathbb{C}}

\def\d{\mathup{d}}

\def\LraDef{\stackrel{\mathrm{Def}}{\Lra}}
\def\eqDef{\stackrel{\mathrm{Def}}{=}}

%% Change Chapter and Section numeration style
\renewcommand{\thechapter}{\Roman{chapter}}
\renewcommand{\thesection}{\arabic{section}}

%% Environment for theorem body
\newcounter{theorem}[section]
\renewcommand{\thetheorem}{\thesection.\arabic{theorem}}
\newcommand*{\theoremheader}[1]{\par\refstepcounter{theorem}%
\textbf{Теорема \thetheorem.\ifthenelse{\equal{#1}{}}{}{ #1.}}}
\newenvironment*{theorem}[1]{%
\theoremheader{#1}%
}{%
\par%
}

%% Environment for consequence body
\newcounter{conseq}[theorem]
\renewcommand{\theconseq}{\arabic{conseq}}
\newcommand*{\conseqheader}{\par\refstepcounter{conseq}%
\textit{Следствие \thetheorem.} }
\newenvironment*{conseq}{%
\conseqheader%
}{%
\par%
}

%% Environment for proof body. I like this style, but you are free to change it.
\newenvironment{proof}{%
\par$\blacktriangleright$%
}{%
\hfill$\blacktriangleleft$\par%
}

%% Environment for definitions. Pretty raw one.
\newenvironment{Def}{%
\par$\mathfrak{Def\colon}$%
}{%
\par%
}

%% Environment for remarks.
\newenvironment{Rem}{%
\par\textit{REM: }%
}{%
\par%
}

%% In-line code highlighting. Using: \py|a = input()|
% \newmintinline[cinl]{c}{} %\c is defined :(
% \newmintinline[cpp]{cpp}{}
% \newmintinline[python]{python}{}
% \newmintinline[bash]{bash}{}
% \newmintinline[make]{make}{}

%% Escaped code highlighting. Using: \begin{cppcode} ... \end{cppcode}
% \setminted{obeytabs,tabsize=4,linenos,texcomments}
% \newminted{c}{}
% \newminted{cpp}{}
% \newminted{python}{}
% \newminted{bash}{}
% \newminted{make}{}

\newcommand{\ok}{& \textcolor{green!60!black}{Правильно}}
\newcommand{\bad}{& \textcolor{red}{Неправильно}}

\begin{document}
\begin{center}
{\huge \bf Guideline} \\
{\Large \bf по вёрстке конспектов в \XeLaTeX} \\
\vspace{0.5em}
{\large \bf Авторы: Лапшин Дмитрий, Егор Суворов} \\
\vspace{0.5em}
{Собрано: \today~\currenttime} \\
\end{center}

\section{Общие идеи}
\begin{enumerate}
\item Не стоит как-то руками выставлять пробелы (в том числе неразрывные \verb'~') и переводы строк (не используем \verb'\\' вне таблиц,
где эта комбинация является частью синтаксиса). Просто аккуратно делите на абзацы. Также не стоит
использовать \verb`\vspace` и похожие команды, которые выставляют отступы.

\item Математические символы, записи следует оборачивать в \verb'$...$'. Также следует оборачивать запятые
и прочие символы пунктуации, если они идут в одном <<математическом предложении>>. Если пунктуация
скорее относится к русскому языку, чем к математическому выражению, то её стоит вынести за доллары. Примеры:

\begin{center}\begin{tabular}{|c|c|c|}
\hline Это... & ...верстается так & \\
\hline \verb!Имеется n слагаемых! & Имеется n слагаемых \bad \\
\hline \verb!Имеется $n$ слагаемых! & Имеется $n$ слагаемых \ok \\
\hline \verb!с графиком $\Gamma_f,$ обозначим! & с графиком $\Gamma_f,$ обозначим \bad \\
\hline \verb!с графиком $\Gamma_f$, обозначим! & с графиком $\Gamma_f$, обозначим \ok \\
\hline \verb!пусть f, g непрерывны! & пусть f, g непрерывны \bad \\
\hline \verb!пусть $f, g$ непрерывны! & пусть $f, g$ непрерывны \bad \\
\hline \verb!пусть $f$, $g$ непрерывны! & пусть $f$, $g$ непрерывны \ok \\
\hline \verb!где a, b $\in \R$! & где a, b $\in \R$ \bad \\
\hline \verb!где $a$, $b \in \R$! & где $a$, $b \in \R$ \bad \\
\hline \verb!где $a$, $b$ $\in \R$! & где $a$, $b$ $\in \R$ \bad \\
\hline \verb!где $a, b \in \R$! & где $a, b \in \R$ \ok \\
\hline 
\end{tabular}\end{center}

В примере ниже это критично (хотя таких ситуаций лучше вообще избегать):
	
\begin{center}\begin{tabular}{|c|c|l|}
\hline \verb!$a < b, c < d$! & $a < b, c < d$ & Двойное неравенство на $b$, $c$ \\
\hline \verb!$a < b$, $c < d$! & $a < b$, $c < d$ & Два независимых неравенства \\
\hline
\end{tabular}\end{center}

\item Выключенные формулы (которые отдельно от текста), если их несколько подряд, набирайте в \verb'\begin{gather*}':
\begin{center}\begin{tabular}{|c|c|}
\hline
\begin{minipage}{8cm}
\begin{verbatim}
\begin{gather*}
\int x \d x = \frac{x^2}{2} + c \\
\int x^2 \d x = \frac{x^3}{3} + c
\end{gather*}
\end{verbatim}
\end{minipage}
&
\begin{minipage}{8cm}
\begin{gather*}
\int x \d x = \frac{x^2}{2} + c \\
\int x^2 \d x = \frac{x^3}{3} + c
\end{gather*}
\end{minipage} \\
\hline
\end{tabular}\end{center}
Перед ним не стоит разрывать параграф пустой строкой!
\item Математику в условии теоремы, утверждения пишем в тексте (\verb'$...$'), а вывод "--- уже выделенно (\verb'\begin{gather*}').
\item Если вы перечисляете в тексте какие-то математические записи, запятые, их разделяющие, пишутся через не в математическом режиме.
\begin{center}\begin{tabular}{|c|c|}
\hline
\begin{minipage}{8.6cm}
\begin{verbatim}
\begin{theorem}{Теорема Больцано
"--- Коши}
$f\colon [a, b] \ra \R$, $f$ 
"--- непрерывна на $[a, b]$. Тогда
$$\forall C\in [f(a), f(b)]\: 
\exists c \in (a, b)\colon f(c) = C$$
\end{theorem}
\end{verbatim}
\end{minipage} & \begin{minipage}{8.6cm}
\begin{theorem}{Теорема Больцано "--- Коши}
$f\colon [a, b] \ra \R$, $f$ "--- непрерывна на $[a, b]$. Тогда
$$\forall C\in [f(a), f(b)]\: \exists c \in (a, b)\colon f(c) = C$$
\end{theorem}\end{minipage}
\\
\hline
\end{tabular}\end{center}
\item Текст посреди математики всегда включаем в \verb'\text'. Иначе вы его даже не увидите, или увидите сильно не то!
\item Не допускаются сокращения, кроме <<и т. п.>> и <<<и т. д.>> (а эти выражения не рекомендуется использовать). Следует писать полностью (<<то есть>>, <<так как>>).
\item Не используйте псевдографику, не заменяйте одни символы другими, не забывайте, что одни и те же команды имеют разный
смысл в math-mode и в text-mode (и могут иметь разные пробелы, соответственно):
\begin{center}\begin{tabular}{|c|c|c|}
\hline \verb!A --> B! & A --> B \bad \\
\hline \verb!A $-->$ B! & A $-->$ B \bad \\
\hline \verb!A \rightarrow B! & A \rightarrow B \bad \\
\hline \verb!A $\rightarrow$ B! & A $\rightarrow$ B \ok \\
\hline \verb!A ==> B! & A ==> B \bad \\
\hline \verb!A $==>$ B! & A $==>$ B \bad \\
\hline \verb!A \Rightarrow B! & A \Rightarrow B \bad \\
\hline \verb!A $\Rightarrow$ B! & A $\Rightarrow$ B \ok \\
\hline \verb!$\Sigma_l^r x_n$! & $\Sigma_l^r x_n$ \bad \\
\hline \verb!$\sum_l^r x_n$! & $\sum_l^r x_n$ \ok \\
\hline \verb!$\sqcap_l^r x_n$! & $\sqcap_l^r x_n$ \bad \\
\hline \verb!$\prod x_n$! & $\prod_l^r x_n$ \ok \\
\hline
\end{tabular}\end{center}
\item Не используйте программистские обозначения в формулах, используйте математические (\verb!\cdot! для умножения чисел,
\verb!\times! для декартова произведения):
\begin{center}\begin{tabular}{|c|c|c|}
\hline \verb!$2*n$! & $2*n$ \bad \\
\hline \verb!$2n$! & $2n$ \ok \\
\hline \verb!$2\cdot n$! & $2\cdot n$ \ok \\
\hline \verb!$2\times n$! & $2\times n$ \bad \\
\hline \verb!$[a,b]\cdot [c,d]$! & $[a,b]\cdot [c,d]$ \bad \\
\hline \verb!$[a,b]\times [c,d]$! & $[a,b]\times [c,d]$ \ok \\
\hline
\end{tabular}\end{center}
\item Используйте окружения для единого стиля:
\begin{enumerate}
\item \verb'theorem' для тела теорем, у которых может быть название (не для доказательств!).
\item \verb'conseq' для следствий. Они сами нумеруются, не вписывайте все следствия в одно.
\item \verb'lemma' для тел лемм.
\item \verb'proof' для доказательств чего угодно.
\item \verb'assertion' для утверждений.
\item \verb'Def' для определений.
\item \verb'Rem' для замечаний.
\item \verb'exmp' для примеров.
\end{enumerate}
\item Вообще говоря, есть всякие сокращения для нужных символов (например, \verb'\Lra' для $\Lra$). Они помогают! Если вам нужно ещё "--- добавляйте в преамбулу на здоровье.
\item В продолжение предыдущего пункта "--- абсолютно все операторы математичского режима прямым шрифтом (вот так: $\inf$ (\verb'$\inf$'), а не так: $inf$ (\verb'$inf$')). Для этого в преамбулу можно добавлять \verb'\DeclareMathOperator', если вдруг такого оператора совсем нет (но скорее всего есть!).
\item Некоторые правила набора символов
\begin{itemize}
\item Черта в множествах "--- \verb'\mid', а не \verb'|'.
\begin{center}\begin{tabular}{|c|c|c|}
\hline Это... & ...верстается так & \\
\hline \verb'$\{n | 2 \div n\}$' & $\{n | nx > y\}$ \bad \\
\hline \verb'$\{n \mid 2 \div n\}$' & $\{n \mid nx > y\}$ \ok \\
\hline
\end{tabular}\end{center}
\item Двоеточие в определении отношения и функции "--- \verb'\colon', а не \verb':'.
\begin{center}\begin{tabular}{|c|c|c|}
\hline Это... & ...верстается так & \\
\hline \verb'$f: \R \ra \R$' & $f:\R\ra\R$ \bad \\
\hline \verb'$f\colon \R \ra \R$' & $f\colon\R\ra\R$ \ok \\
\hline
\end{tabular}\end{center}
\item Нестрогие равенства "--- \verb'\leqslant' и \verb'\geqslant', а не \verb'\leq' и \verb'\geq' (возможно, мы просто переопределим эти комманды, и тогда достаточно коротких комманд).
\begin{center}\begin{tabular}{|c|c|c|}
\hline Это... & ...верстается так & \\
\hline \verb'$a \le b$' & $a \le b$ \bad \\
\hline \verb'$a \leq b$' & $a \leq b$ \bad \\
\hline \verb'$a \leqslant b$' & $a \leqslant b$ \ok \\
\hline
\end{tabular}\end{center}
\item Черта над одной буквой "--- \verb'\bar', а не \verb'\overline'.
\begin{center}\begin{tabular}{|c|c|c|}
\hline Это... & ...верстается так & \\
\hline
\begin{minipage}{6cm}
\begin{verbatim}
$\overline{z_1z_2} = 
\overline{z_1}\overline{z_2}$
\end{verbatim}
\end{minipage}
& $\overline{z_1z_2} = \overline{z_1}\overline{z_2}$ \bad \\
\hline \verb'$\overline{z_1z_2} = \bar z_1 \bar z_2$' & $\overline{z_1z_2} = \bar z_1 \bar z_2$ \ok \\
\hline
\end{tabular}\end{center}
\item Треугольные скобки "--- \verb'\left< \right>' или хотя бы \verb'\langle \rangle', но никак не \verb'< >'!
\begin{center}\begin{tabular}{|c|c|c|}
\hline Это... & ...верстается так & \\
\hline \verb'$<G, \cdot>$' & $<G, \cdot>$ \bad \\
\hline \verb'\langle G, \cdot \rangle' & $\langle G, \cdot \rangle$ \ok \\
\hline \verb'\left<G, \cdot \right>' & $\left<G, \cdot \right>$ \ok \\
\hline \verb'$<a, \cfrac23>$' & $<a, \cfrac23>$ \bad \\
\hline \verb'\langle a, \cfrac23\rangle' & $\langle a, \cfrac23\rangle$ \bad \\
\hline \verb'\left<a, \cfrac23\right>' & $\left<a, \cfrac23\right>$ \ok \\
\hline
\end{tabular}\end{center}
\item Тире ставим так: \verb'"---', окружая пробелами (не минусиками "--- это дефисы!!!).
\begin{center}\begin{tabular}{|c|c|c|}
\hline Это... & ...верстается так & \\
\hline \verb'Слово - не воробей' & Слово - не воробей \bad \\
\hline \verb'Слово"---не воробей' & Слово"---не воробей \bad \\
\hline \verb'Слово "--- не воробей' & Слово "--- не воробей \ok \\
\hline
\end{tabular}\end{center}
\end{itemize}
\end{enumerate}
\end{document}
