\section{Многочлены Чебышева}
\begin{theorem}{}
$\exists$ многочлены $T_n(x), U_n(x)$ такие, что:
\begin{itemize}
\item $\cos{n \phi} = T_n(\cos \phi)$
\item $\frac{\sin{n \phi}}{sin \phi} = U_n(\cos \phi)$ при $\phi \ne 2 \pi k$
\item $T_0(x) = 1, T_1(x) = x, T_{n + 1}(x) = 2xT_n(x) - T_{n - 1}(x)$
\item $U_0(x) = 0, U_1(x) = 1, U_{n + 1}(x) = 2xU_n(x) - U_{n - 1}(x)$
\end{itemize}
\end{theorem}
\begin{Def} Многочлены $T_n$ и $U_n$ называются многочленами Чебышева первого и второго рода соответсвенно.
\end{Def}

\underline{Пример:} 

$T_0 = 1, T_1(x) = x, T_2(x) = 2x^2 - 1$ и $\cos 2\phi = 2 \cos^2 \phi - 1, T_3(x) = 4x^3 - 3x$ и $\cos 3\phi = 4 \cos^3 \phi - 3\cos \phi$ 

$U_0 = 0, U_1 = 1, U_2 = 2x$ и $\frac{\sin 2\phi}{\sin \phi} = 2\cos \phi$, $U_3(x) = 4x^2 - 1$ и $\sin 3\phi = (4\cos^2 \phi - 1)\sin \phi$

\begin{proof}

$U_n, T_n$ - многочлены с целыми коэффициентами по их заданию. 

$deg T_n = n, deg U_n = n - 1$(Индукция по n).  

$cos{n\phi} = T_n(\cos\phi)$

\underline{Индуция} по $n$:

\underline{База:}

$n = 0, n = 1$

\underline{Переход:}

$$z = \cos\phi + i \sin\phi$$
$$cos{(n + 1)\phi} = \frac{z^{n + 1} + z^{-(n + 1)}}{2} = (z + z^{-1})\frac{z^n + z^{-n}}{2} - \frac{z^{n - 1} + z^{-(n - 1)}}{2}=$$
$$= 2\cos\phi\cos{n\phi} - \cos{(n - 1)\phi} = 2 \cos \phi T_n(\cos \phi) - T_{n - 1}(\cos\phi) = T_{n + 1}(\cos \phi)$$

Доказательство для $U_n$ \underline{упражнение}.

$$\frac{\sin{n\phi}}{\sin\phi} = \frac{z^n - z^{-n}}{z - z^{-1}}$$
\end{proof}