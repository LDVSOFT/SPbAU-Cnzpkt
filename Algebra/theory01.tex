\section{Отображения. Композиция отображений.}

\begin{Def}
$A$,$B$ "--- множества. $\Gamma_{f} \subset A \times B$. $\Gamma_{f}$ "--- график отображения если выполнены два условия:
\begin{enumerate}
\item $\forall a \in A, \exists b \in B\colon (a, b) \in \Gamma_{f}$
\item $\forall a \in A, \exists b_1, b_2 \in B (a, b_1)\colon \in \Gamma_f \wedge (a, b_2) \in \Gamma_f \Ra b_1 = b_2$
\end{enumerate} 
\end{Def}

\begin{Def}
$A$, $B$, $\Gamma_f \subset A \times B$. Говорим, что задано отображение $f$ из $A$ в $B$ с графком $\Gamma_f$.
$$f:A \to B$$
$$A \xrightarrow{f} B$$
$$(a, b) \in \Gamma_f \Lra b = f(a)$$
$A$ "--- область определения, $B$ "--- область назначения.
\end{Def}

\begin{Def}
Равенство отношений: $f: A \to B$, $f_1: A_1 \to B_1$.
$$f = f_1 \Lra A = A_1 \land B = B_1 \land \Gamma_f = \Gamma_{f_1}$$
\end{Def}

\begin{Def}
Композиция отображений: $A \xrightarrow{f} B \xrightarrow{g} C$
$$g \circ f: A \to C$$
$$(g \circ f)(a) = g(f(a))$$
$$\Gamma_{g \circ f}\colon (a, c) \in \Gamma_{g \circ f} \Lra \exists b \in B (a, b) \in \Gamma_{f} \wedge (b, c) \in \Gamma_g$$
Область определения $g \circ f$ "--- область определения $f$ ($Dom(f)$). \\
Область назначения $g \circ f$ "--- область назначения $g$ ($coDom(g)$).
\end{Def}

\begin{theorem}{Композиция отображений ассоциативна}
$ A \xrightarrow{f} B \xrightarrow{g} C \xrightarrow{h} D $
$$h \circ (g \circ f) = (h \circ g) \circ f $$
\end{theorem}

\begin{proof}
Область определения 
$$Dom(h \circ (g \circ f)) = Dom(g \circ f) = Dom(f) = A$$
$$Dom((h \circ g) \circ f) = Dom(f) = A$$
Область назначений 
$$Dom(h \circ (g \circ f)) = coDom(h) = D$$
$$Dom((h \circ g) \circ f) = coDom((h \circ g)) = coDom(h) = D$$

$\forall a \in A\colon$
$$(h \circ (g \circ f))(a) = h(g \circ f(a)) = h(g(f(a)))$$
$$((h \circ g) \circ f)(a) = (h \circ g)(f(a)) = h(g(f(a)))$$     
\end{proof}
