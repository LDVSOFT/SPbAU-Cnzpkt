\section{Отображения. Композиция отображений}

\begin{Def}
$A$, $B$ "--- множества. $\Gamma_{f} \subset A \times B$. $\Gamma_f$ "--- график отображения, если выполнены два условия:

\begin{enumerate}
\item $\forall a \in A, \exists b \in B \colon (a, b) \in \Gamma_{f}$
\item $\forall a \in A, \exists b_1, b_2 \in B\colon (a, b_1) \in \Gamma_f \wedge (a, b_2) \in \Gamma_f \Ra b_1 = b_2$
\end{enumerate} 
\end{Def}

\begin{Def}
Есть $A, B$ и $\Gamma_f \subset A \times B$.
Говорим, что задано отображение $f$ из $A$ в $B$ с графиком $\Gamma_f$, обозначение:
\begin{gather*}
f:A \to B \\
A \xrightarrow{f} B \\
(a, b) \in \Gamma_f \Lra b = f(a)
\end{gather*}

Также называем:

$A$ "--- областью определения ($A=\Dom(f)$),

$B$ "--- областью назначения ($B=\coDom(f)$).
\end{Def}

\begin{Def}
Равенство отображений $f: A \to B$ и $f_1: A_1 \to B_1$:
\begin{gather*}
f = f_1 \Lra A = A_1, B = B_1, \Gamma_f = \Gamma_{f_1}
\end{gather*}
\end{Def}

\begin{Def}
Композиция отображений $g \circ f$, где $A \xrightarrow{f} B \xrightarrow{g} C$:
\begin{gather*}
g \circ f: A \to C \\
(g \circ f)(a) = g(f(a)) \\
(a, c) \in \Gamma_{g \circ f} \Lra \exists b \in B \colon (a, b) \in \Gamma_{f} \wedge (b, c) \in \Gamma_g
\end{gather*}

Область определения $g \circ f$ "--- область определения $f$ ($\Dom(f)$)

Область назначения $g \circ f$ "--- область назначения $g$ ($\coDom(f)$)

\end{Def}

\begin{theorem}{Композиция отображений ассоциативна}
Если $A \xrightarrow{f} B \xrightarrow{g} C \xrightarrow{h} D$, то:
\begin{gather*}
h \circ (g \circ f) = (h \circ g) \circ f \\
\end{gather*}
\end{theorem}

\begin{proof}
Проверим область определения:
\begin{gather*}
\Dom(h \circ (g \circ f)) = \Dom(g \circ f) = \Dom(f) = A \\
\Dom((h \circ g) \circ f) = \Dom(f) = A
\end{gather*}

Проверим область назначения:
\begin{gather*}
\Dom(h \circ (g \circ f)) = \coDom(h) = D \\
\Dom((h \circ g) \circ f) = \coDom((h \circ g)) = \coDom(h) = D
\end{gather*}

Проверим, что образ одинаков для любого $a \in A$:
\begin{gather*}
(h \circ (g \circ f))(a) = h(g \circ f(a)) = h(g(f(a))) \\
((h \circ g) \circ f)(a) = (h \circ g)(f(a)) = h(g(f(a)))
\end{gather*}

\end{proof}
