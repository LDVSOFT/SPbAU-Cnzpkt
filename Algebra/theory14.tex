\section{Циклические группы. Теорема о классификации циклических групп с точностью до изоморфности.}
\begin{Def}
	Группа G - циклическая группа, если она порождена одним элементом
	$G = <a>$
\end{Def}

Примеры:
\begin{enumerate}
	\item $(\mathbb{Z}, +) = <1>$
	\item ${e} = <e>$ - тривиальная группа
	\item $n \in \mathbb{N}$, $C_n$ - группа поворотов на цикл $\frac{2\pi k}{n}$, $C_n = <\frac{2\pi}{n}>$
\end{enumerate}

{\bf Степени:}
$$i > 0: a^i = \underbrace{aa \ldots a}_{i}$$
$$a^{-i} = \underbrace{a^{-1}a^{-1} \ldots a^{-1}}_{i}$$
$$a^{m + n} = a^ma^n$$
\begin{theorem}{Лемма о делении с остатком}
	$\forall n \in \mathbb{N}, a \in \mathbb{Z}, \exists$ и единственны $q, r \in \mathbb{Z},  0 \leqslant r < n: a = qn + r$
\end{theorem}
\begin{proof}
\\	Существование
	$$q = \lfloor \frac{a}{n} \rfloor$$
	$$nq \leqslant a < nq + n, r = a - nq$$
	Единственность
	$$a = nq_1 + r_1 = nq_2 + r_2, 0 \leqslant r1, r2 < n$$
	$$0 = n * (q_1 - q_2) + (r_1 - r_2), n|q_1 - q_2| = |r_1 - r_2|$$
	$n \geqslant 1$, левая часть хотя бы $n$, а правая часть строго меньше $n$. Противоречие.
\end{proof}

\begin{theorem}{Теорема о классификации циклических групп с точностью до изоморфности}
	Всякая циклическая группа изоморфна либо $(\mathbb{Z}, +)$, либо $C_n, n \in \mathbb{N}$
\end{theorem}
\begin{proof}
	$G = <a>$
	Рассмотрим отображение $i \ra a^i, i \in \mathbb{Z}$                              \\
	Если все элементы различны, то это изоморфизм $(i + j \ra a^{i+j} = a^i a^j)$ \\
	Тогда $\exists i > j: a^i = a^j \Ra a^{i-j}=e$	                                \\
	$n$ - наименьшее число такое, что $a^n = e$. Тогда $G = \{e = a^0, a^1, a^2, \dots, a^{n-1} \}, G \ra C_n, a^k a^l = a^{(k+l) mod n}$	
\end{proof}

\begin{Def}
	$G$ - группа, $G$ - конечна, тогда порядком $G$ называют число элементов в ней, иначе порядок равен $\infty$ \\
\end{Def}

\begin{Def}
	$a \in G, n \in \mathbb{N}$ - минимальное число такое, что $a^n = e$. Тогда $n$ - порядок $a$. Если такого элемента нет, то порядок $a$ равен $\infty$
\end{Def}
\begin{Rem}
	Альтернативное определение - порядок элемента равен порядку циклической группы, порожденной этим элементом
\end{Rem}