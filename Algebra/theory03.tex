\section{Тождественное отображение. Теорема о свойствах отображений, композиция которых есть
тождественное отображение}

\subsection{Тождественное отображение}

\begin{Def} Пусть есть $A$. Тогда тождественное отображение $id_{A}: A \to A$ задаётся как:

\[\forall a \in A\colon id_A(a) = a\]

$\Gamma_{id_A}$ есть диагональ $A \times A$, то есть $\Gamma_{id_a} = \{(a, a) \mid a \in A\}$
\end{Def}

\begin{theorem}{}
$f: A \to B$, тогда:

\[f \circ id_A = f =  id_B \circ f\]
\end{theorem}

\begin{proof}

Области определения и назначения совпадают.

Пусть $a \in A$. Проверим первое равенство:
\[(f \circ id_A)(a) = f(id_A(a)) = f(a)\]
и второе:
\[(id_B \circ f)(a) = id_B(f(a)) = f(a)\]

\subsection{Инъекция, сюръекция, биекция}
\begin{Def}
$f: A \to B$. Тогда $f$ "--- инъективное отображение (инъекция), если:

\begin{enumerate}
\item $\forall a_1, a_2 \in A, \exists b\colon (a_1, b) \in \Gamma_f \wedge (a_2, b) \in \Gamma_f \Ra a_1 = a_2$
\item $\forall a_1, a_2 \in A\colon f(a_1) = f(a_2) \Ra a_1 = a_2$
\end{enumerate}

Обозначается $f: A \rat B$.
\end{Def}

\begin{Def}
Отображение $f: A \to B$ назывется сюръективным (сюрекцией, или <<отображение \textit{на} $B$>>), если:
\[\forall b \in B, \exists a \in A\colon b = f(a)\]

Обозначние: $f: A \thra B$
\end{Def}

\begin{Def}
$f$ называется биективным (или биекцией), если $f$ и сюръективно, и инъективно.

Обозначение: $f: A \thrat B$.
\end{Def}

\begin{Def} 
Образ $C \subset A$: $f(C) = \{b \in B \mid \exists c \in C b = f(c)\}$.
\end{Def}

\begin{Def} 
Полный прообраз $D \subset B$: $f^{-1}(D) = \{a \in A \mid f(a) \in D\}$.
\end{Def}

$f(f^{-1}(D)) \subseteq D$, но может не совпадать.

$f$ инъективно $\iff$ прообраз любого одноэлементного множества содержит не более одного элемента.

$f : A \to B$ сюръективно $\iff$ $f(A) = B$.

\subsection{Теорема}
\begin{theorem}{}
$f:A \to B$, $g:B \to A$. Если $g \circ f = id_A$, то $f$ "--- инъективно, $g$ "--- сюръективно.
\end{theorem}

\begin{proof}
\begin{enumerate}
\item Пусть $a_1, a_2 \in A\colon f(a_1) = f(a_2)$. Тогда:
\begin{gather*}
g(f(a_1)) = g(f(a_2)); \\
(g \circ f)(a_1) = (g \circ f)(a_2); \\
id_A(a_1) = id_A(a_2); \\
a_1 = a_2; \\
\Ra f \text{ "--- инъекция}
\end{gather*}

\item Пусть $a \in A$ и $b=f(a)$, тогда:
\begin{gather*}
g(f(a)) = (g \circ f)(a) = id_A(a) = a; \\
a = g(b); \\
\Ra \forall a \in A, \exists b \in B\colon a = g(b) \Ra g \text{ "--- сюръекция}
\end{gather*}

\end{enumerate}
\end{proof}

\end{proof}