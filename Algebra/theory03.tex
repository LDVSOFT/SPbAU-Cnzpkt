\section{Тождественное отображение}

\begin{Def} Пусть есть $A$. Тогда тождественное отображение $id_{A}: A \to A$ задаётся как:

$\forall a \in A\colon id_A(a) = a$.

$\Gamma_{id_A}$ есть диагональ $A \times A$, то есть $\Gamma_{id_a} = \{(a, a) \mid a \in A\}$
\end{Def}

\begin{theorem}{}
$f: A \to B$, тогда:

$f \circ id_A = f =  id_B \circ f$
\end{theorem}

\begin{proof}

Области определения и назначения совпадают.

Пусть $a \in A$. Проверим первое равенство:
\[(f \circ id_A)(a) = f(id_A(a)) = f(a)\]
и второе:
\[(id_B \circ f)(a) = id_B(f(a)) = f(a)\]

\end{proof}