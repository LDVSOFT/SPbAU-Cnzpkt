\section{Обратимые отображения и их свойства}

\begin{Def}
Пусть $f: A \to B$. Тогда $f$ обратимо справа, если $\exists g: B \to A \colon$ \[f \circ g = id_B\]

$f$ обратимо слева, если $\exists g: B \to A \colon$ \[g \circ f = id_A\]

$f$ обратимо, если $\exists g: B \to A \colon$ \[g \circ f = id_A, f \circ g = id_B\]

Здесь $g$ "--- отображение, обратное к $f$, обозначается $g=f^{-1}$.
\end{Def}

\begin{theorem}{}

\begin{enumerate}
\item $f$ обратимо $\iff$ $f$ обратимо слева и справа.
\item $f$ обратимо $\Ra$ обратное отображение единственно.
\end{enumerate}

\end{theorem}

\begin{proof}
\begin{enumerate}
\item
Покажем, что если у $f$ есть и левый и правый обратный, то они совпадают.
Пусть $g$ "--- правый обратный к $f$, $h$ "--- левый, тогда:
\begin{gather*}
(h \circ f) \circ g = id_A \circ g = g; \\
h \circ (f \circ g) = h \circ id_B = h; \\
\Ra g = h\\
\end{gather*}

\item Пусть $f$ обратимо и $g$, $h$ "--- два обратных. В частности, $g$ "--- обратное справа, $h$ "--- обратное слева, то есть по п.1 они совпадают.
\end{enumerate}
\end{proof}

\begin{theorem}{}
Если $A \xrightarrow{f} B \xrightarrow{g} C$, то:

\begin{enumerate}
\item Если $f$, $g$ обратимы справа, то и $g \circ f$ обратимо справа.
\item Если $f$, $g$ обратимы слева, то и $g \circ f$ обратимо слева.
\item Если $f$, $g$ обратимы, то $g \circ f$ обратимо и $(g \circ f)^{-1} = f^{-1} \circ g^{-1}$
\end{enumerate}
\end{theorem}

\begin{proof}
\begin{enumerate}
\item Из условия знаем, что существуют:
\begin{gather*}
u: B \to A, f \circ u = id_B; \\
v: C \to B, g \circ v = id_C; \\
\end{gather*}

Покажем, что $u \circ v$ "--- правый обратный к $g \circ f$:
\begin{gather*}
(g \circ f) \circ (u \circ v) = g \circ (f \circ (u \circ v)) = \\
= g \circ ((f \circ u) \circ v) = g \circ (id_B \circ v) = g \circ v = id_C;
\end{gather*}

\item Аналогично.

\item 
\begin{gather*}
(g \circ f)(f^{-1} \circ g^{-1}) = g \circ ((f \circ f^{-1}) \circ g^{-1}) = g \circ (id_B \circ g^{-1}) = g \circ g^{-1} = id_C; \\
(f^{-1} \circ g^{-1})\circ(g \circ f) = f^{-1}(g^{-1} \circ g) \circ f = f^{-1} \circ id_B \circ f = f^{-1} \circ f = id_A;
\end{gather*}
\end{enumerate}
\end{proof}

\begin{conseq}{}
Композиция сюръективных "--- сюръекция.

Композиция инъективных "--- инъекцию.

Композиция биективных "--- биекция.
\end{conseq}

\begin{theorem}{}
Если $f: A \to B$ и f обратима, тогда $f^{-1}$ обратима и $(f^{-1})^{-1} = f$
\end{theorem}

\begin{proof}
\begin{gather*}
f \circ f^{-1} = id_{B}; \\
f^{-1} \circ f = id_A \Ra f \text{ "--- обратное к }f^{-1}
\end{gather*}

В силу единственности обратного $(f^{-1})^{-1} = f$.
\end{proof}
