\section{Обратимые отображения и их свойства}

$f: A \to B$

\begin{Def}
f ~--- обратное справа, если $\exists g: B \to A$

$f \circ g = id_B$

f ~--- обратим слева, если $\exists g: B \to A$

$g \circ f = id_A$

f обратимо, если $\exists g: B \to A$

$$g \circ f = id_A, f \circ g = id_B$$

g ~--- отображение, обратное к f.(обозначение $f^{-1}$)
\end{Def}

\begin{theorem}{}

\begin{enumerate}
\item f обратимо $\Lra$ f обратимо слава и справа.
\item f обратимо, то обратное отображение единственно.
\end{enumerate}

\end{theorem}

\begin{proof}
\begin{enumerate}
\item f обратимо $\Ra$ f обратимо слева и справа.

Если у f есть и левый и правый обратный, то они совпадают. 

g ~--- правый обратный к f, h ~--- левый.

$(h \circ f) \circ g = id_A \circ g = g$

$h \circ (f \circ g) = h \circ id_B = h$

$\Ra g = h$

\item Пусть f обратимое и g и h ~--- два обратных. В частности g ~--- обратное справа, h ~--- обратное слева.
\end{enumerate}
\end{proof}