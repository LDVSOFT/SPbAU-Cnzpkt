\section{} % 45
Матрица на кольце заданного размера, сложение, транспонирование, умножение (из линейных преобразований),
ассоциативность и дистрибутивность умножения, $(AB)^{T}=B^{T}A^{T}$ (если носитель коммутативен),
умножение на скаляр (ассоциативность и дистрибутивность разных видов + на единицу). Кольцо квадратных
матриц с единицей, для чётных $n$ в нём есть делители нуля.

\section{} % 46
$\left(\begin{matrix}a & -b \\ b & a\end{matrix}\right)$, доказали, что поле, отобразили (сопряжение "--- транспонирование).

\section{} % 47
Построили как $\left(\begin{matrix}z & w \\ -\bar w & \bar z\end{matrix}\right)$ при $z, w \in \mathbb{C}$. Получили тело (не поле).

\section{} % 48
Ввели три мнимые единицы, $i^2=j^2=k^2=ijk=-1$ ($ij=k$ и по циклу). Проверили, что умножается так же, как в матрицах.

\section{} % 49
Вещественная "--- при единице, мнимая "--- при мнимых. Сопряжение "--- поменяли три знака ($\alpha+\bar\alpha \in \R$, $\alpha\bar \alpha=a^2+b^2+c^2+d^2$,
$\overline{\alpha \bar \alpha} = \bar \alpha \alpha$, коммутативности нет),
модуль.

\section{} % 50
Мультипликативность: перешли к квадрату модуля, $|ab|^2 = (ab)\overline{(ab)} = ab\bar b \bar a = a(b \bar b)\bar a = a\bar a(b \bar b) = (a \bar b)(b \bar b) = |a|^2|b|^2$
(если $a\in \R$, то оно коммутирует с любым $b$). Написали тождество Эйлера, представим произведение двух сумм четырёх квадратов как сумму четырёх
квадратов.
