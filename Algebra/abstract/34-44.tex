\section{} % 34
Ввели операции без символа $i$, проверили.

\section{} % 35
Инъективный гомоморфизм $\R \to \mathbb{C}$, ввели $i$ и проверили, что $i^2=-1$, записали пару как $a+bi$,
ввели $\Re z$ и $\Im z$. Сопряжённое: алгебра и геометрия. Свойства: $\bar{\bar z} = z$, $\bar z=z \iff z \in \R$,
$\overline{z_1+\dots+z_n} = \bar z_1+\dots+\bar z_n$ (и для $\cdot$), $\overline{f(z)} = f(\bar z)$, $z\bar z \in \R$, для $z\bar z$: равенство нулю и неотрицательность.
Сопряжение "--- автоморфизм (нетождественный изоморфизм на себя).

\section{} % 36
Модуль: $|z|=\sqrt{z\bar z}=\sqrt{a^2+b^2}$, квадрат модуля мультипликативен (отсюда формула для произведения двух сумм двух квадратов; аналогичная
только для $\R$, $\mathbb{H}$ и октав).

\section{} % 37
Перешли в полярную систему (аргумент "--- класс эквивалентности), главный аргумент из $[0, 2\pi)$, но можно выбирать любой
(показали для пар, что всегда можно выбрать).
Тригонометрическая: $z=a+bi=r(\cos \varphi + i\sin\varphi)$, причём $r$ "--- модуль. Проверили умножение и деление.

\section{} % 38
$||z_1|-|z_2|| \le |z_1 \pm z_2| \le |z_1| + |z_2|$, причём для четырёх возможных равенств можно
нарисовать картинку. Д-во: возвели в квадрат (т.к. $\ge 0$), раскрыли в тригонометрической форме ($\cos\varphi_1\cos\varphi_2+\sin\varphi_1\sin\varphi_2=\cos(\varphi_1-\varphi_2$).

\section{} % 39
Для $n \in \Z$ написали $z^n=r^n(\cos n\varphi + i\sin n\varphi)$ (проверили для отрицательных!).

\section{} % 40
Пусть $w^n = z$, тогда модуль фиксирован, расписали возможные аргументы, всего получили не более $n$ корней, а на самом деле даже ровно $n$.
Можно еще многочлен для большей уверенности расмотреть.

\section{} % 41
$\eps$ "--- корень из единицы, если $\eps = \cos\frac{2\pi k}{n} + i\sin\frac{2\pi k}{n}$, где $k=0..(n-1)$. Первообразный, если
не является корнем меньшей степени. Лемма: $\iff (n,k)=1$. Следствие: число первообразных равно $\varphi(n)$. Геометрический смысл
(где корни и как вращают). Упражнения: корни образуют циклическую группу по умножению и бегают по кругу, $S=\{|z|=1\}$ тоже группа по умножению, в группе $G=\cup G_n$
есть счётная система образующих, но нет конечной.

\section{} % 42
\TODO

\section{} % 43
\TODO

\section{} % 44
\TODO
