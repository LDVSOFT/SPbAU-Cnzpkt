\section{Корни из 1. Первообразные корни из 1}

\begin{Def}
$\epsilon = \cos{\frac{2 \pi k}{n}} + i \sin{\frac{2 \pi k}{n}}, k = 0, \dots, n - 1$ \\
\hspace*{1cm} $\epsilon$ - корень из 1 степени n. Если $k = 0$, $\epsilon = 1$ \\
\hspace*{1cm} $\epsilon$ - первообразный корень степени n, если $\epsilon$ не является корнем из 1 меньшей степени.
\end{Def}
\begin{assertion}
$\epsilon$ - первообразный корень степени $n \Lra (n, k) = 1$ 
\end{assertion}
\begin{proof}
	Если дробь $k/n$ сократима, то $k/n = k'/n', n' < n \Ra \epsilon$ - корень степени $n'$: 
	$$\epsilon = \cos{\frac{2 \pi k}{n'}} + i \sin{\frac{2 \pi k}{n'}}$$
\end{proof}
\begin{conseq} \\
\hspace*{1cm}Число первообразных корней степени $n$ из 1 равно функции Эйлера: $\phi(n) = \{k | 1 \le k \le n, (k, n) = 1\}$
\end{conseq}
$$
	\begin{tabular}{|c|c|c|c|}
	\hline
	$n$ & все корни & первообразные корни & $\phi(n)$ \\
	\hline
	1 & 1 & 1 & 1 \\
	\hline
	2 & 1,-1 & -1 & 1 \\
	\hline
	3 & 1,$-\frac{1}{2} \pm \frac{\sqrt{3}i}{2}$ & $-\frac{1}{2} \pm \frac{\sqrt{3}i}{2}$ & 2 \\
	\hline
	4 & $\pm1,\pm i$ & $\pm i$ & 2 \\
	\hline
	6 & $\pm1,\pm\frac{1}{2} \pm \frac{\sqrt{3}i}{2}$ & $\frac{1}{2} \pm \frac{\sqrt{3}i}{2}$ & 2 \\
	\hline
	8 & \multicolumn{3}{c|}{Упражнение} \\
	\hline
	12 & \multicolumn{3}{c|}{Упражнение} \\
	\hline
	\end{tabular}
$$	

Для $n = 6$:
$$x^6 -1 = (x^2)^3 - 1 = (x^2 - 1)(x^4 + x^2 + 1) = (x^2 + x + 1)(x^2 - x + 1)(x - 1)(x + 1)$$

\underline{Упражнение}:
\begin{enumerate}
\item $n > 2 \Ra \phi(n)$ - чётно.
\item $n = Const, G_n = \{\epsilon | \epsilon$ - корень из 1 степени $n\}$. Тогда $|G_n| = n \Ra G_n$ - группа по умножению и $G_n \cong C_n$.
\item $S = \{z \in \C | |z| = 1\}$ Доказать, что $S$ - группа относительно умножения.
\item $G = \bigcup\limits_{n = 1}^{\inf} G_n$ Доказать, что $G$ - группа. Доказать, что в $G$ есть счётная система образующих, но нет конечной системы образующих. Доказать, что каждый элемент $G$ имеет конечный порядок. 
\end{enumerate}

\underline{Геометрический смысл:}
Рисунок3. 
Корни из 1 лежат в вершинах правильного $n$-угольника, вписанного в круг радиуса 1.

$\epsilon \in G_n, f: z \mapsto \epsilon z$, тогда $f$ - это поворот на угол $arg(\epsilon) = {2 \pi k}/n$