\section{Корни из 1. Первообразные корни из 1.}

\begin{Def}
$\epsilon$ "--- корень из $1$ степени $n$, если
\begin{gather*}
\epsilon = \cos{\frac{2 \pi k}{n}} + i \sin{\frac{2 \pi k}{n}}, k = 0..(n - 1)
\end{gather*}
\end{Def}

\begin{Def}
$\epsilon$ "--- первообразный корень единицы степени $n$, если $\epsilon$ не является корнем меньшей степени.
\end{Def}

\begin{assertion}
$\epsilon$ "--- первообразный корень степени $n$ тогда и только тогда, когда $(n, k) = 1$.
\end{assertion}

\begin{proof}
Если дробь $\frac{k}{n}$ сократима, то 
\begin{gather*}
\frac{k}{n} = \frac{k'}{n'}; n' < n
\end{gather*}
Но тогда $\epsilon$ - корень степени $n'$: 
\begin{gather*}
\epsilon = \cos\frac{2\pi k'}{n'} + i \sin\frac{2\pi k'}{n'}
\end{gather*}
\end{proof}

\begin{conseq}
Число первообразных корней степени $n$ из $1$ равно функции Эйлера
\begin{gather*}
\phi(n) = \left|\left\{k \mid 1 \le k \le n, (k, n) = 1\right\}\right|
\end{gather*}
\end{conseq}

\begin{center}
\begin{tabular}{|c|c|c|c|}
\hline
$n$ & все корни & первообразные корни & $\phi(n)$ \\
\hline
$1$ & $1$ & $1$ & $1$ \\
\hline
$2$ & $1$, $-1$ & $-1$ & $1$ \\
\hline
$3$ & $1$,$-\frac{1}{2} \pm \frac{\sqrt{3}i}{2}$ & $-\frac{1}{2} \pm \frac{\sqrt{3}i}{2}$ & $2$ \\
\hline
$4$ & $\pm1,\pm i$ & $\pm i$ & $2$ \\
\hline
$6$ & $\pm1,\pm\frac{1}{2} \pm \frac{\sqrt{3}i}{2}$ & $\frac{1}{2} \pm \frac{\sqrt{3}i}{2}$ & $2$ \\
\hline
$8$ & \multicolumn{3}{c|}{Упражнение} \\
\hline
$12$ & \multicolumn{3}{c|}{Упражнение} \\
\hline
\end{tabular}
\end{center}

Для $n = 6$:
$$x^6 -1 = (x^2)^3 - 1 = (x^2 - 1)(x^4 + x^2 + 1) = (x^2 + x + 1)(x^2 - x + 1)(x - 1)(x + 1)$$

\underline{Упражнения}:
\begin{enumerate}
\item $n > 2 \Ra \phi(n)$ "--- чётно.
\item $n = const$, $G_n = \left|\left\{\epsilon \mid \text{$\epsilon$ "--- корень из $1$ степени $n$}\right\}\right|$. Тогда $|G_n| = n \Ra G_n$ "--- группа по умножению и $G_n \cong C_n$.
\item $S = \{z \in \C \mid |z| = 1\}$. Доказать, что $S$ "--- группа относительно умножения.
\item $G = \bigcup\limits_{n = 1}^{\infty} G_n$ Доказать, что $G$ - группа. Доказать, что в $G$ есть счётная система образующих, но нет конечной системы образующих. Доказать, что каждый элемент $G$ имеет конечный порядок. 
\end{enumerate}

\underline{Геометрический смысл:}
\begin{center}
\def\svgwidth{6.0cm}
\input{theory41-root.pdf_tex}
\end{center}
Корни из 1 лежат в вершинах правильного $n$"---угольника, вписанного в круг радиуса $1$.

$\epsilon \in G_n$, $f\colon z \mapsto \epsilon z$, тогда $f$ "--- это поворот на угол $\arg \epsilon = \frac{2 \pi k}n$