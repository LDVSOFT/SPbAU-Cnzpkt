\section{Группы. Простейшие следствия из аксиом группы}
\begin{Def}
	Бинарная операция на $A$ -- отображение $f : A \times A \ra A$
\end{Def}
$G \neq \emptyset, \cdot : G \times G \ra G$

$<G, \cdot > $ - группа, если выполняются следующие свойства:
\begin{enumerate}
	\item Ассоциативность: $(a \cdot b) \cdot c = a \cdot (b \cdot c)$		
	\item $\exists$ нейтральный елемент $e : a \cdot e = e \cdot a = a$
	\item $\forall a \: \exists a^{-1} : a \cdot a^{-1} = a^{-1} \cdot a = e$		
	\item $a \cdot b = b \cdot a$ - если это свойство выполняется, то группа Абелева(коммутативная) 
\end{enumerate}

В дальнейшем под записью $ab$ будет пониматься $a \cdot b$, если, разумеется, не сказано другого.     \\

{\bf Пример:} 
$X$ ~--- Множество.

$S(X)$ ~--- множество биекций из X в X.

$f \in S(x)$

$e = id_x$

$f^{-1} =f^{-1}$

$X = \{1, \ldots, n\}$

$S_n = S(\{1, \ldots n\})$ ~--- симметрическая группа(степени n)
	
\begin{theorem}{Простейшие свойства групп}
	\begin{enumerate}
		\item Единственность нейтрального
		\item Единственность обратного
		\item 
			Уравнения вида $ax=b$, $ya=b$ имеют решение, причем единственное
		\item $(a_1 a_2 \dots a_n)^{-1} = a_n^{-1} a_{n-1}^{-1} \dots a_1^{-1}$
   	\end{enumerate}
\end{theorem}
\begin{proof}
	\begin{enumerate}
		\item
			Пусть существуют два нейтральных элемента $e_1, e_2$
			$$e_2 = e_1 e_2 = e_1 \Ra e_2 = e_1$$
		\item
			$a', a''$ - обратные к $a$
			$$a'aa'' = (a'a)a''= ea'' = a''$$
			$$a'aa'' = a'(aa'') = a'e = a'$$
			$$a' = a''$$
		\item
			$ax = b, b^{-1}ax = e \Ra x = (b^{-1}a)^{-1}$.
			Так мы доказали одним махом и существование и единственность(так как обратный элемент существует и единственен)
		\item
			$(a_1 \dots a_n) (a_1^{-1} \dots a_n^{-1}) = (a_1 \dots a_{n-1})(a_n a_n^{-1})(a_{n-1}^{-1} \dots a_1^{-1})$.
			Центральная скобка это $e$. Таким образом избавляемся от всех переменных, получаем, что правая скобка обратна левой. 
	\end{enumerate}
\end{proof}
	
