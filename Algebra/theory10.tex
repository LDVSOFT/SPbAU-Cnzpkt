\section{Группы. Простейшие следствия из аксиом группы}
\begin{Def}
	Бинарная операция на $A$ "--- отображение $f : A \times A \ra A$
\end{Def}

\begin{Def}
Пусть $G \neq \emptyset$, $\cdot : G \times G \ra G$.
Тогда $\left<G, \cdot \right>$ "--- группа, если выполняются следующие свойства:
\begin{enumerate}
	\item Ассоциативность: $(a \cdot b) \cdot c = a \cdot (b \cdot c)$ (для любых троек)
	\item $\exists$ нейтральный елемент $e$ такой, что $\forall a \in G \colon a \cdot e = e \cdot a = a$ 
	\item $\forall a \in G, \exists a^{-1} \colon a \cdot a^{-1} = a^{-1} \cdot a = e$		
	\item $\forall a, b \in G \colon a \cdot b = b \cdot a$ "--- если это свойство выполняется, то группа Абелева (коммутативная).
\end{enumerate}
\end{Def}

В дальнейшем под записью $ab$ будет пониматься $a \cdot b$, если, разумеется, не сказано другого.

\begin{exmp}
Пусть $X$ "--- множество. $S(X)$ "--- множество биекций из X в X.
Если взять операцию <<композиция>>, получится группа. Ассоциативность композиции знаем,
нейтральный элемент "--- $id_X$, обратимость биекции тоже знаем. Коммутативность отсутствует.
\end{exmp}

\begin{exmp}
Пусть $X = \{1, \ldots, n\}$. $S_n = S(\{1, \ldots n\})$ "--- симметрическая группа степени $n$,
группа перестановок чисел от 1 до $n$.
\end{exmp}
	
\begin{theorem}{Простейшие свойства групп}
	\begin{enumerate}
		\item Единственность нейтрального
		\item Единственность обратного
		\item 
			Уравнения вида $ax=b$, $ya=b$ имеют решение, причем единственное
		\item $(a_1 a_2 \dots a_n)^{-1} = a_n^{-1} a_{n-1}^{-1} \dots a_1^{-1}$
   	\end{enumerate}
\end{theorem}
\begin{proof}
	\begin{enumerate}
		\item
			Пусть существуют два нейтральных элемента $e_1, e_2$:
			\[e_2 = e_1 e_2 = e_1 \Ra e_2 = e_1\]
		\item
			Пусть $a', a''$ "--- обратные к $a$, тогда:
			\begin{gather*}
			a'aa'' = (a'a)a''= ea'' = a''; \\
			a'aa'' = a'(aa'') = a'e = a'; \\
			a' = a'';
			\end{gather*}
		\item
			$ax = b \iff b^{-1}ax = e \Ra x = (b^{-1}a)^{-1}$.
			Так мы доказали одним махом и существование, и единственность (так как обратный элемент существует и единственен).
		\item
			$(a_1 \dots a_n) (a_1^{-1} \dots a_n^{-1}) = (a_1 \dots a_{n-1})(a_n a_n^{-1})(a_{n-1}^{-1} \dots a_1^{-1})$.
			Центральная скобка равна $e$. Таким образом избавляемся от всех переменных, получаем, что правая скобка обратна левой. 
	\end{enumerate}
\end{proof}
	
