\section{Алгебраические замкнутые поля}
\begin{Def}
Поле А ~--- алгебраически замкнуто, если любой $f \in A[x] \backslash A$ имеет в А хотя бы 1 корень.
\end{Def}

\begin{theorem}{}
Следующие условия равносильны. 
\begin{enumerate}
\item A ~--- алгебраически замкнуто. 
\item $\forall f \in A[x]$ c $\deg f \ge 1$ делится на линейный многочлен. 
\item $\forall f \in A[x]$ c $\deg f \ge 1$ имеет $deg f$ корней (с учетом кратности).
\item $\forall f \in A[x]$ c $deg f \ge 1$ полностью раскладывается на линейные множества в колце многочленов.
\end{enumerate}
\end{theorem}

\begin{proof}
$1 \Lra 2$(следствие теоремы Безу)

$3 \Ra 1$ очевидно.

$1 \Ra 3$ Индукция и $deg f$

\begin{enumerate}
\item {\bf База:} $deg f = 1$ 

$ax = b$

$x = \frac{b}{a}$ ~--- корень.
\item {\bf Переход:} $f deg f \ge 2$

$\exists c \in A$ корень f кратности $k \ge 1, f = (x - c)^{k}g$

По индукционному предположению число корней g = deg g.

Все корни f отличные от с это в точности корни g, причем той же кратности. 

Число корней $f = k + $ число корней g = k + deg g = deg f.

$4 \Ra 2$ очевидно.

$2 \Ra 4$ индукция по deg f.


\end{enumerate}

\end{proof}