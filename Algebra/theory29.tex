\section{Число корней многочлена}
\begin{lemma}
A "--- область целостности. $0 \ne f, g \in A[x]$

c "--- корень f кратности k, корень g кратности l $\Ra$

c "--- корень fg кратности k + l
\end{lemma}

\begin{proof}
$$f = (x - c)^kf_1, f_1(c) \ne 0$$
$$g = (x - c)^lg_1, g_1(c) \ne 0$$
$$fg = (x - c)^{k + l}f_1g_1$$
$$f_1(c)g_1(c) \ne 0$$
$\Ra$ c "--- корень fg кратности k + l.
\end{proof}

\begin{lemma}
A "--- область целостности. 
Какие бы ни были $c \ne d \in A$, $0 \ne f, g \in A[x], a, k \in \N$, такие, что $f = (x - c)^{k}g, g(c) \ne 0$, то
$(x - d)^{a}|f \Lra (x - d)^{a}|g$
\end{lemma}

\begin{proof}
$\La$
$$ (x - d)^{a}|g \Ra (x - d)^{a}|f$$
$\Ra$
Индукция по а.
{\bf База:}
$$a = 1$$
$$x - d|f \Ra f(d) = 0$$
$$(c - d)^kg(d) = 0 \Ra g(d) = 0$$
$$\Ra (x - d)|g$$
{\bf Переход} $a - 1 \to a$
$a - 1$ для всех f и g удовлетворяет условию леммы
$$f = (x - c)^{k}g$$
$$(x - d)^{a}|f \Ra (x - d)^{a - 1}|f$$
$(x - d)^{a - 1}|d$ по индукционномупредположению.
$$f = (x - d)^{a}f_1$$
$$g = (x - d)^{a - 1}g_1$$
$$(x - d)^af_1 = (x - c)^{k}(x - d)^{a - 1}g_1$$
$$(x - d)f_1 = (x - c)^kg_1$$
$$\Ra x - d|g_1$$
(по доказанному при a = 1)
$$(x - d)^{a}|g$$  
\end{proof}

\begin{theorem}{}
 A "--- область целостности. $0 \ne f \in A[x]$
 $\Ra$ число корней f с учетом кратности не превосходит $deg f$
\end{theorem}
\begin{proof}
Индукция по $deg f$
\begin{enumerate}
\item {\bf База:} $deg f = 0, f = const \ne 0$
нет корней.
\item {\bf Переход:} f с "--- корень f кратности k.
$f = (x - c)^{k}g, g(c) \ne 0$ c "--- не корень g. 

Все корни g "--- это в точности все корни f(кроме с), причем кратность сохраняется. 

Число корней g(с учетом кратности) $\le deg g$

число корней $f = k + $число корней $g \le k + deg g = deg f$
\end{enumerate}

\end{proof}

\begin{Rem}
Предположение, что $A$ "--- область целостности существенно.  
\end{Rem}

\begin{Def}
$$A, f \in A[x]$$
$$\tilde f: A \to A$$
$$c \to f(c)$$
$$f, g \tilde f = \tilde g$$
\end{Def}
{\bf Примеры:}
$$A = \F_2$$
$$f = 0, g = x^2 + x$$
$$ \tilde f: 0 \to 0, 1 \to 0$$
$$ \tilde g: 0 \to 0, 1 \to 0$$

\begin{conseq}
A "--- область целостности.
$$f, g \in A[x], |A| > \max(deg f, deg g)$$
Тогда, если $\tilde f = \tilde g$, то $f = g$.
\end{conseq}
\begin{proof}
f - g

$$\tilde{f - g} = \tilde f - \tilde g \text{"--- тождественно не нулевое отображение}$$
$$\forall c \in A, f(c) - g(c) = 0$$
Число корней $f - g > deg(f - g) \Ra f - g = 0$
\end{proof}

\begin{conseq}
Если A "--- область целостности.

$|A| = \infty$ и $\tilde f = \tilde g$, то и $f = g$
\end{conseq}