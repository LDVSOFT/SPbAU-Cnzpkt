\section{Характеристика кольца}
$A$ - кольцо с $1$\\

\begin{Def}
	Характеристика кольца - наименьшее $n > 0$, т.ч. $\underbrace{1+...+1}_{n} = 0$
		$$ \mathrm{char} A = n $$
	Если такого $n$ нет, тосчитается, что $\mathrm{char} A = 0$\\
\end{Def}

\textbf{Примеры:}
	$$ \mathrm{char} \Z = 0, \mathrm{char} \Q = 0, \mathrm{char} \R = 0 $$
	$ \mathbb{F}_2, \mathbb{F}_3$ - поля из 2-х и 3-х элементов соответсвенно\\
	$$ \mathrm{char} \mathbb{F}_2 = 2, \mathrm{char} \mathbb{F}_3 = 3 $$

\begin{Rem} 
	$A$ - поле $\Ra$ $\mathrm{char} A$ либо $0$, либо простое число
\end{Rem}

\begin{proof}
	\begin{enumerate}
	\item $\forall n \underbrace{1 + ... + 1}_{n} \neq 0 \Ra \mathrm{char} A = 0$
	\item $\mathrm{char} A > 0 \underbrace{1 + ... + 1}_{n} = 0, n > 1$ т.к. в поле $1 \neq 0$\\
	$n = ab, 1 < a, b < n$ \\
	по дистрибутивности $\Ra \underbrace{1+...+1}_a = 0 \vee \underbrace{1+...+1}_b = 0 \Ra$
	наименьшее $n$, чтобы $\underbrace{1 + ... + 1}_{n} = 0$ должно быть простым
	\end{enumerate}
\end{proof}
