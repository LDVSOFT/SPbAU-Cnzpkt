\section{Теорема о деление с остатком}
\begin{theorem}{}
	$A$ - коммутативное кольцо с $1$
	$f, g \in A[x]$\\
	$f = a_0 + a_1x + a_2x^2 + ... + a_nx^n, n = \deg f, a_n \in A^*$\\
	тогда $\exists q, r \in A[x] :  g = qf + r, \deg r < \deg f$\\
\end{theorem}

\begin{Rem}
	Если $A$ - область целостности, то такое представление единственно\\
\end{Rem}

\begin{proof}
	\textbf{Существоване:}\\
	Индукция по $m = \deg g$:\\
	\textbf{База:} $m < n$\\
	$$ q = 0, r = g $$
	\textbf{Переход:} доказали для всех многочленов $\deg g < m$, докажем для $m$
	$$ g = b_mx^m + ... + b_0 $$
	$$ g_1 = g - b_m a_n^{-1}x^{m-n}f $$
	коэффицент при $x^m$ в $g_1$: $b_m - b_m a_n^{-1} a_n = 0 \Ra \deg g_1 < m$\\
	по предположению индукции $g_1 = fq_1 + r_1, \deg r_1	< \deg f $, тогда:
		$$ r = r_1 $$
		$$ q = q_1 + b_m a_n^{-1} x^{m-n} $$
		$$ g = fq	+ r $$
		
	\textbf{Единственность:}\\
	$A$ - область целостности\\
	$$ g = fq + r = f \tilde{q} + \tilde{r}, \deg r, \deg \tilde{r} < f$$
	$$ f (q - \tilde{q}) = \tilde{r} - r $$
	если $q - \tilde{q} \neq 0$, то степень левого многочлена $\geq \deg f$ и степень правого $< deg f$ $\Ra$ 
	$q -\tilde{q} = 0, r - \tilde{r} = 0 \Ra q = \tilde{q}, r = \tilde{r}$
\end{proof}

\begin{Rem}
	условие обратимости старших коэффицентов существенно:
	$A = \Z$\\
	$f = 2x, g = x^2 + 1$\\
	в $\Z[x]$ разложения: $g = fq + r, q, r \in \Z[x], deg r < deg f$ - не существует
\end{Rem}
