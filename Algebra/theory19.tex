\section{Четность перестановки. Теорема об изменении четности перестановки при умножении на транспозицию. Следствия из неё.}

\begin{Def}
$\sigma \in S_n; i, j \in \{1, \dotsc, n\}. i, j $ образуют инферсию относительно $\sigma$, если $i < j$, а $\sigma(i) > \sigma(j)$.
\end{Def}
\begin{Def}
$INV(\sigma)$ -- множество всех инверсий относительно $\sigma$.
\end{Def}
\begin{Def}
$\sigma $-- четная, если $|Inv(\sigma)|$ четно.
\end{Def}
\begin{Def}
$\sigma $-- нечетная, если $|Inv(\sigma)|$ нечетно.
\end{Def}

\begin{theorem}{$\sigma \in S_n, \tau = (i j)$ ~--- транспозиция. Тогда $\sigma$ и $\sigma\tau$ имеют различную четность.}
\begin{proof}
Не умаляя общности, будем считать, что $i < j$. Тогда\\
$\sigma = (\sigma(1) \dotsc \sigma(i) \dotsc \sigma(j) \dotsc \sigma(n))$\\
$\sigma\tau = (\sigma(1) \dotsc \sigma(j) \dotsc \sigma(i) \dotsc \sigma(n))$\\

$$
	\begin{tabular}{|c|c|c|c|}
	\hline
	\multicolumn{2}{|c|}{$\sigma$} & \multicolumn{2}{c|}{$\sigma\tau$}\\
	\hline
	$(k l), \{k, l\} \bigcap \{i, j\} = \varnothing$ & есть инверсия & $(k l), \{k, l\} \bigcap \{i, j\} = \varnothing$ & есть инверсия\\
	\hline
	$(k l), \{k, l\} \bigcap \{i, j\} = \varnothing$ & есть инверсия & $(k l), \{k, l\} \bigcap \{i, j\} = \varnothing$ & есть инверсия\\
	\hline
	$(k i), k < i$ & есть & $(k j), k < i$ & есть\\
	\hline
	$(k i), k < i$ & нет & $(k j), k < i$ & нет\\
	\hline
	$(k i), k > j$ & есть & $(k j), k > j$ & есть\\

	\hline
	$(k j), k < i$ & есть & $(k i), k < i$ & есть\\
	\hline
	$(k j), k < i$ & нет & $(k i), k < i$ & нет\\
	\hline
	$(k j), k > j$ & есть & $(k i), k > j$ & есть\\
	\hline
	$(k j), k > j$ & нет & $(k i), k > j$ & нет\\
	
	\hline
	\multicolumn{4}{|c|}{$\sigma(\sigma(i)\sigma(k)\sigma(j))$}\\
	\hline
	\multicolumn{4}{|c|}{$\sigma\tau(\sigma(j)\sigma(k)\sigma(i))$}\\
	\hline
	$(k i), i < k < j$ & есть & $(k j), i < k < j$ & нет\\
	\hline
	$(k i), i < k < j$ & нет & $(k j), i < k < j$ & есть\\
	\hline
	$(k j), i < k < j$ & есть & $(k i), i < k < j$ & нет\\
	\hline
	$(k j), i < k < j$ & нет & $(k i), i < k < j$ & есть\\

	\hline
	(i j) & есть & (i j) & нет\\
	\hline
	(i j) & нет & (i j) & есть\\
	\hline

	\end{tabular}
$$	
Как глубокоуважаемый читатель уже догадался, самое интересное здесь - это последние 6 стрк.\\
$r = \{k | i < k < j \wedge (i k)$ образует инверсию $\}$\\
$s = \{k | i < k < j \wedge (k j)$ образует инверсию $\}$\\
$|Inv(\sigma\tau)| = |Inv(\sigma)| - r + (j - i - 1 - r) - s + (j - i - 1 - s) \pm 1 \Rightarrow$ четность изменилась.\\
\end{proof}

\begin{theorem}{Следствия}
\end{theorem}
\begin{enumerate}
\item $\sigma = \sqcap_{j = 1}^r \tau_j, \tau_j$ -- транспозиция.\\
$\sigma$ четна(нечетна) $\Leftrightarrow$ число сомножителей четно(нечетно, соответственно).
\begin{proof}\\
$"\Leftarrow:" id$ четна. При каждом добавлении $\tau_j$ четность меняется.\\
$"\Rightarrow": \sigma$ четна, $r$ -- число транспозиций, и если у него другая четность, то приходим к противоречию.\\
\end{proof}
\item $\sigma \in S_n, \tau  $-- транспозиция. $\sigma$ и $\tau\sigma$ имеют различную четность. (из первого следствия)
\item При перемножении двух перестановок их четность меняется так же, как при суммировании их четностей как чисел.
\item Множество четных перестановок -- подгруппа $S_n$(знакопеременная группа). Обозначается $A_n$.
\begin{proof}
\begin{enumerate}
\item Непусто($id$ четна)
\item Замкнуто(по третьему следствию)
\item Обратный элемент к $\sigma = \sqcap_{j = 1}^{2r} \tau_j$ -- это $\sigma^{-1} = \sqcap_{j = 1}^{2r} \tau_{2r + 1 - j}$ 
\end{enumerate}
\end{proof}
\end{enumerate}
\end{theorem}