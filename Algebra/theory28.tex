\section{Кратные корни}
A "--- поле. 
$f \in A[x],f \ne 0$
c "--- корень f в A $\Lra (x - c)|f$ в A[x](теорема Безу)

\begin{Def}
Если для некоторого $k \ge 2$, $(x - c)^k|f$, но $(x - c)^{k + 1} \nmid f$, то говорим, что c "--- корень f кратности k.
\end{Def}

c "--- корень f кратности k, если  $f(x) = (x - c)^{k}g(x), (x - c) \nmid g(x) \Lra f(x) = (x - c)^{k}g(x), g(c) \ne 0$

\begin{theorem}{}
A "--- поле, $char A = 0, f \in A[x], f \ne 0$

c "--- корень f кратности $k \ge 1 \Lra$
\begin{enumerate}
\item c "--- корень f.
\item с "--- корень f' кратности k - 1.
\end{enumerate}

\end{theorem}

\begin{proof}
$$\Ra$$
$$f = (x - c)^{k}g(x), g(c) \ne 0 \Ra c \text{ "--- корень}$$
$$f' = k(x - c)^{k - 1}g(x) + (x - c)^{k}g' = (x - c)^{k - 1}(kg + (x - c)g')$$
$$\Ra (x - c)^{k - 1}|f'$$
c "--- не корень kg + (x - c)g'

$$kg(c) + (x - c)g'(c) = kg(c) \ne 0$$

$$\La$$
c "--- корень f $\Ra$ корень f кратности l, по доказаному с "--- корень f' кратности l - 1.
$$l - 1 = k - 1$$
$$l = k$$ 
\end{proof}

\begin{Rem}
Предположение $char A = 0$ существенно. 
$$\F_2, f = x^7 + x^2$$
0 "--- корень кратности 2.
$$f' = x^6$$
0 "--- кратности 6.
\end{Rem}
              
\begin{conseq}
    A "--- поле характеристики 0. $0 \ne f \in A[x]$, c "--- корень f кратности $\ge k \Lra$ выполняется равенство 
    $$0 = f(c) = f'(c) = \ldots = f^{(k - 1)}(c)$$
    $$f^{(k)} = (f^{(k - 1)})'$$
\end{conseq}

$$(fg)^(n) = \sum_{r = 0}^{n}C_n^r f^{(r)}g^{(n - r)}$$

