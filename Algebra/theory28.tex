\section{Кратные корни}
$A$ "--- поле. $f \in A[x]$, $f \ne 0$, $c$ "--- корень $f$ в $A$ тогда и только тогда, когда $(x - c) \mid f$ в $A[x]$ (теорема Безу).

\begin{Def}
Если для некоторого $k \ge 2$, $(x - c)^k \mid f$, но $(x - c)^{k + 1} \nmid f$, то говорим, что $c$ "--- корень $f$ кратности $k$.
\end{Def}

$c$ "--- корень $f$ кратности $k$, если 
\begin{gather*}
f(x) = (x - c)^{k}g(x) \\
(x - c) \nmid g(x) \Lra f(x) = g(c) \ne 0
\end{gather*}

\begin{theorem}{}
$A$ "--- поле, $\Char A = 0$, $f \in A[x]$, $f \ne 0$. Тогда $c$ "--- корень $f$ кратности $k \ge 1$ тогда и только тогда, когда
\begin{enumerate}
\item $c$ "--- корень $f$.
\item $c$ "--- корень $f'$ кратности $k - 1$.
\end{enumerate}
\end{theorem}
\begin{proof}
\begin{itemize}
\item[$\Ra$:]
\begin{gather*}
f = (x - c)^{k}g(x), g(c) \ne 0 \Ra \text{$c$ "--- корень степени $k$} \\
f' = k(x - c)^{k - 1}g(x) + (x - c)^{k}g' = (x - c)^{k - 1}(kg + (x - c)g') \\
\Ra (x - c)^{k - 1} \mid f' \\
\text{$c$ "--- не корень $kg + (x - c)g'$} \\
kg(c) + (x - c)g'(c) = kg(c) \ne 0
\end{gather*}
\item[$\La$:]
$c$ "--- корень $f$, поэтому корень $f$ кратности $1$, по доказаному $с$ "--- корень $f'$ кратности $l - 1$.
\begin{gather*}
l - 1 = k - 1 \\
l = k
\end{gather*}
\end{itemize}
\end{proof}

\begin{Rem}
Предположение $\Char A = 0$ существенно. 
\[ \F_2, f = x^7 + x^2 \]
$0$ "--- корень кратности $2$.
\[ f' = x^6 \]
$0$ "--- кратности $6$.
\end{Rem}
              
\begin{conseq}
$A$ "--- поле характеристики $0$. $0 \ne f \in A[x]$, $c$ "--- корень $f$ кратности $\ge k$ тогда и только тогда, когда выполняется равенство 
\begin{gather*}
0 = f(c) = f'(c) = \ldots = f^{(k - 1)}(c) \\
f^{(k)} = (f^{(k - 1)})'
\end{gather*}
\end{conseq}