\section{Кратные корни. Теорема о кратности корня многочлена и его производной}
Напоминание: если $A$ "--- поле, $f \in A[x]$, $f \neq 0$, то $c$ "--- корень $f$ в $A$ $\iff$ когда $(x - c) \mid f$ в $A[x]$ (по теореме Безу).

\begin{Def}
Если для некоторого $k \ge 2$ имеем $(x - c)^k \mid f$, но $(x - c)^{k + 1} \nmid f$, то говорим, что $c$ "--- корень $f$ кратности $k$.
\end{Def}

\begin{Rem}
Переформулировка: $c$ "--- корень $f$ кратности $k$, если:
\begin{gather*}
\begin{cases}
f(x) = (x - c)^{k}g(x) \\
(x - c) \nmid g(x) \Lra g(c) \ne 0
\end{cases}
\end{gather*}
\end{Rem}

\begin{theorem}{}
$A$ "--- поле, $\Char A = 0$, $f \in A[x]$, $f \ne 0$. Тогда $c$ "--- корень $f$ кратности $k \ge 1$ тогда и только тогда, когда
выполняются два условия:
\begin{enumerate}
\item $c$ "--- корень $f$.
\item $c$ "--- корень $f'$ кратности $k - 1$.
\end{enumerate}
\end{theorem}
\begin{proof}
\begin{itemize}
\item[$\Ra$:] то, что $c$ "--- корень $f$, очевидно. Далее:
\begin{gather*}
\text{$c$ "--- корень кратности $k$} \iff
  \begin{cases}
  f = (x - c)^{k}g(x), \\
  g(c) \ne 0
  \end{cases}
\end{gather*}
\begin{align*}
f' &= k(x - c)^{k - 1}g + (x - c)^{k}g' = \\
   &= (x - c)^{k - 1}(kg + (x - c)g') \\
\Ra (x - c)^{k - 1} \mid f' \\
\end{align*}

Покажем, что $c$ "--- не корень $(kg + (x - c)g')$ (то есть $c$ "--- корень $f'$ кратности ровно $k-1$):
\[kg(c) + (c - c)g'(c) = kg(c) \neq 0\]
так как $k\neq 0$ и $g(c) \neq 0$.

\item[$\La$:]
$c$ "--- корень $f$. Пусть $c$ имеет в $f$ кратность $l \ge 1$, по предыдущей часте теоремы $c$ "--- корень $f'$ кратности $l - 1$.
Таким образом, $l-1=k-1 \iff l=k$.
\end{itemize}
\end{proof}

\begin{Rem}
Предположение $\Char A = 0$ существенно:
\[ f = x^7 + x^2 \in \F_2[x] \]
Здесь $0$ "--- корень кратности $2$.
\[ f' = x^6 \]
Здесь $0$ "--- кратности $6$.
\end{Rem}

\begin{Rem}
Будем обозначать $k$-ю производную как $f^{(k)}$:
\begin{gather*}
f^{(0)} = f \\
f^{(1)} = f' \\
\dots \\
f^{(k)} = (f^{(k - 1)})'
\end{gather*}
\end{Rem}
              
\begin{conseq}
$A$ "--- поле характеристики $0$. $0 \ne f \in A[x]$, $c$ "--- корень $f$ кратности $\ge k$ тогда и только тогда, когда:
\begin{gather*}
0 = f(c) = f'(c) = \ldots = f^{(k - 1)}(c) \\
\end{gather*}
\end{conseq}