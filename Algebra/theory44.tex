\section{Теорема о пересечении высот треугольника}
\begin{theorem}{Высоты треугольника} 3 высоты треугольника пересекаются в одной точке.
\end{theorem}
\begin{proof}
\begin{center}
\def\svgwidth{6.0cm}
\input{theory44-altitudes.pdf_tex}
\end{center}

$(a, b), (c, d)$ "--- точки. Тогда $(a, b) \bot (c, d) \Lra ac + bd = 0$ "--- скалярное произведение.

$$z_1 = a + bi, z_2 = c + di, \Re (z_1\bar{z_2}) = ac + bd$$

Известно, что $(z_1 - w) \bot z_2$ и $(z_2 - w) \bot z_1$.

Надо доказать, что $w \bot z_1 - z_2$.
$$\left\{
\begin{aligned}
	z_1 - w \bot z_2 &\Lra \Re((z_1 - w)\bar{z_2}) = 0 \\
	z_2 - w \bot z_1 &\Lra \Re((z_2 - w)\bar{z_1}) = 0
\end{aligned}\right.
\Ra \Re(z_1\bar{z_2} - z_2\bar{z_1} + w(\bar{z_1} - \bar{z_2})) = 0 \Lra
$$
$$\Lra \Re(z_1\bar{z_2} - z_2\bar{z_1}) + \Re(w(\overline{z_1 - z_2})) = 0$$
$z_1\bar{z_2} - z_2\bar{z_1}$ "--- чисто мнимое, поэтому 
$$\Re(z_1\bar{z_2} - z_2\bar{z_1}) = \Ra \Re(w(\overline{z_1 - z_2})) = 0$$
\end{proof}

\underline{Упражнение:}
\begin{enumerate}
\item Медианы.
\item Биссектрисы.
\end{enumerate}