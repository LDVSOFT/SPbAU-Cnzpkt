\section{Теорема о пересечении высот треугольника}
\begin{theorem}{} 3 высоты треугольника пересекаются в одной точке.
\end{theorem}
\begin{proof}
Рисунок4.

$(a, b), (c, d)$ - точки. Тогда $(a, b) \bot (c, d) \Lra ac + bd = 0$ - скалярное произведение.

$z_1 = a + bi, z_2 = c + d_i, Re(z_1\overline{z_2}) = ac + bd$

Известно, что $(z_1 - w) \bot z_2$ и $(z_2 - w) \bot z_1$.

Надо доказать, что $w \bot z_1 - z_2$.
$$
\begin{cases}
	z_1 - w \bot z_2 \Lra Re((z_1 - w)\overline{z_2}) = 0 \\
	z_2 - w \bot z_1 \Lra Re((z_2 - w)\overline{z_1}) = 0
\end{cases}
$$$\Ra Re(z_1\overline{z_2} - z_2\overline{z_1} + w(\overline{z_1} - \overline{z_2})) = 0 \Lra$

$\Lra Re(z_1\overline{z_2} - z_2\overline{z_1}) + Re(w(\overline{z_1 - z_2})) = 0$

$z_1\overline{z_2} - z_2\overline{z_1}$ - чисто мнимое $\Ra Re(z_1\overline{z_2} - z_2\overline{z_1}) = 0$

$\Ra Re(w(\overline{z_1 - z_2})) = 0$ чтд.  

\end{proof}

\underline{Упражнение:}
\begin{enumerate}
\item Медианы.
\item Биссектрисы.
\end{enumerate}