\section{Гомоморфизмы групп. Свойства гомоморфизмов}
\begin{Def}
	$F, G$ - группы, $f: G \ra H$
	\begin{enumerate}
		\item $f$ - гомоморфизм, если $\forall a, b \in G \: f(ab) = f(a)f(b)$
		\item $f$ - изоморфизм, если гомоморфизм и биекция
	\end{enumerate}
\end{Def}

\begin{Def}
	$H, G$ - группы. Если между $H, G$ есть изоморфизм, то группы называются изоморфными: $G \cong H$      \\
\end{Def}						

\begin{theorem}{Свойства гомоморфизма}
	\begin{enumerate}
		\item $f(e_G) = e_H$
		\item $f(x^{-1}) = (f(x))^{-1}$
		\item $f(x) \subset H$
		\item $f: G \ra H, g: H \ra K$. $f, g$ - гомоморфизмы, тогда $g \circ f: G \ra K$ тоже гомоморфизм
		\item $f: G \ra H$ - изоморфизм, тогда $f^{-1} H \ra G$ - изоморфизм
	\end{enumerate}
\end{theorem}
\begin{proof}
	\begin{enumerate}
		\item $$f(e_G) = f(e_G e_G) = f(e_G)f(e_G) = e_H e_H = e_H$$
		\item $$f(e_G) = f(x x^{-1}) = f(x)f(x^{-1}) = e_H \Ra f(x^{-1}) = (f(x))^{-1}$$
		\item 
			$$e_H \in f(G)$$
			$$c, d \in f(G), \exists a, b \in G: c = f(a), d = f(b)$$
			$$cd = f(a)f(b) = f(ab), cd \in f(G)$$ 
			По п. 2 $f(a)^{-1} = f(a^{-1}) \in G \Ra G$ - подгруппа(замкнутость относительно операции и взятия обратного).
		\item
			$$a, b \in G$$
			$$(g \circ f)(ab) = g(f(ab)) = g(f(a)f(b)) = g(f(a))g(f(b)) = (g \circ f)(a)(g \circ f)(b)$$
		\item
			$$c, d \in H$$
			$$\exists a, b \in G: c = f(a), d = f(d)$$
			$$a = f^{-1}(c), b = f^{-1}(d)$$
			$$f^{-1}(cd) = ab = f^{-1}(c)f^{-1}(d)$$
			Тогда $f^{-1}$ - изоморфизм			
	\end{enumerate}   
\end{proof}		

Пример: $< \mathbb{R}, +>, < \mathbb{R}, \times>$ \\
$x \ra e^x$ - изоморфизм

