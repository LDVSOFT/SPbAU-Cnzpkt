\section{Теорема о разложении перестановки в произведение непересекающихся циклов}

\begin{Def}
	$S_n$ - множество биекций на $\{1, \dots, n\}$ - группа перестановок (симметрическая группа степени $n$) с операцией "композиция функций"
\end{Def}
\begin{theorem} {}
	$(S_n, \circ)$ - группа
\end{theorem}
\begin{proof}
	Так как композиция биекций - биекция, а обратное отображение к биекции - биекция, то $S_n$ замкнуто относительно этих операций. \\

	$$e = id_{\{1 \dots n \} }$$ \\

	Ассоциативность следует из ассоциативности композиций отображений \\

	Обратный элемент существует, так как замкнуто относительно взятия обратного
\end{proof}

\begin{Def}
	Цикл $\sigma = (i_1, \dots, i_k) \Lra \sigma(i_1) = i_2, \dots, \sigma(i_k) = i_1, \forall j \notin \{i_1, \dots, i_k \} \sigma(j) = j $ \\
	$k$ - длина цикла
\end{Def}

\begin{Def}
	Транспозиция - цикл длины 2. \\
\end{Def}

\begin{Def}
	Произведение циклов - композиция отображений, которые задают эти циклы \\
\end{Def}

\begin{Def}
	Циклы неперекаются, если $\{i_1, \dots i_k\} \cap \{j_1 \dots j_k\} = \emptyset$ \\
\end{Def}

\begin{theorem}{Всяка перестановка может быть представлена в виде произведения непересекающихся циклов}
\begin{Def}
	$\sigma \in S_n, j \in \{1, \dotsc, n\}. j$ -- неподвижная точка относительно $\sigma$, есди $\sigma(j) = j$
\end{Def}
\end{theorem}
\begin{proof}
	Индукция по $m$ -- числу неподвижных точек $\sigma$.\\
База: $m = 0 \Leftrightarrow n $ -- число неподвижных точек, то есть $\forall j \in \{1, \dotsc, n\} \sigma(j) = j, $ то есть $\sigma = id$\\
Переход: $m > 0 \Rightarrow \exists i: \sigma(i) \ne i$ -- с него и начнём.\\
$i_1 = i$\\
$i_2 = \sigma(i_1)$\\
$i_3 = \sigma(i_2) = \sigma^2(i_1)$\\
$\dotsb$\\
И так до тех пор, пока не встретим повторение(а его мы обязательно встретим, потому что чисел у нас всего $n$ -- конечное число)

$i_1 \mapsto i_2 \mapsto \dotsc \mapsto i_k \mapsto i_{k+1}$ -- уже встречался. 

Заметим, что $i_{k+1} \ne i_j \forall j \in \{2, \dotsc, n\}$ в силу инъективности \sigma (иначе у какого-то элемента было бы два различных прообраза -- $i_{j-1}$ и $i_{k+1}$) $\Rightarrow i_{k+1} = i_1$.  

Итак, мы получили цикл $\tau = (i_1, i_2, \dotsc, i_k)$. 

Рассмотрим $\sigma\tau^{-1}$. 

Неподвижные точки $\sigma\tau^{-1}$ - это неподвижные точки $\sigma $ плюс $i_1, i_2, \dotsc, i_k \Rightarrow \sigma\tau^{-1}$ и $\tau$ - незацепляющиеся $\Rightarrow($по индукционному предположению$) \sigma\tau^{-1} = \sqcap_{j = 1}^r \tau_j$. 

Домножим обе части равества на $\tau$ и получим, что $\sigma = (\sqcap_{j = 1}^r \tau_j)\tau$ -- произведение незацепляющихся циклов.\\
\end{proof}
\begin{theorem}{Следствие}
$S_n$ порождается всеми циклами (так как для каждой $\sigma$ можно выбрать свой набор).
\end{theorem}

Небольшой забавный бонус:\\
$\sigma = \sqcap_{j = 1}^r \tau_j$, где $\tau_j$ -- попарно непересекающиеся циклы. Тогда $\sigma^m = (\sqcap_{j = 1}^r \tau_j)^m$, так как непересекающиеся циклы коммутируют.\\
\begin{Def}
Прядок цикла -- его длина.
\end{Def}
\begin{Def}
Прядок $\sigma$ -- наименьшее общее кратноедлин циклов при разложении $\sigma$ на непересекающиеся циклы.
\end{Def}
\begin{Def}
Цикловым типом $\sigma \in S_n$ называется набор длин её непересекающихся циклов, упорядоченных по неубыванию, + набор единиц для неподвижных элементов.
\end{Def}