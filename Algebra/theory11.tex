\section{Подгруппы. Критерий того, что непустое подмножество группы является подгруппой. Пересечение подгрупп}
\begin{Def}
	Пусть $f: A \to B$ и $C \subset A$. Введём $g$ "--- сужение $f$ на $C$:
	\begin{gather*}
	g: C \ra B \\
	\forall c \in C\colon g(c) = f(c)
	\end{gather*}
\end{Def}

\begin{Def}
	$H \subset G$ "--- подгруппа в $G$, если она является группой относительно сужения операции в $G$ на $H$.
\end{Def}


\begin{Def}
	Множество $A$ замкнуто относительно операции $\cdot$, если $\forall a, b \in A \colon a \cdot b \in A$

	Множество $A$ замкнуто относительно операции взятия обратного, если $\forall a \in A \colon a^{-1} \in A$ \\
\end{Def}

\begin{theorem}{Достаточные условия для подгруппы}
Для того, чтобы доказать, что $H$ "--- подгруппа $G$ ($H \neq \varnothing$, $H \subset G$), достаточно проверить только замкнутость относительно операциий $\cdot$ и взятия обратного элемента. \\
\end{theorem}
\begin{proof}
Ассоциативность к нам переходит из исходной группы $G$.

Если существует обратный элемент, то $a a^{-1} = e \in H$

Так как есть замкнутость, то операция $\cdot$ действует из $H \times H$ в $H$.
\end{proof}

\begin{conseq}
Пусть $\varnothing \ne H \subset G$ и $\forall a, b \in H: ab^{-1} \in H \Ra H$ "--- подгруппа.  
\end{conseq}
\begin{proof}
Нейтральный элемент есть: $a \in H \Ra aa^{-1} \in H \Ra e \in H$.

Замкнутость относительно взятия обратного: $\forall a \in H \colon ea^{-1} \in H \Ra a^{-1} \in H$.

Замкнутость относительно операции $\cdot$: $\forall a, b \in H \Ra a, b^{-1} \in H \Ra a\left(b^{-1}\right)^{-1} \in H \Ra ab \in H$.
\end{proof}

\begin{theorem}{}
	$H_\alpha$ "--- подгруппы в $G$. Тогда $\bigcap H_\alpha$ "--- подгруппа в $G$.
\end{theorem}
\begin{proof}
    \begin{gather*}
	H = \bigcap H_\alpha; \\
	e \in H_{\alpha} \Ra e \in H \Ra H \ne \varnothing; \\
	a, b \in H \Ra \forall \alpha\colon a, b \in H_\alpha \Ra ab \in H_\alpha \Ra ab \in H; \\
	a \in H \Ra \forall \alpha \colon a \in H_\alpha \Ra a^{-1} \in H_\alpha \Ra a^{-1} \in H;
	\end{gather*}
\end{proof}
