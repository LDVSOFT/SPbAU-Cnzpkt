\section{Подгруппы. Критерий того, что непустое подмножество группы является под-
группой. Пересечение подгрупп}
\begin{Def}
	$H \subset G$ - подгруппа в $G$, если она является группой относительно сужения операции в $G$ на $H$. \\
\end{Def}

\begin{Def}
	$f: A \ra B$
	
	$C \subset A$
	
	$g: C \ra B, \forall c \in C g(c) = f(c)$
	Тогда $g$ - сужение $f$ на $C$	\\
\end{Def}

\begin{Def}
	Множество $A$ замкнуто относительно операции $\cdot$, если $\forall a, b \in A a \cdot b \in A$
	Множество $A$ замкнуто относительно операции взятия обратного, если $\forall a \in A a^{-1} \in A$ \\
\end{Def}

\begin{theorem}{Достаточные условия для подгруппы}
Для того, чтобы доказать, что $H$ - подгруппа $G$, достаточно проверить только замкнутость относительно операциий $\cdot$ и взятия обратного элемента. \\
\end{theorem}
\begin{proof}
Ассоциативность к нам переходит из исходной группы $G$.
Если существует обратный элемент, то $a a^{-1} = e \in H$
Так как есть замкнутость, то операция $\cdot$ действует из $H \times H$ в $H$.   \\
\end{proof}

\begin{theorem}{}
	$H_\alpha$ - подгруппы в $G$. Тогда $\cap H_\alpha$ - подгруппа в $G$.
\end{theorem}
\begin{proof}
	$$H = \cap H_\alpha$$
	$$a, b \in H \Ra \forall \alpha \: a, b \in H_\alpha \Ra ab \in H_\alpha \Ra ab \in H$$
	$$a \in H \Ra \forall \alpha \: a \in H_\alpha \Ra a^{-1} \in H_\alpha \Ra a^{-1} \in H$$
	
\end{proof}
