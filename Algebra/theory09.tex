\section{Отношение эквивалентности}

\begin{Def}
    $R$ "--- отношение эквивалентности, если оно рефлексивное, симметрично и транзитивно. Стандартное
    обозначение: $a \sim b$.
\end{Def}
\begin{Def}
    Для $a \in A$ определён
    класс эквивалентности: $[a] = \{b \in A \mid a \sim b\}$
\end{Def}

\begin{theorem}{}
Пусть $a, b \in A$. Тогда либо $[a]\cap [b] = \varnothing$, либо $[a] = [b]$ 
\end{theorem}

\begin{proof}
\begin{enumerate}
\item $[a] \cap [b] = \varnothing$ "--- всё доказано. Заметим, что тогда $a \nsim b$, в противном случае $a, b \in [a], [b]$.
\item $\exists c \in [a] \cap [b]$. Тогда $c \sim a$ и $c \sim b$ $\Ra a \sim b$.

Покажем, что $[a] \subset [b]$ и $[b] \subset [a]$:

\begin{enumerate}
\item $x \in [a] \Ra x \sim a \Ra x \sim b \Ra x \in [b]$
\item Аналогично
\end{enumerate}
\end{enumerate}

\end{proof}

Множество классов эквивалентности называется фактормножеством по отношению $R$.
