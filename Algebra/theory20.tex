\section{Кольца, тела, поля}
$A \neq \oslash$\\
$+: A \times A \ra A$\\
$\cdot: A \times A \ra A$\\

\begin{enumerate}
\item ассоциативнсть сложения:
	 $$ \forall a, b, c \in A (a + b) + c = a + (b + c) $$
\item существование нейтрального элемента по сложению:
	$$ \exists 0 \in A \forall a \in A a + 0 = 0 + a = a $$
\item существование обратного элемента по сложению:
	$$ \forall a \in A \exists -a \in A a + (-a) = (-a) + a = 0 $$
\item коммутативность сложения:
	$$ \forall a, b \in A a \cdot b = b \cdot a $$
\item ассоциативность умножения:
	$$ \forall a, b, c \in A a \cdot b = b \cdot a $$
\item коммутативность умножения:
	$$ \forall a, b \in A a \cdot b = b \cdot a $$
\item существование нейтрального элемента по умножению:
	$$ \exists 1 \in A \forall a \in A a \cdot 1 = 1 \cdot a = a $$
\item существование обратного элеменрта по умножению:
 	$$ \forall a \in A \setminus \lbrace 0 \rbrace \exists a^{-1} \in A a \cdot a^{-1} = a^{-1} \cdot a = 1 $$
\item дистрибутивность: \\
	a) $ \forall a, b, c \in A (a + b) \cdot c = a \cdot c + b \cdot c $\\
	b) $ \forall a, b, c \in A c \cdot (a + b) = c \cdot a + c \cdot b $\\
\end{enumerate}

\begin{Def}
	Кольцо - непустое множество $A$ с операциями $"+"$, $"\cdot"$, удовлетворяющее свойствам 1 - 5, 9 (a, b)
\end{Def}
\begin{Def}
	Кольцо, в котором выполнена аксиома 6 - коммутативное кольцо
\end{Def}
\begin{Def}
	Кольцо, в котором выполнена аксиома 7 - кольцо с единицей
\end{Def}
\begin{Def}
	Тело - кольцо с $1$, в котором $1 \neq 0$ и выполнена аксиома 8
\end{Def}
\begin{Def}
	Поле - коммутативное кольцо с $1$, в котором $1 \neq 0$ и выполнена аксиома 8 (т.е. все 9 аксиом)
\end{Def}

\begin{Rem}
	иногда кольца, для которых выполнены аксиомы 1-4, 9 называют ассоциативными кольцами
\end{Rem}

\begin{Rem}
$(A, +, \cdot)$ - кольцо, $(A, +)$ - абелева группа\\
\end{Rem}

\textbf{ Примеры: }
\begin{itemize}
\item $\Z$ - коммутативное кольцо с $1$, но $2$ не имеет обратного в $\Z$ $\Ra$ не поле
\item $N$ - не кольцо
\item $2\Z$ - кольцо без $1$
\item $\Q, \R$ - поля
\end{itemize}

\textbf{Простейшие свойства колец}

\begin{enumerate}
\item $0$ - единственный
\item $-a$ - единственный
\item $1$ - единственная (если есть) 
	\begin{proof}
		$$ 1 = 1 \cdot 1' = 1 $$
	\end{proof}
\item если у $a$ есть обратный по умножению, то он единственен 
	\begin{proof}
		$$ a' = a'aa''=a'' $$
	\end{proof}
\item если в кольце с 1 у элемента $a$ есть 2 левых обратных, то левых обратных к $a$ бесконечно много (упражнение)
	
\item $0 \cdot a = a \cdot 0 = 0 $
	\begin{proof}
		$$ a \cdot 0 + a \cdot 0 = a(0 + 0) = a\cdot0 |  + (a\cdot0)' $$
		$$ a \cdot 0 + a \cdot 0 + (a \cdot 0)' = a \cdot 0 + (a \cdot 0)' = a \cdot 0 = 0 $$
	\textit{ второе равенство аналогично }
	\end{proof}
\item $ a(-b) = (-a)b = -(ab)$
	\begin{proof}
		$$ a + (-a) = 0 $$
		$$ ab + (-a)b = (a + (-a))b = ab = 0 $$
		$$ (-a)b = -ab $$
	\textit { второе равенство аналогично }
	\end{proof}
\item $ 0 = 1, возможно лишь если |A| = 1, A = \lbrace 0 \rbrace $
	\begin{proof}
		$$ a \in A a = 1 \cdot a = 0 \cdot a = 0 $$
	\end{proof}
\end{enumerate}

\begin{Def}
$A$ - кольцо (тело, поле)\\
	$A \supseteq B \neq \oslash$ - подкольцо (подтело, подполе), если являктся кольцом (телом, полем), относительно сужения операций на $B$\\
\end{Def}


\begin{Rem}
\begin{itemize}
	\item $ B \neq \oslash ,	B \supset A$ - подкольцо в $A$ если оно замкнуто относительно умножения, сложения, взятия обратного по сложению
	\item $B$ - подтело, если подкольцо и замкнуто по взятию обратного ненулевого элемента по умножению и содержит элементы отличные от нуля:\\
	$ \forall a, b \in B a + b \in B$\\
	$ \forall a \in B a -a \in B$\\
	$ \forall a,b \in B ab \in B$\\
	$ \forall a \in B \setminus a^{-1} \in B$\\
\end{itemize}
\end{Rem}

\begin{Def}
 $A, B$ - кольца\\
	$f: A \ra B$\\
	$f$ - гомоморфизм, если:\\
		$$ \forall a_1, a_2 \in A f(a_1 + a_2) = f(a_1) + f(a_2) $$
		$$ \forall a_1, a_2 \in A f(a_1a_2) = f(a_1)f(a_2) $$
\end{Def}

\begin{Def}		
	$f$ - изоморфизм, если $f$ - гомоморфизм и биекция\\
\end{Def}

$A, B$ изоморфны, если существует изоморфизм между $A$ и $B$
	 $$ A \cong B $$
\begin{Rem}
	$f$ - гомоморфизм и $f(0_A)$обратим по сложению, тогда $f(0_A) = 0_B$
\end{Rem}
\begin{proof}
	$f(0_A) = f(0_A + 0_A) = f(0_A) + f(0_A)$, говорим, что у $f(1_A)$ есть обратный по сложению,
	прибавляем его и получаем: $f(0_A) = 0_B$
\end{proof}

\begin{Rem}
	Если $f$ - гомоморфизм и $f(1_A)$ обратим в B то $f(1_A)=1_B$
\end{Rem}
\begin{proof}
	$f(1_A) = f(1_A \cdot 1_A) = f(1_A)f(1_A)$, говорим, что у $f(1_A)$ есть обратный по умножению,
	умножаем на него и получаем: $f(1_A) = 1_B$
\end{proof}
	
\textbf{Делимость в кольцах}

$A$ - кольцо, $a, b, c \in A, c = ab$\\
$a$ - левый делитель $c$\\
$b$ - правый делитель $c$\\
$ 0 = a, 0 = 0 \cdot b$ \\
    
\begin{Def}
	$a, b$ - нетривиальные делители нуля, eсли $0 = ab, a \neq 0, b \neq 0$\\
\end{Def}
	
\begin{Def}	
	Область целостности - коммутативное кольцо с $1$, без нетривиальных делителей нуля\\
\end{Def}

\begin{Rem}
	Поле - область целостности ($\forall a, b (ab = 0 \Ra a = 0 \vee b = 0)$)\\
\end{Rem}
	
\begin{theorem}{}
	$A$ - область целостности $a \in A \setminus \lbrace 0 \rbrace$\\
	$ab = ac \Ra b = c$
\end{theorem}
	
\begin{proof}
	$$ \underbrace{a}_{neq 0}(b - c) = 0 \Ra b - c = 0 \Ra b = c$$
\end{proof}	
