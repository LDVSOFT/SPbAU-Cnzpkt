\section{Бинарные отношения}
\begin{Def}
На А задано бинарное отношение R, если задано $R \subset A$

$(a, b) \in R$

a и b находятся в отношение с R

$a R b$

R = 0 пустое

$R = A^2$ полное. 
\end{Def}

\begin{Def}
$A, R \subset A^2$
\begin{enumerate}
\item R рефлексивно,если $\forall a \in A, aRa (a, a)\in R$
\item R антирефлексивно, если $\forall a \in A \neg (aRa)$
\item R симметрично, если $\forall a, b \in A aRb \Ra bRa$
\item R асимметрично, если $\forall a, b \in A aRb \Ra \neg(bRa)$
\item R антисимметрично, если $\forall a, b \in A (aRb \wedge bRa) \Ra a = b$
\item R транзитивно, если  $\forall a, b, c \in A (aRb \wedge bRc) \Ra aRc$
\end{enumerate}
\end{Def}

\begin{Def}
R называется отношением несторого частичного порядка, если оно рефлексивно, транзетивно и антисимметрино. 
\end{Def}
\begin{Def}
R называется отношением сторого частичного порядка, если оно антирефлексивно, транзетивно и асимметрино. 
\end{Def}

Если на А задано отношение частичного порядко, то А "--- частично упорядоченное множество.