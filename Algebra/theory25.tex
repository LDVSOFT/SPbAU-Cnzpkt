\section{Теорема Безу}

\begin{Def}{\bf Значение многочлена в точке}
Пусть есть $A \subset B$, и $B$ "---  коммутативное кольцо с 1 (будем уметь подставлять
в многочлены из $\Z[x]$ не только целые числа). Есть многочлен:
\[f \in A[x], f = a_nx^n + \ldots + a_0, c \in B\]
Тогда значение многочлена в точке $c$ есть
\[f(c) = a_nc^n + \ldots + a_0 = \sum_{k = 0}^{n}a_kc^k\]
\end{Def}

\begin{Rem}
Таким образом мы многочленом $f \in A[x]$ задали отображение
$\tilde f : B \to B$, которое переводит $c \to f(c)$. Имеются свойства:
\begin{enumerate}
\item $(f+g)(c)=\tilde f(c) + \tilde g(c)$
\item $(f\cdot g)(c)=\tilde f(c) \cdot \tilde g(c)$
\end{enumerate}
\end{Rem}

\begin{theorem}{}
\textbf{(Безу)}
	Пусть $A$ "--- коммутативное кольцо с $1$, $c \in A$ и $f \in A[x]$. Тогда:
	\[\exists q \in A[x] \colon f(x) = (x - c)q(x) + f(c)\]
\end{theorem}

\begin{proof}
	Рассмотрим $x - c$, по теореме о делении многочленов с остатком получаем:
	\begin{gather*}
	f(x) = (x - c)q(x) + r_0; \\
	\deg r_0 < \deg (x - c) = 1; \\
	\Ra r = r_0 \in A; \\
	f(x) = (x - c)q(x) + r_0; \\
	f(c) = (c - c)q(c) + r_0; \\
	f(c) = r_0; \\
	\Ra f(x) = (x - c)q(x) + r_0; \\
	\end{gather*}
\end{proof}

\begin{Def}
	$A$, $B$ "--- коммутативные кольца с 1 и $A \subseteq B$; $f \in A[x]$. Тогда
	$c \in B$ "--- корень $f$, если $f(c) = 0$.
\end{Def}

\begin{conseq}
	$c$ "--- корень $\Lra$ $(x - c) \mid f$\\
\end{conseq}
\begin{proof}
	\begin{itemize}
	\item $\La$: $f(x) = (x - c)g(x) \Ra f(c) = (c-c)g(c) = 0$ $\Ra$ $c$ - корень
	\item $\Ra$: $f(c) = 0 \Ra$ теорема Безу $\Ra f(x) = (x-c)g(x) + f(c) = (x-c)g(x) \Ra (x-c) | f$
	\end{itemize}
\end{proof}
