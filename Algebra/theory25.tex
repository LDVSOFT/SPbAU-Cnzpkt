\section{теорема Безу}
{\bf Значение многочлена в точке}

$A \subset B$ ~---  коммутативное кольцо с 1.

$$f \in A[x], f = a_nx^n + \ldots + a_0, c \in B$$

$$f(c) = a_nc^n + \ldots + a_0 = \sum_{k = 0}^{n}a_kc^k \text{~--- значение многочлена в точке}$$

$$A[x] \subset B[x]$$

$$f \in A[x]$$

$$\tilde f: B \to B$$
$$c \to f(c)$$
$$(\tilde f +(*) \tilde g)(c) = \tilde f(c) +(*) \tilde g(c)$$ 
$$\tilde{f +(*) g} = \tilde f +(*) \tilde g$$

\begin{theorem}{}
\textbf{(Безу)}\\
	$A$ - коммутативное кольцо с $1$
	$$ c \in A, f \in A[x] \Ra \exists q \in A[x]: f(x) = (x - c)q(x) + f(c) $$
\end{theorem}

\begin{proof}
	Рассмотрим $x - c$, по теоремо о делении с остаток получаем:
	$$ f(x) = (x - c)q(x) + r_0, \deg r_0 < \deg (x - c) = 1 $$
	$$ r = r_0 \in A $$
	$$ f(x) = (x - c)q(x) + r_0 $$
	$$ f(c) = (c - c)q(c) + r_0 $$
	$$ f(c) = r_0 \Ra f(x) = (x - c)q(x) + r_0 $$
\end{proof}

\begin{Def}
	$A, B, A \subseteq B$, коммутативные с 1\\
	$f \in A[x]$\\
	$c \in B$ - корень $f$, если $f(c) = 0$\\
\end{Def}

\begin{conseq}
	$c$ - корень $\Lra$ $(x - c) | f$\\
\end{conseq}

\begin{proof}
	$"\La"$	$f(x) = (x - c)g(x) \Ra f(c) = (c-c)g(c) = 0$ $\Ra$ $c$ - корень\\
	$"\Ra"$	$f(c) = 0 \Ra$ теорема Безу $\Ra f(x) = (x-c)g(x) + f(c) = (x-c)g(x) \Ra (x-c) | f$
\end{proof}
