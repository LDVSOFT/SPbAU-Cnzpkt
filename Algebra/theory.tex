%% texify: XeLaTeX + MakeIndex + BibTeX, modificated
%% texify
%% --pdf
%% --engine=xelatex
%% --tex-option=$synctexoption // You may delete this, affords to skip from preview to code in one click, that i seldom do
%% --tex-option=-8bit // Else minted fails on tabs
%% --tex-option=--shell-escape // For minted to live
%% $fullname
%% I prefer to build with TeXworks for better view of errors and warnings. It's hard to read all log file. 

\documentclass[12pt,a4paper]{article}
\usepackage{polyglossia} %% Better than babel on XeLaTeX
\usepackage{amsmath, amssymb} %% Cool math!
\usepackage{color} %% Coloring almost anything
\usepackage[russian]{hyperref} %% Clickable links is pdf
\usepackage{indentfirst}
\usepackage{ifthen}
\usepackage[left=1cm,right=1cm,top=2cm,bottom=2cm]{geometry}
\usepackage{wrapfig}
%% WARNING: latest minted is used. Download from github!
%% Works fine, though
%\usepackage{minted} %% Highlighting code. Installation is hard: requires python2 and script Pygments. Look at documentation for help!
\usepackage[math-style=ISO,vargreek-shape=unicode]{unicode-math} %% MAGIC! INCLUDE AS LAST!

\setdefaultlanguage[spelling=modern]{russian} %% Languages for polyglossia
\setotherlanguage{english}

\defaultfontfeatures{Ligatures={TeX}} %% Fonts and ligatures.
\setmainfont{CMU Serif} %% There are original Knuth's fonts in Unicode, called Computer Modern Unicode. Download anywhere, just install them
\setsansfont{CMU Sans Serif}
\setmonofont{CMU Typewriter Text}  
\setmathfont{Latin Modern Math} %% Download too. You may change it :)
\AtBeginDocument{\def\setminus{\mathbin{\backslash}}}
%\setmathfont{XITS}

%% Magic as black as my working table
%\DeclareSymbolFont{cyrletters}{\encodingdefault}{\familydefault}{m}{it}
%\newcommand{\makecyrmathletter}[1]{%
%  \begingroup\lccode`a=#1\lowercase{\endgroup
%  \Umathcode`a}="0 \csname symcyrletters\endcsname\space #1
%}
%\count255="409
%\loop\ifnum\count255<"44F
%  \advance\count255 by 1
%  \makecyrmathletter{\count255}
%\repeat
%% Simpy adds cyrillic to maths!

\frenchspacing %% One space before sentence, not two!

%% Shortcuts:
\def\la{\leftarrow}
\def\ra{\rightarrow}
\def\lra{\leftrightarrow}
\def\La{\Leftarrow}
\def\Ra{\Rightarrow}
\def\Lra{\Leftrightarrow}
\def\lrh{\leftrightharpoons}
\def\xra{\xrightarrow} 
\def\btu{\bigtriangleup} 
\def\rat{\rightarrowtail}
\def\thra{\twoheadrightarrow}
\def\thrat{\twoheadrightarrowtail}

\def\N{\mathbb{N}}
\def\Z{\mathbb{Z}}
\def\Q{\mathbb{Q}}
\def\R{\mathbb{R}}
\def\C{\mathbb{C}}

\def\LraDef{\stackrel{\mathrm{Def}}{\Lra}}
\def\eqDef{\stackrel{\mathrm{Def}}{=}}
\def\d{\mathup{d}}

% ======================================

%% Change Chapter and Section numeration style
%%\renewcommand{\thechapter}{\Roman{chapter}}
%%\renewcommand{\thesection}{\thechapter.\arabic{section}}

%% Indent for first par in chapter
\makeatletter
%%\renewcommand{\chapter}{\clearpage %% no double page, only
\thispagestyle{empty}%% not plain, empty. wanna number of page!
\global\@topnum=0
\@afterindenttrue %% Set to true!
%%\secdef\@chapter\@schapter}
\makeatother

%% Environment for theorem body
\newcounter{theorem}[section]
\renewcommand{\thetheorem}{\thesection.\arabic{theorem}}
\newcommand*{\theoremheader}[1]{\par\refstepcounter{theorem}%
\textbf{Теорема \thetheorem.\ifthenelse{\equal{#1}{}}{}{ #1.}}}
\newenvironment*{theorem}[1]{
\theoremheader{#1}%
}{%
\par%
}

%% Environment for consequence body
\newcounter{conseq}[theorem]
\renewcommand{\theconseq}{\thetheorem.\arabic{conseq}}
\newcommand*{\conseqheader}{\par\refstepcounter{conseq}%
\textit{Следствие \theconseq.}}
\newenvironment*{conseq}{
\conseqheader%
}{%
\par%
}

\newcounter{lemma}[section]
\renewcommand{\thelemma}{\thesection.\arabic{lemma}}
\newcommand*{\lemmaheader}{\par\refstepcounter{lemma}%
\textit{Лемма \thelemma.}}
\newenvironment*{lemma}{
	\lemmaheader%
}{%
}

\newenvironment{assertion}{%
\par\textbf{Утверждение. }%
}{%
\par%
}

%% Environment for proof body. I like this style, but you are free to change it.
\newenvironment{proof}{%
\par$\blacktriangleright$%
}{%
\hfill$\blacktriangleleft$%
}

%% Environment for definitions. Pretty raw one.
\newenvironment{Def}{%
\par$\mathfrak{Def\colon}$%
}{%
\par%
}

%% Environment for remarks.
\newenvironment{Rem}{%
\par\textit{REM: }%
}{%
\par%
}

\setcounter{MaxMatrixCols}{40}

% ==================================

%% In-line code highlighting. Using: \py|a = input()|
%\newmintinline[cinl]{c}{} %\c is defined :(
%\newmintinline[cpp]{cpp}{}
%\newmintinline[python]{python}{}
%\newmintinline[bash]{bash}{}
%\newmintinline[make]{make}{}

%% Escaped code highlighting. Using: \begin{cppcode} ... \end{cppcode}
%\setminted{obeytabs,tabsize=4,linenos,texcomments}
%\newminted{c}{}
%\newminted{cpp}{}
%\newminted{python}{}
%\newminted{bash}{}
%\newminted{make}{}

% ==================================
\DeclareMathOperator{\Int}{int}
\DeclareMathOperator{\cl}{cl}
\DeclareMathOperator{\diam}{diam}
\newcommand{\emod}[1]{\mathop{\equiv}\limits_{#1}}

\newcommand{\Choose}[2]{{\left(#1 \atop #2\right)}}

\begin{document}
\begin{center}
  {\Large \bf Лекции по алгебре} \\ 
  \vspace{0.5em}
  {\Large \bf Лектор: Всемирнов Максим Александрович} \\
\end{center}

\vspace{-1em}
\noindent \underline{\hbox to 1\textwidth{{ } \hfil{ } \hfil{ } }}

\vspace{1em}
\tableofcontents
\pagebreak

\section{Множества}

Не любая совокупность элементов --- множество. Про каждый объект можно сказать, принадлежит ли он множеству ($x \in A$) или нет ($x \notin A$).

\begin{Def}
Множество $A$ - подмножество $B$, если все элементы $A$ содержатся и в $B$. 
$$ A \subset B \LraDef \forall x \in A\; x \in B $$
\end{Def}
\begin{Def}
Множества называются равными, если они содержатся друг в друге.
$$ A = B \LraDef A \subset B \land B \subset A $$
\end{Def}
\begin{Def}
Пустое множество --- это множество без элементов.
$$ \forall x\: x \notin \emptyset $$
\end{Def}
\begin{Def}
$2^A$ --- множество всех подмножеств $A$.
$$ 2^A \eqDef \left\{B \mid B \subset A \right\} $$
\end{Def}

\begin{itemize}
\item $\N$ --- множество натуральных чисел. 
\item $\Z$ --- множество целых чисел.
\item $\Q$ --- множество рациональных чисел.
\item $\R$ --- множества вещественных чисел.
\item $\C$ --- множества комплексных чисел.
\end{itemize}

Задание множеств:
\begin{itemize} 
\item $\left\{a,b,c\right\}$
\item $\left\{a_1, a_2, \ldots, a_n\right\}$
\item $\left\{a_1, a_2, \ldots\right\}$
\item $\left\{x \in A \mid \Phi(x)\right\}, \Phi(x) - \text{условие}$.
\end{itemize} 
Например, $\left\{p \in \N \mid p \text{ имеет ровно 2 натуральных делителя}\right\}$.

Бывают некорректно заданные <<множества>>. Например, множество художественных произведений на русском языке --- плохо заданное множество. Рассмотрим 
$\Phi(n)$ --- истина, если n нельзя записать в не более чем тридцать слов русского языка. Тогда
$\left\{n \in \N \mid \Phi(n)\right\}$~--- не множество. Если бы это было множеством, то в нём есть наименьший элемент, 
который описывается как <<Наименьший элемент множества...>>

\begin{Def}
Пересечение двух множеств~--- множество, состоящие из всех элементов, находящихся одновременно в обоих множествах.
$$ A \cap B \eqDef \left\{x \in A \mid x \in B \right\} $$
\end{Def}
\begin{Def}
Объединение двух множеств~--- множество, состоящее из элементов обоих множеств.
$$ A \cup B \eqDef \left\{x \mid x \in A \lor x \in B \right\} $$
\end{Def}
\begin{Def}
Разность множеств~--- это множество тех элементов, которые лежат в первом, но не во втором.
$$ A \setminus B \eqDef \left\{ x \in A \mid x \notin B \right\}$$
\end{Def}
\begin{Def}
Симметрическя разность~--- объединение разностей.
$$ A \btu B \eqDef \left(A \setminus B\right) \cup \left(B \setminus A\right) $$
\end{Def}

Объединение и пересечение множно записать для многих множеств.
$$ \bigcup_{i \in I} A_i = \left\{x \mid \exists i \in I\colon x \in A_i\right\}; 
\bigcap_{i \in I} A_i = \left\{x \mid \forall i \in I\: x \in A_i \right\} $$

Свойства операций со множествами:
\begin{enumerate}
\item Ассоциативность
$$ A \cap B = B \cap A; A \cup B = B \cup A $$
\item Коммутативность
$$ \left(A \cap B \right) \cap C = A \cap \left(B \cap C \right); \left(A \cup B \right) \cup C = A \cup \left(B \cup C \right) $$
\item Рефлексивность
$$ A \cap A = A; A \cup A = A $$
\item Дистрибутивность
$$ A \cap \left(B \cup C \right) = \left(A \cap B\right) \cup \left(A \cap C \right) $$
$$ A \cup \left(B \cap C \right) = \left(A \cup B\right) \cap \left(A \cup C \right) $$
\item Нейтральный элемент
$$ A \cap \emptyset = \emptyset$$
$$ A \cup \emptyset = A$$
\end{enumerate}

\begin{theorem}{Правила де Моргана}
$ A, B_\alpha, \alpha \in I $.
Тогда 
$$ A \setminus \bigcup_{\alpha \in I} B_\alpha = \bigcap_{\alpha \in I} \left(A \setminus B_\alpha\right) ; 
A \setminus \bigcap_{\alpha \in I} B_\alpha = \bigcup_{\alpha \in I} \left(A \setminus B_\alpha\right) $$
\end{theorem} 
\begin{proof}
$$
x \in A \setminus \bigcup_{\alpha \in I} B_{\alpha} \Lra \left\{\begin{aligned}x &\in A \\ x &\notin \bigcup_{\alpha \in I} B_{\alpha}\end{aligned}\right. \Lra 
\left\{\begin{aligned} x &\in A \\ \forall \alpha \in I\: x &\notin B_\alpha \end{aligned}\right. \Lra
\forall \alpha \in I\: \left\{\begin{aligned} x &\in A \\ x &\notin B_\alpha \end{aligned}\right.  
\Lra x \in \bigcap_{\alpha \in I} \left(A \setminus B_\alpha\right) 
$$
$$
x \in A \setminus \bigcap_{\alpha \in I} B_{\alpha} \Lra \left\{\begin{aligned}x &\in A \\ x &\notin \bigcap_{\alpha \in I} B_{\alpha}\end{aligned}\right. \Lra 
\left\{\begin{aligned} x &\in A \\ \lnot \forall \alpha \in I\: x &\in B_\alpha \end{aligned}\right. \Lra
\exists \alpha \in I\colon \left\{\begin{aligned} x &\in A \\ x &\notin B_\alpha \end{aligned}\right.  
\Lra x \in \bigcup_{\alpha \in I} \left(A \setminus B_\alpha\right) 
$$
\end{proof}

\begin{theorem}{Обобщение дистрибутивности}
$ A, B_\alpha, \alpha \in I $.
Тогда 
$$ A \cap \bigcup_{\alpha \in I} B_\alpha = \bigcup_{\alpha \in I} (A \cap B_\alpha) $$
$$ A \cup \bigcap_{\alpha \in I} B_\alpha = \bigcap_{\alpha \in I} (A \cup B_\alpha) $$
\end{theorem}
\begin{proof}
$$
x \in A \cap \bigcup_{\alpha \in I} B_{\alpha} \Lra \left\{\begin{aligned}x &\in A \\ x &\in \bigcup_{\alpha \in I} B_{\alpha}\end{aligned}\right. \Lra 
\left\{\begin{aligned} x &\in A \\ \exists \alpha \in I\colon x &\in B_\alpha \end{aligned}\right. \Lra
\exists \alpha \in I\colon \left\{\begin{aligned} x &\in A \\ x &\in B_\alpha \end{aligned}\right.  
\Lra x \in \bigcup_{\alpha \in I} \left(A \cap B_\alpha\right) 
$$
$$
x \in A \cup \bigcap_{\alpha \in I} B_{\alpha} \Lra \left[\begin{aligned}x &\in A \\ x &\in \bigcap_{\alpha \in I} B_{\alpha}\end{aligned}\right. \Lra 
\left[\begin{aligned} x &\in A \\ \forall \alpha \in I\: x &\in B_\alpha \end{aligned}\right. \Lra
\forall \alpha \in I\: \left[\begin{aligned} x &\in A \\ x &\in B_\alpha \end{aligned}\right.  
\Lra x \in \bigcap_{\alpha \in I} \left(A \cup B_\alpha\right) 
$$
\end{proof}

\begin{Def}
Упорядоченная пара $\langle a, b \rangle$ или $(a, b)$ --- объект
$$ (a_1; b_1) = (a_2; b_2) \LraDef a_1 = a_2 \land b_1 = b_2 $$
\end{Def}
\begin{Def}
Упорядоченная $n$-ка, или кортеж --- объект
$$ (a_1, a_2, \ldots, a_n) = (b_1, b_2, \ldots, b_n) \LraDef \forall i=1..n\: a_i = b_i $$
\end{Def}

\begin{Def}
Декартого произведение множеств --- множество кортежей, состоящих из элементов соответствующих множеств.
$$ \left(a_1, a_2, \ldots, a_n\right) \in A_1 \times A_2 \times \ldots \times A_n \LraDef \forall i=1..n\: a_i \in A_i $$
\end{Def}
\section{Обратимые отображения и их свойства}

$f: A \to B$

\begin{Def}
f "--- обратное справа, если $\exists g: B \to A$

$f \circ g = id_B$

f "--- обратим слева, если $\exists g: B \to A$

$g \circ f = id_A$

f обратимо, если $\exists g: B \to A$

$$g \circ f = id_A, f \circ g = id_B$$

g "--- отображение, обратное к f.(обозначение $f^{-1}$)
\end{Def}

\begin{theorem}{}

\begin{enumerate}
\item f обратимо $\Lra$ f обратимо слава и справа.
\item f обратимо, то обратное отображение единственно.
\end{enumerate}

\end{theorem}

\begin{proof}
\begin{enumerate}
\item f обратимо $\Ra$ f обратимо слева и справа.

Если у f есть и левый и правый обратный, то они совпадают. 

g "--- правый обратный к f, h "--- левый.

$(h \circ f) \circ g = id_A \circ g = g$

$h \circ (f \circ g) = h \circ id_B = h$

$\Ra g = h$

\item Пусть f обратимое и g и h "--- два обратных. В частности g "--- обратное справа, h "--- обратное слева.
\end{enumerate}
\end{proof}

\begin{theorem}{}
$f:A \to B, g:B \to C$

$g \circ f: A \to C$

\begin{enumerate}
\item Если f, g обратимы справа, то и $g \circ f$ обратима справа.
\item Если f, g обратимы слева, то и $g \circ f$ обратима слева.
\item Если f, g обратимы, то $g \circ f$ обратима $(g \circ f)^{-1} = f^{-1} \circ g^{-1}$
\end{enumerate}
\end{theorem}

\begin{proof}
\begin{enumerate}
\item
$$u: B \to A, f \circ u = id_B$$
$$v: C \to B g \circ v = id_C$$
$$(g \circ f) \circ (u \circ v) = g \circ (f \circ (u \circ v)) = $$
$$= g \circ ((f \circ u) \circ v) = g \circ (id_B \circ v) = g \circ v = id_C$$

$u \circ v$ "--- правый обратный к $g \circ f$

\item аналогично

\item 
$$(g \circ f)(f^{-1} \circ g^{-1}) = g \circ ((f \circ f^{-1}) \circ g^{-1}) = g \circ (id_B \circ g^{-1}) = g \circ g^{-1} = id_C$$

$$(f^{-1} \circ g^{-1})\circ(g \circ f) = f^{-1}(g^{-1} \circ g) \circ f = f^{-1} \circ id_B \circ f = f^{-1} \circ f = id_A$$

\end{enumerate}
\end{proof}

\begin{conseq}{}
Композиция сюръективных "--- сюръективна.

Композиция инъективных "--- инъективна.

Композиция биективных "--- биекция.

\end{conseq}

\begin{theorem}{}
$f: A \to B$ f "--- обратима, тогда $f^{-1}$ обратима и $(f^{-1})^{-1} = f$
\end{theorem}

\begin{proof}
$f \circ f^{-1} = id_{B}$

$f^{-1} \circ f = id_A \Ra f$ "--- обратное к $f^{-1}$

В силу единственности обратного $(f^{-1})^{-1} = f$
\end{proof}
\section{Тождественное отображение}

\begin{Def}
$A, id_{A}: A \to A$

$\forall a \in A id_A(a) = a$

$id_A$ ~--- тождественное  отображение множетсва A.

$\Gamma_{id_A}$ = диагональ $A \times A \{(a, a)| a \in A\}$ 
\end{Def}

\begin{theorem}{}
$f: A \to B$

$f \circ id_A = f =  id_B \circ f$
\end{theorem}

\begin{proof}

Области определения и назначения совпадают. 

$\forall y \in B, id_{B}(y) = y$

$a \in A$

$(f \circ id_A)(a) = f(id_A(a)) = f(a)$

$a \in A$

$(id_B \circ f)(a) = id_B(f(a)) = f(a)$

\end{proof}
\section{Равносильность инъективности и обратимости слева}

\begin{theorem}{}
Пусть $f:A \to B$ и $A \ne \varnothing$. Тогда $f$ обратима слева $\iff$ $f$ инъективна.
\end{theorem}

\begin{proof}
\begin{enumerate}
\item $\Ra$

$\exists g \colon g \circ f = id_A \Ra f$ инъективно.

\item $\La$

Пусть $C = f(A)$. Построим $h_1: C \to A$ такое, что
\[(c, a) \in \Gamma_{h_1} \Lra (a, c) \in \Gamma_{f}\]. Проверим, что это график:

\begin{enumerate}
\item Определённость для $c \in C$:
\begin{gather*}
\forall c \in C, \exists a \in A \colon (a, c) \in \Gamma_{f}; \\
\forall c \in C, \exists a \in A \colon (c, a) \in \Gamma_{h_1}; \\
\end{gather*}
\item Однозначность. Знаем, что $f$ инъективно. 
\begin{gather*}
\forall a_1, a_2 \in A, \exists b \in B \colon (a_1, b) \in \Gamma_{f} \wedge (a_2, b)\in \Gamma_{f} \Ra a_1 = a_2; \\
\forall a_1, a_2 \in A, \exists b \in C \colon (a_1, b) \in \Gamma_{f} \wedge (a_2, b)\in \Gamma_{f} \Ra a_1 = a_2; \\
\forall a_1, a_2 \in A, \exists b \in C \colon (b, a_1) \in \Gamma_{h_1} \wedge (b, a_2)\in \Gamma_{h_1} \Ra a_1 = a_2;
\end{gather*}
\end{enumerate}

$\Ra \Gamma_{h_1}$ "--- график.

Теперь построим $h: B \to A$. Для этого выберем произвольный $a \in A$ и положим:

$h(b) = \begin{cases} h_1(b), & \text{если~} b \in C\\ a, &\text{если~} b \notin C\end{cases}$

Проверим, что $h \circ f = id_A$. Рассмотрим $x \in A$:
\[(h \circ f)(x) = h(f(x)) = h_1(f(x)) = x\]
\end{enumerate}
\end{proof}


\noindent \underline{\hbox to 1\textwidth{{ } \hfil{ } \hfil{ } }}
\end{document}
