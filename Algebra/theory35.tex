\section{Алгебраическая форма записи комплексного числа. Комплексное сопряжение.
Свойства комплексного сопряжения.}

$\R \mapsto \C:~a \mapsto (a, 0)$ - инъективный гомоморфизм колец: \\
$$
\begin{cases}
	\phi(a+b) = \phi(a) + \phi(b) \\
	\phi(ab) = \phi(a) * \phi(b):\quad(a, 0)*(b, 0) = (ab - 0, 0 + 0) = (ab, 0)
\end{cases}
$$

$\C \supseteq \phi(\R) = \{(a, 0) | a \in \R\}$

$\phi(\R) \cong \R$, поэтому говорят, что $\R \subseteq \C$, имея в виду, что $\phi(\R) \subseteq \C$

$i = (0, 1) \Ra i^2 = (-1, 0)$
\begin{Def}
	$(a, b) = (a, 0)*(1, 0) + (b, 0)*(0, 1) = a + bi$ - алгебраическая запись числа.\\
	\hspace*{1cm}$a$ называется вещественной частью комплексного числа($a = Re(z), z \in \C$) \\
	\hspace*{1cm}$b$ называется мнимой частью комплексного числа($b = Im(z), z \in \C$) 
\end{Def}  

\begin{Def}
	$z \in \C,~z = a + bi,~a, b \in \R$ \\
	\hspace*{1cm} $\overline{z}$ называется комплексно сопряжённым с $z$, если $\overline{z} = a - bi$ 
\end{Def}
\begin{Rem}
	Сопряжение $\equiv$ симметрия относительно вещественной оси. \\
\begin{center}
\def\svgwidth{6.0cm}
\input{theory35-vector.pdf_tex}
\end{center}
\end{Rem}

\underline{Cвойства:}
\begin{itemize}
\item[1.] $\overline{\overline{z}} = z$
\item[2.] $z = \overline{z} \Lra z \in \R$
\item[3.] $\overline{z_1 + z_2} = \overline{z_1} + \overline{z_2}$
\item[3'.] $\overline{z_1 + z_2 + \dots + z_n} = \overline{z_1} + \overline{z_2} + \dots + \overline{z_n}$ (По индукции из св-ва 3.)
\item[4.] $\overline{z_1*z_2} = \overline{z_1} * \overline{z_2}$
\item[4'.] $\overline{z_1 * z_2 * \dots * z_n} = \overline{z_1} * \overline{z_2} * \dots * \overline{z_n}$ (По индукции из св-ва 4.)
\item[5.] $f \in \R[x]~f = a_0 + a_1x + a_2x^2 + \dots + a_nx^n$ Тогда: $\overline{f(z)} = f(\overline{z})$
\item[6.] \begin{itemize}
			\item[\bullet] $z + \overline{z} \in \R$ 
			\item[\bullet] $z*\overline{z} \in \R,~z*\overline{z} \ge 0$
			\item[\bullet] $z*\overline{z} = 0 \Lra z = 0$
		 \end{itemize}	
		 Два последних пункта следуют из того, что $z*\overline{z} = a^2 + b^2$
\begin{proof}
Только 5 свойство:
$f(z)=a_0 + a_1z + \dots + a_nz^n$

$\overline{f(z)} = \overline{a_0 + a_1z + \dots + a_nz^n} = \overline{a_0} + \overline{a_1z} + \dots + \overline{a_nz^n} = \overline{a_0} + \overline{a_1} \cdot \overline{z} + \dots + \overline{a_n} \cdot \overline{z^n} = a_0 + a_1\overline{z} + \dots + a_n\overline{z^n} = a_0 + a_1\overline{z} + \dots + a_n{\overline{z}}^n = f(\overline{z})$
\end{proof}   
\\

$\overline{z}$(Сопряжение): $\C \mapsto \C$ - гомоморфизм из $\C$ в $\C$:\\
$$
	\begin{cases}
	\overline{z_1 + z_2} = \overline{z_1} + \overline{z_2} \\
	\overline{z_1 \cdot z_1} = \overline{z_1} \cdot \overline{z_2}
	\end{cases}
$$

$\overline{z} \circ \overline{z} = id \Ra $ сопряжение - нетождественный изоморфизм из $\C$ на себя(автоморфизм).
\begin{Def}
	Автоморфизм - изоморфизм поля с самим собой.
\end{Def}
\item[7.] $z \ne 0,~z \cdot \overline{z} = |z|^2,~|z| \ne 0$(т.к. $z \ne 0$) \\
$$ z \cdot \frac{\overline{z}}{|z|^2} = 1 \Ra \boxed{z^{-1} = \frac{\overline{z}}{|z|^2} = \frac{a - bi}{a^2 + b^2}} $$ \\
PS: определение и проч. про модуль в следующем вопросе.
\end{itemize}

