\section{Равносильность инъективности и обратимости слева}

\begin{Def}
$f: A \to B$. Тогда $f$ "--- инъективное отображение (инъекция), если:

\begin{enumerate}
\item $\forall a_1, a_2 \in A, \exists b\colon (a_1, b) \in \Gamma_f \wedge (a_2, b) \in \Gamma_f \Ra a_1 = a_2$
\item $\forall a_1, a_2 \in A\colon f(a_1) = f(a_2) \Ra a_1 = a_2$
\end{enumerate}

Обозначается $f: A \rat B$.
\end{Def}

\begin{Def}
Отображение $f: A \to B$ назывется сюръективным (сюрекцией, или <<отображение \textit{на} $B$>>), если:
\[\forall b \in B, \exists a \in A\colon b = f(a)\]

Обозначние: $f: A \thra B$
\end{Def}

\begin{Def}
$f$ называется биективным (или биекцией), если $f$ и сюръективно, и инъективно.

Обозначение: $f: A \thrat B$.
\end{Def}

\begin{Def} 
Образ $C \subset A$: $f(C) = \{b \in B \mid \exists c \in C b = f(c)\}$.

Полный прообраз $D \subset B$: $f^{-1}(D) = \{a \in A \mid f(a) \in D\}$.
\end{Def}

$f(f^{-1}(D)) \subseteq D$, но может не совпадать.

$f$ инъективно $\iff$ прообраз любого одноэлементного множества содержит не более одного элемента.

$f : A \to B$ сюръективно $\iff$ $f(A) = B$.

\begin{theorem}{}
$f:A \to B$, $g:B \to A$. Если $g \circ f = id_A$, то $f$ "--- инъективно, $g$ "--- сюръективно.
\end{theorem}

\begin{proof}
\begin{enumerate}
\item Пусть $a_1, a_2 \in A\colon f(a_1) = f(a_2)$. Тогда:
\begin{gather*}
g(f(a_1)) = g(f(a_2)); \\
(g \circ f)(a_1) = (g \circ f)(a_2); \\
id_A(a_1) = id_A(a_2); \\
a_1 = a_2; \\
\Ra f \text{ "--- инъекция}
\end{gather*}

\item Пусть $a \in A$ и $b=f(a)$, тогда:
\begin{gather*}
g(f(a)) = (g \circ f)(a) = id_A(a) = a; \\
a = g(b); \\
\Ra \forall a \in A, \exists b \in B\colon a = g(b) \Ra g \text{ "--- сюръекция}
\end{gather*}

\end{enumerate}
\end{proof}

\begin{theorem}{}
Пусть $f:A \to B$ и $A \ne \varnothing$. Тогда $f$ обратима слева $\iff$ $f$ инъективна.
\end{theorem}

\begin{proof}
\begin{enumerate}
\item $\Ra$

$\exists g \colon g \circ f = id_A \Ra f$ инъективно.

\item $\La$

Пусть $C = f(A)$. Построим $h_1: C \to A$ такое, что
\[(c, a) \in \Gamma_{h_1} \Lra (a, c) \in \Gamma_{f}\]. Проверим, что это график:

\begin{enumerate}
\item Определённость для $c \in C$:
\begin{gather*}
\forall c \in C, \exists a \in A \colon (a, c) \in \Gamma_{f}; \\
\forall c \in C, \exists a \in A \colon (c, a) \in \Gamma_{h_1}; \\
\end{gather*}
\item Однозначность. Знаем, что $f$ инъективно. 
\begin{gather*}
\forall a_1, a_2 \in A, \exists b \in B \colon (a_1, b) \in \Gamma_{f} \wedge (a_2, b)\in \Gamma_{f} \Ra a_1 = a_2; \\
\forall a_1, a_2 \in A, \exists b \in C \colon (a_1, b) \in \Gamma_{f} \wedge (a_2, b)\in \Gamma_{f} \Ra a_1 = a_2; \\
\forall a_1, a_2 \in A, \exists b \in C \colon (b, a_1) \in \Gamma_{h_1} \wedge (b, a_2)\in \Gamma_{h_1} \Ra a_1 = a_2;
\end{gather*}
\end{enumerate}

$\Ra \Gamma_{h_1}$ "--- график.

Теперь построим $h: B \to A$. Для этого выберем произвольный $a \in A$ и положим:

$h(b) = \begin{cases} h_1(b), & \text{если~} b \in C\\ a, &\text{если~} b \notin C\end{cases}$

Проверим, что $h \circ f = id_A$. Рассмотрим $x \in A$:
\[(h \circ f)(x) = h(f(x)) = h_1(f(x)) = x\]
\end{enumerate}
\end{proof}
