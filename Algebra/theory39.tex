\section{Формула Муавра}

\begin{theorem}{}
$z = r(\cos \phi + i \sin \phi):\quad r = |z|, \phi = arg(z), z \ne 0, n \in \Z$ \\
\hspace*{1cm}Тогда $z^n = r^n(\cos{\phi n} + i \sin{ \phi n})$
\end{theorem} 
\begin{proof}
\begin{itemize}
\item Если $n \in \N$, то очевидно.
\item $n = 0$, 1 = 1.
\item $n = -1:$ 
$$z^{-1} = \frac{\overline{z}}{|z|^2} = \frac{r\overline{(\cos \phi + i \sin \phi)}}{r^2} = r^{-1}(\cos \phi - i \sin \phi) = r^{-1}(\cos{-\phi} + i \sin{-\phi})$$ 
чтд.
\item Общий случай: $n < 0$.
$$z^n = (z^{-1})^n = (r^{-1}(\cos{-\phi} + i \sin{-\phi}))^{|n|} = r^{-|n|}( \cos(-|n| \phi) + i \sin(-|n| \phi)) = r^n(\cos{n \phi} + i \sin{n \phi})$$
чтд.
\end{itemize}
\end{proof}


