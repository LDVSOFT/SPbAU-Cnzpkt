\section{Неравенство треугольника}

\begin{theorem}{} $||z_1| - |z_2|| \le |z_1 \pm z_2| \le |z_1| + |z_2|$ \\
\hspace*{1cm}Причём:
\begin{enumerate}
	\item $|z_1 + z_2| = |z_1| + |z_2| \Lra$ $z_1$ и $z_2$ лежат на одном луче, проведённом из начала координат.
	\item $|z_1 - z_2| = |z_1| + |z_2| \Lra$ $z_1$ и $z_2$ лежат на дополнительных лучах, проведённых из начала координат.
	\item $|z_1 + z_2| = ||z_1| - |z_2|| \Lra$ $z_1$ и $z_2$ лежат на дополнительных лучах, проведённых из начала координат.
	\item $|z_1 - z_2| = ||z_1| - |z_2|| \Lra$ $z_1$ и $z_2$ лежат на одном луче, проведённом из начала координат.
\end{enumerate} 
\end{theorem}
\begin{proof}
Так как все величины неотрицательны, то исходной неравенство равносильно следующему:
$$||z_1|-|z_2||^2 \le |z_1 \pm z_2|^2 \le (|z_1| + |z_2|)^2$$
$$z_1 = r_1(\cos \phi_1 + i \sin \phi_1),\quad z_2 = r_2(\cos \phi_2 + i \cos \phi_2)$$
$$|z_1 \pm z_2|^2 = (r_1 \cos \phi_1 \pm r_2 \cos \phi_2)^2 + (r_1 \sin \phi_1 \pm r_2 \sin \phi_2)^2 = r_1^2 \cos^2 \phi_2 \pm 2r_1r_2 \cos \phi_1 \cos \phi_2 +$$
$$+ r_2^2 \cos^2 \phi_2 + r_1^2 \sin^2 \phi_1 \pm 2r_1r_2 \sin \phi_1 \sin \phi_2 + r_2^2 \sin^2 \phi_2 = r_1^2 + r_2 ^2 \pm 2r_1r_2(\cos \phi_1 \cos \phi_2 + \sin \phi_1 \sin \phi_2) =$$
$$= r_1^2 + r_2^2 \pm 2r_1r_2 \cos(\phi_1 - \phi_2)$$
$$|z_1 \pm z_2|^2 \le r_1^2 + r_2^2 + 2r_1r_2 = (r_1 + r_2)^2 = (|z_1| + |z_2|)^2$$
\begin{enumerate}
	\item Если знак +, то равенство $\Lra \cos(\phi_1 - \phi_2) = 1 \Lra \phi_1 - \phi_2 = 2 \pi k, k \in \Z \Lra \phi_1 = \phi_2 + 2 \pi k, k \in \Z$, то есть $z_1$ и $z_2$ лежат на одном луче.
	\item Если знак +, то равенство $\Lra \cos(\phi_1 - \phi_2) = -1 \Lra \phi_1 - \phi_2 = 2 \pi k + \pi, k \in \Z \Lra \phi_1 = \phi_2 + 2 \pi k + \pi, k \in \Z$, то есть $z_1$ и $z_2$ лежат на дополнительных лучах.
\end{enumerate}
Левое неравенство - \underline{Упражнение.}
\end{proof}