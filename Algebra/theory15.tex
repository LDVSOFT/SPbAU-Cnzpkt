\section{Классы смежности}

Есть группа $G$ и ее подгруппа $H$. Введем отношение $\sim$: $a\sim b \iff a^{-1}b \in H$.
Докажем, что это отношение "--- отношение эквивалентности:

\begin{gather*}
a^{-1}a = e \in H \Ra a \sim a; \\
a \sim b \Lra a^{-1}b \in H \Lra (a^{-1}b)^{-1} = b^{-1}a \in H \Lra b \sim a; \\
\left.
\begin{align*}
  a \sim b \Lra a^{-1}b \in H; \\
  b \sim c \Lra b^{-1}c \in H; 
\end{align*}
\right\} \Ra a^{-1}c = a^{-1}bb^{-1}c \in H \Ra a \sim c;
\end{gather*}

Рассмотрим класс эквивалентности элемента $a$:
\begin{align*}
[a] &= \{b \in G \mid a^{-1}b \in H\} = \\
    &= \{b \in G \mid \exists h \in H \colon a^{-1}b = h\} = \\
    &= \{b \in G \mid \exists h \in H \colon b = ah\}
\end{align*}

Словами: класс эквивалентности $a$ "--- это образ отображения $f: H \to G$, где $f(h)=ah$.

\begin{Def}
	$aH = [a]$ "--- левый класс смежности по подгруппе $H$, $Ha$ "--- правый класс смежности по подгруппе $H$ (определяется симметрично).
\end{Def}
\begin{Rem}
	Левые классы смежности для разных элементов или не пересекаются, или совпадают, так как $aH$ "--- класс эквивалентности. Аналогично для
	правых классов.
\end{Rem}

\begin{exmp}
$G = S_3$ "--- группа перестановок из трех элементов. Пусть $H = \left<(12)\right> = \{e, (12)\}$.
Тогда $(13)H = \{(13), (123)\}, H(13) = \{(13), (132)\}$.
\end{exmp}
