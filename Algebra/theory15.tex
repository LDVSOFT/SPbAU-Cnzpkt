\section{Классы смежности}

Есть группа $G$ и ее подгруппа $H$. Введем отношение $\~: a\~b \Lra a^{-1}b \in H$.
Докажем, что это отношение - отношение эквивалентности.

$$a^{-1}a = e \in H \Ra a \sim a$$
$$a \sim b \Lra a^{-1}b \in H \Lra (a^{-1}b)^{-1} = b^{-1}a \in H \Lra b \sim a$$
$$a \sim b \Lra a^{-1}b \in H, b ~ c \Lra b^{-1}c \in H \Ra a^{-1}c = a^{-1}bb^{-1}c \in H \Ra a \sim C$$

$$[a] = \{b: a^{-1}b \in H\} = \{b: \exists h \in H a^{-1}b = h\} = \{b: \exists h \in H b = ah\} = aH$$

\begin{Def}
	$aH$ - левый класс смежности по подгруппе $H$, $Ha$ - правый класс смежности по подгруппе $H$
\end{Def}
\begin{Rem}
	Левые(правые) смежности не пересекаются, или же совпадают, так как $aH$ - множество элементов, эквивалентных $a$
\end{Rem}

Пример: $G = S_3$ - группа перестановок из трех элементов, $H = <(12)> = \{e, (12)\}$ \\
$(13)H = \{(13), (123)\}, H(13) = \{(13), (132)\}$.                 \\
