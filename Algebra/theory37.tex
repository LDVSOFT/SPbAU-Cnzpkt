\section{Аргумент комплексного числа. Тригонометрическая форма записи. Арифметические операции над комплексными числами в тригонометрической форме.}

\begin{center}
\def\svgwidth{6.0cm}
\input{theory37-vector.pdf_tex}
\end{center}

$z \in \C,~z = a + bi \Ra (a, b)$ - координата в декартовой системе координат.

В полярной системе координат два других параметра: $r$ - радиус вектор, $\phi$ - угол.

$$
\begin{cases}
	a = r\cos(\phi) \\
	b = r\sin(\phi)
\end{cases}
$$  

Пары $(r, \phi)$ и $(r, \phi + 2 \pi k)$ определяют одну и ту же точку на комплексной плоскости.

\begin{Def}
	$\phi$ - аргумент $z$($arg z$) \\
	\hspace*{1cm}Для любого вещественного числа $arg = 0$.
\end{Def}

$\R, \sim: \phi_1 \sim \phi_2 \Lra \phi_1 - \phi_2 = 2 \pi k, k\in \Z$

\underline{Упр.}: Доказать, что $\sim$ отношение эквивалентности.

\begin{Def}
	$[\phi] = \{\phi + 2 \pi k | k \in \Z\}$
	\hspace*{1cm}Arg z = $[\phi] \Lra arg z = \phi$
\end{Def}

Пусть $z = a + bi |z| = \sqrt(a^2 + b^2)$. arg z = ?: \\
\begin{enumerate}
	\item $a > 0$ 
	$$ \frac{b}{a} = \tg \phi,~\phi \in (-\pi/2, \pi/2) \Ra arg z = \arctg(\frac{b}{a}) $$
	\item $a < 0$ 
	$$ \phi \in (\pi/2, 3 \pi/2) \Ra arg z = \pi + \arctg(\frac{b}{a}) $$
	\item $a = 0, b > 0$
	$$ arg z = \pi/2$$
	\item $a = 0, b < 0$
	$$ arg z = -\pi/2$$
\end{enumerate}

\begin{Def}{Тригонометрическая форма записи числа} \\
\hspace*{1cm}$z = a + bi = r\cos \phi + ir \sin \phi = r(\cos \phi + i \sin \phi)$, где $r$ - модуль($r \ge 0$), а $\phi$ - аргумент комплексного числа.
\end{Def}

$|\cos \phi + i \sin \phi| = \sqrt{\cos^2 \phi + \sin^2 \phi} = 1$

\underline{Свойство:} $z_1 = r_1(\cos \phi_1 + i \sin \phi_1),~z_2 = r_2(\cos \phi_2 + i \sin \phi_2)$ 

Тогда:
$$z_1z_2 = r_1r_2(\cos \phi_1 \cos \phi_2 - \sin \phi_1 \sin \phi_2 + i(\cos \phi_1 \sin \phi_2 + \cos \phi_2 \sin \phi_1)) = r_1r_2(\cos(\phi_1 + \phi_2) + i \sin(\phi_1 + \phi_2))$$
$$\boxed{|z_1z_2| = r_1r_2 = |z_1||z_2|,\quad Arg(z_1z_2) = Arg(z_1) + Arg(z_2)}$$

