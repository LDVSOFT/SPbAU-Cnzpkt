\section{Симетрическая группа. Порождение симметрической группы транспозициями}


$S_n$ -- биекции на $\{1, \dotsc  , n\}\\ S_n $-- симметрическая группа(группа перестановок) степени $n.\\ \tau, \sigma \in S_n\\$ В произведении $\sigma\tau$ первой выполняется перестановка $\tau$.\\

\begin{Def}
	Цикл $(i_1, i_2, \dotsc, i_k)$ -- это перестановка $\sigma$, такая что 
	$\sigma(i_1) = i_2$, 
	$\sigma(i_2) = i_3$, 
	$\dotsc$\\
	$\sigma(i_k) = i_1$ 
	и $\sigma(j) = j$, где $j \notin \{i_1, \dotsb, i_k\}, 1 \le j \le n$
\end{Def}

$(i_1, i_2, \dotsc, i_k) = (i_2, i_3, \dotsc, i_k) = \dotsc = (i_s, i_{s+1}, \dotsc, i_k, i_1, \dotsc, i_{s - 1})$\\
$k$ -- длина цикла. Порядок цикла длины $k$ равен $k$.
$(i_1, i_2, \dotsc, i_k)^k = id$\\

\begin{Def}
	Транспозиция - цикл длины 2.
	$(i, j)$ -- $i$ и $j$ меняются местами, остальные остаются на месте.
\end{Def}

\begin{Def}
	$(i_1, i_2, \dotsc, i_k), (j_1, j_2, \dotsc, j_s)$ -- циклы. Эти циклы называются незацепляющимися(непересекающимися, если $\{i_1, i_2, \dotsc, i_k\} \bigcap \{j_1, j_2, \dotsc, j_s\} = \varnothing$)
\end{Def}

Их легко перемножать. Если $\sigma, \tau$ -- непересекающиеся циклы, то $\sigma\tau = \tau\sigma$.

\begin{theorem}{$S_n$ порождается транспозициями}
\end{theorem}
\begin{proof}
Покажем, что любой цикл $(i_1, i_2, \dotsc, i_k)$ есть произведение транспозиций $(i_1 i_2)(i_2 i_3)\dotsc(i_{k - 1}i_k)$. Заметим, что если применить это произведение к перестановке(вот просто взять и последовательно применить справа налео все транспозиции аккуратно), то мы и получим тот же самый результат, как и от применения исходного цикла.\\
Осталось лишь заметить, что каждая из остальных $n - k$ точек любо неподвижная, либотакже лежит на каком-то цикле - а бить циклы на произведение транспозиций мы только что научились.\\
\end{proof}