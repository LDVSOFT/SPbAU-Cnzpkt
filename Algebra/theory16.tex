\section{Теорема Лагранжа и следствия из нее}
\begin{lemma}
	$H \subset G$, $f: H \ra aH$, $f(h)=ah$, $H$ "--- подгруппа. Тогда $f$ "--- биекция. В частности, из этого будет следовать, что $|H| = |aH|$, то есть мощности всех левых классов смежности равны друг другу
\end{lemma} 
\begin{proof}
	Заметим, что это отображение "--- сюрьекция по определению $aH$ (у каждого элемента есть прообраз). Докажем, что это инъекция.

	От противного: пусть $ah_1 = ah_2$, домножим слева на $a^{-1}$, получим $h_1 = h_2$.

	Таким образом, $f$ "--- биекция.
\end{proof}

\begin{Def}
	Число левых классов смежности по $H$ называется индексом $H$ в $G$. Обозначение: $[G:H]$
\end{Def}

\begin{theorem}{Теорема Лагранжа}
	Пусть $G$ - конечная группа, возьмём элемент $a$ и его левые классы смежности $aH_{\alpha}$.
	Очевидно, что $G = \bigcup aH_{\alpha}$ и $H_{\alpha_1} \cap H_{\alpha_2} = \emptyset$.
	Тогда $|G| = [G : H] \cdot |H|$
\end{theorem}
\begin{proof}
	Все эти классы имеют одинаковую мощность, равную $|H|$ (по лемме). Тогда $|G| = [G : H] \cdot |H|$, так как эти классы не пересекаются.
\end{proof}
\begin{conseq}
	Количество правых и левых классов смежности одинаково (достаточно провести аналогичные действия для правых классов смежности)
\end{conseq}
\begin{conseq}
	Порядок любой подгруппы делит порядок конечной группы.
\end{conseq}
\begin{conseq}
	Порядок любого элемента делит порядок конечной группы (рассмотрим циклическую подгруппу, порожденную этим элементом)
\end{conseq}
\begin{conseq}
	Группа порядка $p$ (где $p$ "--- простое число) циклична, так как порядок любого элемента равен либо 1 ($e$), либо $p$ (все остальные), а тогда все элементы кроме $e$ порождают всю группу порядка $p$.
\end{conseq}
