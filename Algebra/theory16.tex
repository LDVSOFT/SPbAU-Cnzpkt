\section{Теорема Лагранжа и следствия из нее}
\begin{lemma}
	$H \subset G, f: H \ra aH, h \ra ah$. Тогда $f$ - биекция. В частности, из этого будет следовать, что $|H| = |aH|$, то есть мощности всех левых классов смежности равны друг другу \\	
\end{lemma} 
\begin{proof}
	Заметим, что это отображение - сюрьекция по определению $aH$ (есть прообраз). Докажем, что это инъекция.  \\
	Пусть $ah_1 = ah_2$, домножим слева на $a^{-1}$, получим $h_1 = h_2$, значит инъекция и сюрьекция, значит биекция. \\
\end{proof}


\begin{Def}
	Число левых классов смежности по $H$ называется индексом $H$ в $G$. Обозначение: $[G:H]$
\end{Def}

\begin{theorem}{Теорема Лагранжа}
	Пусть $G$ - конечная группа, $G = \cup aH_alpha, H_alpha_1 \cap H_alpha_2 = \emptyset$ \\
	Тогда $|G| = [G : H] * |H|$                                                                                                    \\
\end{theorem}
\begin{proof}
	Все эти классы имеют одинаковую мощность, равную $|H|$ (лемма). Тогда $|G| = [G : H] * |H|$, так как эти классы не пересекаются
\end{proof}
\begin{conseq}
	Количество правых и левых классов смежности одинаково(достаточно провести аналогичные действия для правых классов смежности)
\end{conseq}
\begin{conseq}
	Порядок любой подгруппы делит порядок конечной группы.
\end{conseq}
\begin{conseq}
	Порядок любого элемента делит порядок конечной группы (рассмотрим циклическую подгруппу, порожденную этим элементом)
\end{conseq}
\begin{conseq}
	Группа порядка $p$, $p$ - простое число, циклична(порядок любого элемента равен либо 1($e$), либо $p$(все остальные)), тогда все элементы кроме $e$ порождают группу порядка $p$ 
\end{conseq}