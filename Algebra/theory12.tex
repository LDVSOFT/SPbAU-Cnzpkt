\section{Два эквивалентных определения подгруппы, порожденной множеством}
\begin{Def}
	Замыканием множества относительно операции - множество всех элементов, получаемых из элементов этого множества применением данной операции. \\
	Аналогично определяется замыкание относительно множества операций \\
	В частности, для группы это множество будет выглядеть вот так: $\{a_1^{z_1}a_2^{z_2}\dots | a_i \in A, z_i = \pm 1 \}$, где $A$ - данное множество
\end{Def}

\begin{Def}
	Подгруппа, порожденная множеством - замыкание этого множества относительно операций $\cdot$ и взятия обратного элемента \\
\end{Def}

\begin{Def}
	Подгруппа, порожденная множеством - пересечение всех подгрупп, содержащих это множество \\
\end{Def}

\begin{Rem}
	Предыдущие два определения действительно задают подгруппы, именно это мы доказывали в предыдущих теоремах  \\
\end{Rem}

\begin{theorem} {Равносильность определений подгруппы, порожденной множеством}
	$M \subset G$, $G$ - группа.
	$A$ - замыкание $M$ относительно операций $\cdot$ и взятия обратного элемента.
	$B = \cap H_\alpha : H_\alpha$ - подгруппа $G$
	Тогда $A = B$ по построению множества $A$
\end{theorem}
\begin{proof}
	$B \subset A$, так как $A$ - подгруппа, содержащая $M$.
	Но тогда $B = A$ по построению.	
\end{proof}
