\section{Два эквивалентных определения подгруппы, порожденной множеством}
\begin{Def}
	Замыканием множества относительно операции "--- множество всех элементов, получаемых из элементов этого множества применением данной операции.

	Аналогично определяется замыкание относительно множества операций.
\end{Def}

В частности, для группы это замыкание $A$ будет выглядеть как: \[\{a_1^{z_1}a_2^{z_2}\dots | a_i \in A, z_i = \pm 1 \}\], где $A$ "--- данное множество.

\begin{Def}
	Подгруппа, порожденная множеством "--- замыкание этого множества относительно операций $\cdot$ и взятия обратного элемента
\end{Def}

\begin{Def}
	Подгруппа, порожденная множеством "--- пересечение всех подгрупп, содержащих это множество 
\end{Def}

\begin{Rem}
	Предыдущие два определения действительно задают подгруппы, именно это мы доказывали в предыдущих теоремах 
\end{Rem}

\begin{theorem} {Равносильность определений подгруппы, порожденной множеством}
	Пусть $M \subset G$, $G$ "--- группа, $A$ "--- замыкание $M$ относительно операций $\cdot$ и взятия обратного элемента,
    $B = \bigcap \{ H \in 2^G \mid H \supset M \land H \text{"--- подгруппа } G\}$.

	Тогда $A = B$.
\end{theorem}
\begin{proof}
	$A \subset B$, так как любая подгруппа, содержащая $M$, должна содержать замыкание $M$.

	Но так как $A$ "--- подгруппа, то $B \subset A$.

	$\Ra A = B$.
\end{proof}
